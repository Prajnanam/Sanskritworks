घृणा शङ्का भयं लज्जा जुगुप्सा चेति पञ्चमी~।\\ कुलं शीलं तथा जातिरष्टौ पाशाः प्रकीर्तिताः~॥
योगतरङ्गिणी~- १.१३
तुर्यावस्था चिदभिव्यञ्जकनादस्य वेदनं प्रोक्तम्~।\\ तद्भावनार्धचन्द्रादिकं त्रयं व्याप्य कर्तव्या~॥\\ आनन्दैकघनं यद्वाचामपि न गोचरो नॄणाम्~।\\ तुर्यातीतावस्था सा नादान्तादिपञ्चके भाव्या~॥ (वरिवस्यारहस्यम् १.४२)
आत्मनः ज्ञानप्रकाशस्वरूपत्वात् ज्योतिरिति व्यपदेशः~।
नित्याषोडशिकार्णवे  अर्थरत्नावलीटीका ५.१९/२०~। एवं योगतरङ्गिणी~।
घेरण्डसंहिता~(३.३)
'तिस्रः कोट्योऽर्धकोटी च रोमाणि व्यावहारिके’ (गरुडपुराणम्) इत्यादिना तत्तत्सङ्ख्या प्रोक्ता~।
शैवतन्त्रे प्रसिद्धम्~।
परमात्मना अभिन्नस्य जीवात्मनः अभेदाज्ञानम्~।
नित्याषोडशिकार्णवः सेतुबन्धटीका. ४.१६
धर्मः ज्ञानं एेश्वर्यं वैभवम्~।
मुक्तिकोपनिषत्-२.१
सम्यक् कल्पनं~- ‘इदं शोभनम्’ इति~। 
मदीयत्वभिमानः
गङ्गादितीर्थस्नानं पापनिवारकं भवति~।
अभिभवः-तिरस्कारः
गोभिः = किरणैः~। गाः = जलानि~। गोपतिः = सूर्यः~।
तद्यथा-कस्मिंश्चिदरण्ये कपोतदम्पती वसतः स्म~। अतिस्नेहानुरागिणोस्तयोः कालेनापत्यानि व्यजायन्त~। तानि अपत्यानि अतिप्रेम्णा तौ पुपुषतुः~। कदाचिदाहारान्वेषणार्थं गतयोस्तयोः पक्षिघाती व्याध आगत्य तानर्भकान् जाले बबन्ध~। संगृहीताहारा कपोती स्वनीडमागत्य अपत्यानि अपश्यन्ती सोत्कण्ठा दूरे जालं दृष्ट्वा अन्वधावत्~। स्नेहपरवशा धावन्ती सा स्वयमपि जाले पपात~। तदा आगतः कपोतः त्रैवर्गिकमात्मनः संसारं विच्छिद्रतामापन्नं दृष्ट्वा अतीवखिन्नो बभूव~। निर्दारपुत्रस्य आत्मनः जीवितं वृथामन्यमानः सः स्वमपि जाले स्वेच्छया अपातयत्~। व्याधः सर्वानपि तान् गृहीत्वा ससन्तोषं गृहं गतः~। एवम् अशान्तात्मा कुटुम्बी कपोतवत् कुटुम्बं पुष्णन् सानुबन्धः अवसीदति~।~।
सर्पविशेषः
बडिशं~- मत्स्यवेधनम्~।
पिङ्गला काचिद्वेश्या विशेषतोऽलङ्कृत्य बहिर्द्वारि विटान् निरीक्षते स्म~। वित्तदानपि बहूनागतान् कान्तान् बहुद्रव्यमपेक्षमाणा नालक्षयत्~। बहुद्रव्यदः आसायमपि नैव कोऽपि समागतः~। तदाऽऽत्मानं विशेषतो निन्दन्ती मनः परमात्मनि निधाय आशां त्यक्त्वा सुखं शिश्ये~॥
क्रौञ्चः(पक्षिविशेषः) क्वचित् दैवाद् उपलब्धं मांसशकलं गृहीत्वा भक्षणोन्मुखः आसीत्~। बलिनः इतरे पक्षिणः मांसम् अप्राप्य, तञ्च क्रौञ्चं दृष्ट्वा कुपिताः असूयापराः क्रौञ्चात् मांसम् अपहर्तुं प्रयतिरे~। तदा मांसशकलेन सह क्रौञ्चः धावितुमारेभे~। साक्रोशं तमनुधावन्तः इतरे पक्षिणः तं क्रौञ्चं हन्तुम् उपचक्रमिरे~। झटिति क्रौञ्चः मांसशकलं परितत्याज~। अनुपदं क्रौञ्चं विहाय पतितं मांसशकलमनुधावन्तः पक्षिणः परस्परं कलहम् आरेभिरे~। क्रौञ्चः आपद्विनिर्मुक्तः स्वस्थोऽभवत्~।
सम्मानेन हृष्टो न स्यात्, अवमानेन खिन्नो न भवेत्~।
काचित् कुमारी पित्रादिपरवशा गृहागतानतिथीन्  आत्मानं वृणानान् प्रच्छन्नं भोजयितुमिच्छन्ती व्रीहिनवहन्तुं प्रचक्रमे~। तदा तस्याः प्रकोष्ठस्थाः शङ्खाः (भूषणभूताः) प्रचुक्रुशुः~। पित्रादयो जानीयुस्तेन चानर्थो भवेदिति सा चैकैकशः शङ्खान् भङ्क्त्वा स्वकर्म समापितवती~॥
 कश्चित् इषुकारः  शरनिर्माणे प्रवृत्तः एकाग्र्येण तत्रैव कार्ये अवहितचित्तः आसीत्~। दैवात् तत्र आगतः राजा तत्कार्यं परिशीलयन् पार्श्वे चिरं स्थितः~। इषुकारस्तु स्वकार्यैकदृष्टिः नान्यत् किञ्चित् वेद, राजानं नैव ददर्श~॥
 पेशस्कृतं~- कीटविशेषं~। पेशो रूपान्तरम् करोतीति पेशस्कृत्~। अनेन गृहीतस्य कीटस्य रूपान्तरं भवतीति~॥
खट्वाशयनं न कुर्यात्, भूमौ शयनं कुर्यादित्यर्थः
(तन्त्रराजः)
भक्तिश्रद्धादिप्रश्नधर्मोल्लङ्घनम् अन्यायः~।
अर्थस्वीकारेऽपि अध्यापकेन ‘यदि एतावन्मह्यं दीयते तदा एतावद् अध्यापयामि’ इति नियमस्तु न करणीयः~।
यत्र भूमौ बीजम् उप्तं न प्ररोहति स ऊषरः~।
हस्ते शिवं गुरुं ध्यात्वा जपेन्मूलाङ्गमालिनीम्~।\\  गुरुः स्पृशेच्छिष्यतनुं स्पर्शदीक्षा भवेदियम्~॥५३॥
 निमील्य नयने ध्यात्वा परतत्त्वप्रसन्नधीः~।\\  सम्यक् पश्येद्गुरुः शिष्यं दृग्दीक्षा च भवेत् प्रिये~॥५५॥
 चित्तं तत्त्वे समाधाय परतत्त्वोपबृंहितान्~।\\  उच्चरेत् संहतान्मन्त्रान् वाग्दीक्षेति निगद्यते~॥५४॥
धर्माधर्मविवेकः कर्तव्याकर्तव्यविवेकः न्याय्यान्याय्यविवेकः यस्यास्ति सः~।
शुद्धं चरित्रं यस्य सः~।
साधूनां बलं पुण्यमेव, न तु वित्तपुत्रमित्रकलत्रादि~।
तातस्तु जनकः पिता (अमरः)~।
यया जीविनाम् अन्तरङ्गं क्लेशदायकविषयवासनाभ्यो विमुक्तं भवति सा विद्या सर्वोत्कृष्टा~।
यस्य आचरणेन जीविनाम् अन्तरङ्गं प्रशान्तं भवति तत् कर्म सर्वोत्कृष्टम्~।
दया प्रधानं यस्य सः~।
अनिच्छयाऽपि अवश्यमनुसरणीयः धर्मः~।
गुरूणां विषये या उपेक्षा तत्समानं श्रेयोविघातकं कृत्यम् अन्यत् न वर्तते~।
विषयेषु रूपरसादिषु, विषयासक्तेषु जनेषु च स्नेहः अज्ञानं जनयति वर्धयति च~।
तृष्णा सांसारिकबन्धं द्रढयति वर्धयति च~। 
पद्मपत्रगतजलबिन्दुरिव यौवनं धनम् आयुश्च चिरं न तिष्ठति~।
चन्द्रकिरणा  बाह्यतापमस्माकं दूरीकुर्वन्ति आह्लादञ्च जनयन्ति तथा सज्जनाः अस्माकम् अन्तरङ्गस्य तापं दूरीकृत्य आनन्दं जनयन्ति~। 
 पराधीनता महते दुःखाय भवति~।
 विषयाणां विषयलोलुपजनानाञ्च सङ्गः पर्यवसाने अनर्थं जनयति ‘सङ्गात् सञ्जायते कामः  कामात् क्रोधोऽभिजयते’ इत्यादिरीत्या~। अतः तेषां सङ्गस्य परित्यागात् अनर्थनिवृत्या चित्तशान्तिः लभ्यते~।
 येन वचनेन प्राणिनां हितं भवेत् तत् सत्यम्~।
असवः = प्राणाः
अनर्थः फलं यस्मात् सः~। मानः = ‘अहमुत्कृष्ट’ इति चिन्तनम्~।
सुखं ददाति इति सुखदा~।
त्यागी स्वस्य अन्येषां च विपत्तिनिवारणसमर्थो भवति, न तु भोगी तथा~।
यो जनः आपत्तिकाले यद्वस्तु साहाय्यं वा औत्कट्येन आकाङ्क्षति तदा तस्य वस्तुनः साहाय्यस्य वा अल्पस्यापि तस्मै दानम् अपरिमितफलप्रदं भवति~।
तितिक्षा = पराभ्युदयसहनम्
कुत्सितकर्मणि प्रवृत्तः  जनः ‘अन्ध’ इत्युच्यते विवेकदृष्टिशून्यत्वात्~।
‘तीर्थम् ऋषिजुष्टजले, गुरौ’ (अमरकोषः)~। पुण्यसञ्चयाय पापक्षयाय यो जनः वृद्धावस्थायां काश्यादितीर्थयात्रां करोति,  गुरुं भजते,  उपदेशं प्राप्य भक्तिज्ञानसाधनाय प्रयततेस ‘पङ्गुः’
अन्धस्य भौतिकदृष्टिमात्रं नष्टं, विषयलोलुपस्य तु प्रज्ञादृष्टिरेव नष्टा भवति~।
विषयाः रूपरसादयः चक्षुरादीन्द्रियं मनश्च आकृष्य मनुष्याणाम् आत्मप्रज्ञाम्  अपहरन्ति~। विषयाणाम् उपभोगेन इन्द्रियाणां मनसः बुद्धेश्च सात्त्विकं बलमपि क्षीयते~।
प्रत्युपकारस्य फलस्य च प्रतीक्षां विना कृतम्~।
नीचान् दुष्टान् जनान् यः परिचरति उपकरोति आश्रयति वा~।
सम्पत्तौ सत्यामपि दानाय मनः उत्सुकं न भवतीति यत् तत् शोचनीयम्~।
कमला~- लक्ष्मीः~। अनलसम् आलस्यरहितं चित्तं यस्य सः अनलसचित्तः~।
कामपीडया चाञ्चल्यं यस्य न भवति  स शूरेषु अग्रगण्यः~।
स्वयम् उद्वेगरहितः भवेत्~,  अन्येषु उद्वेगं न जनयेच्च~।
तुष्टिः = तृप्तिः
ऐहिकशरीरकारणीभूतं पूर्वकृतं कर्म यावत्तिष्ठति तावत् शरीरं जीर्णं रोगग्रस्तमपि जीवत्येव~। नष्टे कर्मणि तु शरीरं सुदृढमपि न जीवति~।
 अाकरः~- श्रीशङ्कराचार्यविरचिता प्रश्नोत्तरमालिका~।
