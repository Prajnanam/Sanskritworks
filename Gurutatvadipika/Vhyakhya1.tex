\begin{verse}[\versewidth]
\end{verse}
{\fontsize{14}{16}\selectfont \v  अष्टविधपाशैः बद्धाः जीवाः पशव इत्युच्यते~। ईश्वरः गुरुरूपेण आगत्य पशुभ्यः तेभ्यः दीक्षां प्रदाय तान् जीवान्  पाशमुक्तान् करोति~। }
{\fontsize{14}{16}\selectfont \v प्राकृताः = नीचाः पामराः}
{\fontsize{14}{16}\selectfont \v  उत्पादकः - जनकः पिता~। ब्रह्मदाता - ब्रह्मप्राप्तिफलप्राकोपनयनसंस्कारदाता  आचार्यः~। उपनयनजन्यसंस्कारस्य ब्रह्म, शाश्वतं - विप्रं, मतः ब्रह्मप्राप्तिरूपं शाश्वतं फलं जनयति~।}
{\fontsize{14}{16}\selectfont \v  यस्य देवे गुरौ च पराभक्तिरुदेति स महात्मा~। तस्य वेदशास्त्राणां निगूढाः अर्थाः स्फुटं प्रतिभासन्ते~।}
{\fontsize{14}{16}\selectfont \v  मोक्षार्थिनां भोगार्थिनां ब्रह्मपदाकाङ्क्षिणां विष्णुपदाकाङ्क्षिणाम् ईश्वरपदाकाङ्क्षिणाञ्च गुरुभक्त्यैव फलं भवति,  न अन्येन उपायेन~।}
{\fontsize{14}{16}\selectfont \v  भक्तिरहितः चतुर्वेदी सन्नपि ईश्वरस्य गुरोः प्रियो न भवति~। भक्तिमान् चाण्डालोऽपि गुरोः प्रियो भवति~। गुरुभक्तिमान् यः स पूजनीयः~।  }
{\fontsize{14}{16}\selectfont \v  परतत्त्वम् आत्मा~। तदेव अर्थः प्रयोजनम्, तस्य पारं साक्षात्कारं, गच्छति प्राप्नोति यः स परतत्त्वार्थपारगः। आत्मसाक्षात्कारं प्राप्तवान् यो गुरुः स दुर्लभः~।}
{\fontsize{14}{16}\selectfont \v यः वेदस्थानां तन्त्रशास्त्रस्थानाञ्च मन्त्राणां सिद्धिं प्राप्तवान् स गुरुः दुर्लभः।}
{\fontsize{14}{16}\selectfont \v  मनुष्याणां दुःखस्य कारणं मोहः~। स च सङ्कल्पनं विकल्पनमिति द्विविधः~। क्वचिद् वस्तुनि जने वा अवलोकिते 'शोभनमिदम्' इति क्वचिच्च' 'कुत्सितम् इदम्' इति मोहः जायते~। तस्य मोहस्य मूलम् अज्ञानम्~। येन महात्मना तु आत्मतत्त्वं साक्षाकृतम्, तस्य अज्ञानं विनश्यति~। अतः सङ्कल्पलनादि मोहोऽपि न भवति~। स गुरुः दुर्लभः~।}
{\fontsize{14}{16}\selectfont \v घृणादिभिः अष्टभिः पाशैः\footB\  बद्धाः जीवाः पशव इत्युच्यन्ते~। तैः पाशैः असंस्पृष्टः ईश्वरः पतिः इत्युच्यते~। जीवानां पाशविमोचकः गुरुः भवति~।}
{\fontsize{14}{16}\selectfont \v  मनुष्याणां पारमार्थिकं शरीरं चक्रसप्तकयुक्तं विद्यते, मूलाधारादीनि आज्ञाचक्रान्तानि षट् गुदादि भ्रूमध्यान्तेषु देशेषु व्यवस्थितानि~। सप्तमं  सहस्रारं ब्रह्मरन्ध्रे मस्तके व्यवस्थितम् ~। एषु सप्तसु क्रमेण चतुर्दलं,षड्दलं,दशदलं, द्वादशदलं,द्विदलं, सहस्रदलञ्चेति सप्तपद्मानि योगि गम्यानि भवन्ति~। जीवः स्वसाधनबलेन गुरुकृपानुगृहीतेन यदा मूलाधारात् ऊर्ध्वोर्ध्वस्थानम् आरोहति तदा अस्य बाह्यव्यवहारे स्वभावे च परिवर्तनं भवति। क्रमेण पशुस्वभावः अपैति, देवभावः समुदेति~। तदेतत् जीवस्य मूलाधारादिषु चक्रेषु सञ्चरणं, तत्फलञ्च वेत्ति गुरुरेव~।}
{\fontsize{14}{16}\selectfont \v  सदाशिव-विष्णु-चतुर्मुख-वसिष्ठ-शक्ति-पराशर-व्यास-शुक्र-गौडपादादिगुरुपरम्परया अविच्छिन्नया समानीतं तत्त्वज्ञानभारं यो वहति स गुरुः~।}
{\fontsize{14}{16}\selectfont \v  सहजानन्दः - आत्मानन्दः}
{\fontsize{14}{16}\selectfont \v  येन कृतः ज्ञानोपदेशः शिष्यस्य अनुपदमेव विना आयासं ग्राह्यो भवति, शिष्ये विश्वासं मनःप्रसन्नतां चित्तसौख्यं च जनयति स गुरुः देवानामपि दुर्लभः~।}
{\fontsize{14}{16}\selectfont \v  आत्मतत्त्वं साक्षात्कृतवतः गुरोः सान्निध्ये इतरे संसारिणः अपि चित्तशान्तिं मनःप्रसन्नतां क्वचित् परमानन्दञ्च अनुभवन्ति~।}
{\fontsize{14}{16}\selectfont \v  आत्मतत्त्वम् अनुभवतो गुरोः कृपापूर्णावलोकनम्  अज्ञानिनां बन्धं शिथिलीकरोति~।}
{\fontsize{14}{16}\selectfont \v  संशयसङ्कल्पविकल्पादयो मनोविकाराः संसारिणां तत्त्वानुसरणे सन्मार्गप्रवृत्तौ विघ्नान् जनयन्ति~। ते च विकाराः आत्मतत्त्वज्ञानिनः सान्निध्ये शान्ता भवन्ति~।}
{\fontsize{14}{16}\selectfont \v  संसारिणो जनस्य मनःचक्षुरिन्द्रियद्वारा  रूपादिमत्सु बाह्यवस्तुषु एेकाग्र्यं करोति, ततः तानि वस्तूनि विषयीकरोति~। निमीलितचक्षुषोऽपि पुरुषस्य मनः पूर्वदृष्टान् बाह्यविषयान् आकलयति, चिन्तयति~~। विना विषयचिन्तनं मनः न स्थातुं शक्नोति~~। तीव्रसाधनबलेन मनः ध्यानावस्थायां निर्गुणे आत्मतत्त्वे निश्चलम् अवतिष्ठते~~। योगस्य परां काष्ठाम् आरूढवतो योगिनस्तु मनः आत्मतत्त्वे विलीनं भवति~~। अतः  चक्षुरुन्मीलनकालेऽपि तदीयमानः चक्षुर्द्वारा बाह्यवस्तूनि नैव विषयीकरोति~~।  पूर्वानुभूतविषयचिन्तनमपि न करोति~। सोऽयम् 'अनिमेषयोगः'~। सर्वदा गतिशीलः प्राणवायुः प्राणायामादिना कठिनसाधनेन निश्चलो भवति~~। आत्मतत्त्वे प्रतिष्ठितप्रज्ञस्य योगिनस्तु प्राणवायुः प्रयासं विनैव निश्चलो भवति~~। एषः 'सहजसमाधिः' ~।}
{\fontsize{14}{16}\selectfont \v  संसारिणां जाग्रत्स्वप्नसुषुप्तय इति तिस्रोऽवस्थाः भवन्ति~। ज्ञानिनां तु तुरीयावस्था तुरीयातीतावस्था च भवति~। तुरीयावस्थायां चिदभिव्यञ्जकनादस्य वेदनं भवति~। तुरीयातीतावस्थायां नादस्यास्य परमसूक्ष्मता भवति~। एषैव उन्मन्यवस्थोच्यते~। अत्र योगी आनन्दैकघनो भवति~।\footB \ }
{\fontsize{14}{16}\selectfont \v  ब्राह्मणक्षत्रियवैश्यशूद्राणां प्रत्येकशः वर्णधर्माः, ब्रह्मचारि-गृहस्थ-वानप्रस्थ-यतीनां प्रत्येकशः आश्रमधर्माश्च शास्त्रेषु विहिताः वर्तन्ते~। इमे धर्माः आत्मतत्त्वसाक्षात्कारशून्यैः देहाभिमानवद्भिः अवश्यम् अनुसरणीयाः भवन्ति~। देहेन्द्रियादिभ्योऽतिरिक्तं, परिशुद्धम् आत्मस्वरूपं साक्षात्कृतवन्तो ज्ञानिनस्तु एषां वर्णाश्रमधर्माणाम् आचरणे अधिकारिणो न भवन्ति~। तेषां ज्ञानिनः विषये शास्त्रोक्तनियमाः न प्रवर्तन्ते~।ज्योतिरेव वर्णाश्रमौ अस्यास्तीति ज्योतिर्वर्णाश्रमी~।\footB \  }
{\fontsize{14}{16}\selectfont \v  मनुष्याणां सूक्ष्मशरीरे षट्चक्राणि भवन्ति, मूलाधार-स्वाधिष्ठान-मणिपूर-अनाहत-विशुद्ध-आज्ञासंज्ञकानि~। योगिगोचरेषु एषु चक्रेषु मूलाधारचक्रं गुदप्रदेशे भवति~। तत्र कुण्डलिनीशक्तिः अवतिष्ठते~। ततो मूलाधारे निष्पन्नो नादः पराभिधः~। स ऊर्ध्वं नीतः स्वाधिष्ठाने विजृम्भितः पश्यन्त्याख्यो भवति~। स पुनः ऊर्ध्वं नीतः हृदये अनाहतचक्रे मध्यमाख्यः~। स पुनः ऊर्ध्वं नीतः कण्ठदेशे विशुद्धचक्रे वैखरीसंज्ञको भवति~।}
{\fontsize{14}{16}\selectfont \v  बाणलिङ्गं - नर्मदातीरोपलब्धम्~~। स्वयम्भूलिङ्गं - स्वयम् आविर्भूतम्~~। इतरलिङ्गं - पार्थिवादि इति स्थूलार्थः ~। मूलाधार-हृदय-भ्रूमध्येषु ऊर्ध्वस्फुरच्छक्तिज्योतिर्बिन्दुलिङ्गानि पश्चिमाभिमुखानि स्वयम्भू-बाण-इतराख्यानि लिङ्गानि इति आन्तरार्थः ~।}\footB \  
{\fontsize{14}{16}\selectfont \v  हठयोगे प्रसिद्धाः पञ्चविंशतिर्मुद्राः~।\footB \  तदन्तर्गतं मुद्रापञ्चकं महामुद्रादि~~। तच्च जरामरणनिरोधकं भवति ~। }
{\fontsize{14}{16}\selectfont \v  मानुषशरीरस्य व्यावहारिकं पारमार्थिकञ्चेति रूपद्वयं विद्यते ~। आद्यं शिरास्थिप्रभृतिभिः अङ्गैः विविधैः युक्तम् \footB \   इत्यादिना तत्तत्सङ्ख्या प्रोक्ता ~। द्वितीयं ब्रह्माण्डेन तादात्म्यापन्नम् योगिनां गोचरः~। ब्रह्माण्डे ये चतुर्दशलोकाः, सप्तद्विपाः सप्तकुलपर्वताः ते पिण्डे सूक्ष्मशरीरेऽपि भवन्ति ~। तद्यथा पादादिकट्यान्तेषु अवयवेषु अतलादिपातालान्ताः सप्तलोकाः, नाभिप्रभृतिब्रह्मरन्ध्रान्तेषु अवयवेषु  भूरादिसत्यान्ताः सप्तलोकाः भवन्ति~। }
{\fontsize{14}{16}\selectfont \v  योगस्य अष्टौ अङ्गानि भवन्ति - यमः, नियमः, आसनं, प्राणायामः, प्रत्याहारः, धारणा, ध्यानं, समाधिरिति~।  तदन्तर्गतम् आसनं पद्मासनादिभेदेन चतुरशीति प्रभेदं भवति~।}
{\fontsize{14}{16}\selectfont \v  जीवानां त्रिविदं मलम् (अज्ञानं) भवति~~।  \footB \  तत्र आणवमलं - सदाशिवस्य स्वस्य अनवमर्शः~।  \footB \  कार्ममलम् - ‘पुण्यपापवान् अहम्' इति प्रतीतिः ~। मायीयमलम् - महदहङ्कारादौ आत्मबुद्धिः~। एतानि त्रीण्यपि मलानि शोधयितुं समर्थः यः स गुरुः~।}
{\fontsize{14}{16}\selectfont \v  कर्मवासनाः त्रिविधाः भवन्ति~। तत्र आरक्तवासना(कृष्णवासना) दुरात्मनां भवति~। शुक्लवासना तपस्स्वाध्यायध्यानरतानां, मिश्रवासना परानुग्रहपरपीडनोभयवतां जनानाम्~। कर्मफलत्यागिनां ज्ञानिनां तु अरक्ता अशुक्ला वासना भवति~।}
{\fontsize{14}{16}\selectfont \v  पिण्डं - कुण्डलिनी शक्तिः~~। पदं - हंसः ~। रूपं - बिन्दुः ~। रूपातीतं - निष्कलम् ~।}
{\fontsize{14}{16}\selectfont \v   शिवादिक्षितिपर्यन्तेषु षट्त्रिंशत्त्वानां तत्त्वेषु चतुर्धा विभागः कृतः~~। पञ्चभूतप्रभृतिमायात्त्वानि आत्मतत्त्वम्~~। शुद्धविद्या-ईश्वर-सदाशिवाः विद्यातत्त्वम्~~। शिवः शक्तिश्च शिवतत्त्वम्~। सर्वेषामेषां समुदायः तुरीयतत्त्वम् ~। तदुक्तं - 
{\fontsize{14}{16}\selectfont \v  शिवादिक्षितिपर्यन्तानि षट्त्रिंशत् तत्त्वानि - शिवः, शक्तिः, सदाशिवः, ईश्वरः, शुद्धविद्या, माया, कालः, नियतिः, कला~, अशुद्धविद्या, रागः, पुरुषः, गुणप्रकृतिः, बुद्धिः, अहंकारः, मनः, ज्ञानेन्द्रियाणि पञ्च, कर्मेन्द्रियाणि पञ्च,  पञ्चतन्मात्राणि, पञ्चभूतानि इति~। इमानि ईश्वरस्य षट्त्रिंशत् तत्त्वानि~। एषां विशेषाभिज्ञो गुरुः~।}
{\fontsize{14}{16}\selectfont \v  गुरोः बाह्यलक्षणानि बहूनि भवन्ति ~। बाह्यलक्षणरहितोऽपि यः आत्मतत्त्वसाक्षात्कारं प्राप्तवान् स गुरुः एव~~। स एव बन्धाद् विमुक्तः, अन्यान् मोचयितुं प्रभवति ~।}
{\fontsize{14}{16}\selectfont \v  दीक्षाविधेः अङ्गम् अध्वशोधनम् ~। कलाध्वा तत्त्वाध्वा भुवनाध्वा वर्णाध्वा पदाध्वा मन्त्राध्वा इति षण्णाम् अध्वनां शोधनं गुरुणा कर्तव्यम् ~। तत्र निवृत्तिप्रतिष्ठाविघ्नाशान्तिशान्त्यतीताः पञ्चकला \hfil कलाध्वा ~। शिवादिक्षितिपर्यन्तानि षट्त्रिंशत् तत्त्वानि \hfil तत्त्वाध्वा~~।
{\fontsize{14}{16}\selectfont \v  लोकसम्मोहनकरः - जनमानसानाम् आकर्षणीयैः औदार्यदयादिगुणैः युक्तः ~।}
{\fontsize{14}{16}\selectfont \v  आज्ञासिद्धिः - आज्ञया सिद्धिः यस्य ~। आदेशमात्रेण कार्यं सिद्ध्यति इत्यर्थः~~।}
{\fontsize{14}{16}\selectfont \v  वेधकः - वेधदीक्षाप्रदाने प्रवीणः ~। बोधकः - निगूढतत्त्वबोधने कुशलः ~। षड्वर्गविजयक्षमः - कामक्रोधादीनां रिपूणां षण्णां समूहं जितवान् ~।}
{\fontsize{14}{16}\selectfont \v  मन्ददर्शनदूषकः - मन्ददर्शनं दूषकं यस्य यस्मिन् वा ~। गुरौ उपेक्षाबुद्धिः उपेक्षाभवना च उपेक्षकस्य दोषं जनयतीत्यर्थः ~। अग्रगण्यः - गुणिनां गणनावसरे प्रथमतया गण्यते~। शिवविष्णुसमः - शिवविष्णू समौ यस्य ~। हरिहरभेददृष्टिरहितः इत्यर्थः~~। शिवविष्णुभ्यां समः~। मङ्गलकारित्वात् शिवसमानता ~। दृष्ट्या व्यापकत्वात् विष्णुसमानता~। एवं विद्यादानाय कः चात्र  कश्च अपात्रम् इति वेत्ति~।}
{\fontsize{14}{16}\selectfont \v  शिष्टसाधकः  - शिष्टानां साधकः ~। तत्त्वान्वेषणे ज्ञानसाधनायां मन्त्रादिसाधनायां वा शिष्टान् प्रवर्तयति, प्रवृत्तान् अग्रे सारयति ~।
{\fontsize{14}{16}\selectfont \v  वेदविहितानि कर्माणि नित्यनैमित्तिककाम्यभेदेन त्रिविधानि भवन्ति~। अज्ञानां  नित्यनैमित्तिककर्माणि पापक्षयाकारिणि, अतः प्रशस्तानि~~। ज्ञानी तु गुरुः लोकानाम् उन्मार्गप्रवृत्तिनिवारणाय तानि आचरति ~। अज्ञैः काम्यकर्माणि सांसारिकफलमुद्दिश्य क्रिमाणानि बन्धहेतुत्वात् निन्दितानि~। ज्ञानी पुनः लोकहितमेवोद्दिश्य आचरति ~। अतः तानि प्रशस्तानि भवन्ति ~।}
{\fontsize{14}{16}\selectfont \v    गुरुः स्वेन सम्प्रदायप्राप्तविद्यायाः अनुष्ठानपरो भवन् धर्मादिचुतुष्टयम् \footB \  उपार्जयति~।}
{\fontsize{14}{16}\selectfont \v    गुरुः स्त्रीधनादिषु विषयेषु द्यूतादिव्यसनेषु च अनासक्तो भवेत्~~।  गुरुः सर्वेषु जनेषु अहंभावेन संयुक्तो भवति सर्वेषां दुःखं स्वदुःखं भावयति एवं सर्वेषां कल्याणमेव स्वकल्याणं भावयतीत्यर्थः ~। निर्द्वन्द्वः - शीतोष्णसुखदुःखादि द्वन्द्वानि सहमानः समभावेन तिष्ठति ~।}
  {\fontsize{14}{16}\selectfont \v  मनुष्याणां प्रवृत्तिः द्विविधा भवति उच्छास्त्रप्रवृतिः शास्त्रितप्रवृत्तिः इति ~। बाह्यविषयेषु  मनसः सङ्कल्पनवशात्  विवेकं  तिरस्कृत्यैव जायमाना प्रवृत्तिः उच्छास्त्रप्रवृत्तिः~। एषा अज्ञाना भवति ~।   मनस्सङ्कल्पनं तिरस्कृत्य विवेकप्रज्ञया जायमाना प्रवृत्तिः शास्त्रितप्रवृत्तिः ~। एषा ज्ञानिनां भवति ~। इत्थम् अज्ञः संसारी सङ्कल्पपक्षपाती, ज्ञानी गुरुः विवेकपक्षपाती असङ्कल्पपक्षपातीत्युच्यते ~। गुरुः धनादिस्वीकारपूर्वकं मन्त्रादिवितरणं न करोति ~।}
{\fontsize{14}{16}\selectfont \v    क्वचिदपि जने वस्तुनि च गुरोः मोहो न भवति, विपरीतकल्पनं नानाविधकल्पनं वा न भवति~~। साधनीये अर्थे मार्गे च निर्णयो भवति~।}
{\fontsize{14}{16}\selectfont \v  यो गुरुः दीक्षां प्रदाय आत्मतत्त्वमुपदिशति स एव तत्त्वसाक्षात्कारप्राप्तिपूर्वकं मुक्तिप्राप्तौ कारणं भवति~। तस्यैव गुरोः पादुका पूजनीया ~। अन्ये प्रेरकादयः~। पञ्च गुरवस्तु शिष्यस्य दीक्षाप्राप्तियोग्यतासम्पादनकार्यं  निर्वहन्ति~। }  
{\fontsize{14}{16}\selectfont \v  यथा भूमिः मनुष्यपशुस्थावरादिभिः अपरिमितैः जीवजडराशिभिः आक्रान्तापि तद्भारं सहते, न तद्भारेण भिद्यते~। तथा धीरः पुराकृतकर्मणां फलीभूतैः विविधविपत्तिविघ्नसङ्कटैः आक्रम्यमाणोऽपि तानि सहेत,कदापि न्याय्यात् मार्गात् प्रच्युतो न भवेत्~~। }
{\fontsize{14}{16}\selectfont \v  यथा वायुः मलिनेषु कर्दमादिषु पवित्रेषु गङ्गादिषु सर्वत्र  सञ्चरन्नपि तदीयगुणदोषैः अस्पृष्टः निर्लिप्तो भवति ~। तथा योगी मृदुषु क्रूरेषु मूर्खेषु विद्वत्सु स्रीषु पुंसु प्राणिषु अप्राणिषु च  सर्वतो व्यवहरन्नपि तदीयगुणदोषैः अस्पृष्टः निर्लिप्तो भवेत्~~।।}
{\fontsize{14}{16}\selectfont \v  यथा आकाशः वर्षाकाले उत्पद्यमानानां वायुना प्रेरितानां मेघादीनां तैजसजलीयपार्थिवभागानां सञ्चरणाश्रयः सन्नपि तैः न स्पृश्यते, तथा योगी ईश्वरेण सृष्टानां कर्मणा प्रेर्यमाणानां मनोविकाराणां रागद्वेषमोहादिगुणानां सञ्चारं वीक्षमाणोऽपि तैः न स्पृश्येत~।}
{\fontsize{14}{16}\selectfont \v  यथा जलं स्वभावतः परिशुद्धं स्नेहयुक्तं मधुरम्, अन्येषां मनुष्यादीनां बाह्याभ्यन्तरमालिन्यनिवारकञ्च \footB \  भवति, तथा धीरः दीक्षा-नियम-गुरुदेवताभक्त्यादिना स्वयं परिशुद्धः स्नेहमृदुस्वभावश्च सन् अन्येषामपि दृष्टादृष्टदोषान् शोधयेत्। }   
{\fontsize{14}{16}\selectfont \v  यथा अग्निः प्रकाशेन ज्वलनेन च दीप्यमानः अन्यैः स्प्रष्टुमशक्यः मलिनं वस्तु दहन्नपि मालिन्येन लिप्तो न भवति, धीरः तथा तेजस्वी  तपसा दीप्यमानः अन्यैः अभिभवितुमशक्यः\footB \  भवेत्~। आत्मतत्त्वे समाहितचित्तः सन् अशुद्धं प्रतिगृह्णन्नपि अशुद्धिं न आददीत।।}
{\fontsize{14}{16}\selectfont \v  यथा सूक्ष्मकालसम्बन्धानुगुणं चन्द्रस्य कलानां षोडश भागाः भवन्ति ~। कलाया एव वृद्धिक्षयौ भवतः, न चन्द्रस्य ~। तथा पुराकृतसूक्ष्मकर्मवशात् अस्माकं शरीरस्यैव निषेकाद्याः श्मशानान्ताः षोडश संस्काराः भवन्ति, शरीरस्यैव च जननमरणे भवतः, न तु आत्मनः~।}
{\fontsize{14}{16}\selectfont \v  यथा सूर्यः ग्रीष्मकाले स्वकिरणैः नद्यादिगतं जलं सूक्ष्मरूपेण आदाय वृष्टिकाले विमुञ्चति,तथा योगी व्यवहारकाले चक्षुरादिभिः इन्दियैः रुपरसादीन् मैत्रीकरुणादींश्च गुणान् उपादाय समुचितकाले विनियोजयेत्, न तु मनसि स्थापयेत्~~।। \footB \ }
{\fontsize{14}{16}\selectfont \v  केनचिदपि क्वापि अतिस्नेहो न कर्तव्यः~। अतिस्नेहेन मनसि दैन्यं जायते ~। दीनमनस्को जनः कपोत इव सन्तापम् अभिमुखीकरोति ~।।
{\fontsize{14}{16}\selectfont \v  अजगरः\footB \   इव धीरः योगी रुचिरहितं नीरसम् अल्पं बहु वा दैवाल्लब्धमन्नं ग्रसेत्~। यथा अजगरः ग्रासस्य अलाभे सति निराहारः निरुद्यमः सन् बहूनि दिनानि शेते  तथा योगी भोग्यविषयाणाम् असन्निधानदशायां स्वयं विषयान् न आहरेत्, स्वयं कर्माणि न आरभेत~। बाह्यविषयचिन्तां परित्यज्य चित्तम् आत्मन्येव चिरं समाहितं कुर्यात् ~।}
{\fontsize{14}{16}\selectfont \v  यथा सागरः नदीनां समागमेन न वर्धते, असमागमे न शुष्यति ~। सर्वदा क्षोभरहितः प्रशान्तः दुर्विगाहः तिष्ठति ~। तथा मुनिः समृद्धीनां प्राप्त्या न हृष्येत्,अप्राप्त्या न खिद्येत ~। परमात्मपरो भूत्वा सर्वदा समचित्तः निर्विषदः भवेत्~।।}
{\fontsize{14}{16}\selectfont \v  यथा पतङ्गः दीपप्रभां पुष्पमिति आहार इति वा उपभोग्यं मत्वा तदौष्ण्यमपि अपरिगणयन् तत्रैव पतित्वा म्रियते। तथा मूढः पुरुषः विवेकाभावेन स्रीसुवर्णादीन् उपभोग्यान् सुखप्रदान् मत्वा तेष्वासक्तः तत्रैव निमग्नः अनर्थं प्राप्नोति।।}
{\fontsize{14}{16}\selectfont \v  यथा भ्रमरः लघुभ्यः बृहद्भ्यश्च बहुभ्यः पुष्पेभ्यः सारभूतं मधु सङ्गृह्णाति ~। प्रत्येकपुष्पात् अल्पमेव मधु सङ्गृह्य एकमपि पुष्पं न पीडयति स्वयं पुष्टिं लभते ~। तथा मुनिः बहुषु गृहेषु भिक्षाम् अटेत्~~। प्रत्येकशः अल्पमेव  याचन् कमपि गृहिणम् अपीडयन् स्वस्य देहयात्रां निर्वर्तयेत् ~। एवं धीरः अल्पेभ्यः महद्भ्यश्च बहुभ्यः शास्त्रेभ्यः अल्पमपि सारं सङ्गृह्णीयात्~।}
{\fontsize{14}{16}\selectfont \v  गजः करिणीस्पर्शसुखानुभवे निमग्नः स्वात्मानं विस्मृत्य मनुजस्य वशवर्ती भवति, आजीवनं पारतन्त्र्यम् अनुभवन् दुःखी भवति~~। अतः संन्यासी युवतिं पादेनापि न स्पृशेत्, काष्ठमयीं मृण्मयीं वा युवतिप्रतिमामपि न स्पृशेत् ~।}
{\fontsize{14}{16}\selectfont \v  यथा मधुपः महता श्रमेण मधु सञ्चिनोति,तच्च मधु अन्यैः मनुष्यैः भुज्यते। तथा लुब्धःमहता क्लेशेन प्रभूतं द्रव्यं सञ्चिनोति,न भुङ्कते,नैव च ददाति। तच्च सञ्चितं निगूढं द्रव्यं कालान्तरे अन्येषां वशमायाति।}
{\fontsize{14}{16}\selectfont \v  व्याधस्य मृगयागीतेन धनुष्ठेङ्काररवेण च मोहं गतः हरिणः तमनुसरन् व्याधस्य वशवर्ती भवति ~। आमरणं पारतन्त्र्येण दुःखम् अनुभवति ~। अतः यती वने वसन्नपि ग्राम्यगीतं न शृणुयात् ~।।}
{\fontsize{14}{16}\selectfont \v  मीनः बडिशा ग्रस्थितमासंखण्डस्य\footB \   रसेन आकृष्टः तद्भक्षणोन्मुखः बडिशेषु निगृहीतः मरणं याति ~।
{\fontsize{14}{16}\selectfont \v  आशा महद्दुखं जनयति, आशाविरहे तु मनः प्रशान्तं भवति, यथा पिङ्गलायाः~। \footB \ }
{\fontsize{14}{16}\selectfont \v  भोग्यवस्तूनां सङ्ग्रहेण  दुःखपरम्परा समापतति, भोगवस्तुपरित्यागेन दुःखं निवर्तते ~। यथा क्रौञ्चस्य~। \footB \ }
{\fontsize{14}{16}\selectfont \v  यथा बालस्य सम्मानावमानौ न स्तः, गृहपरिवार-तत्पोषणादिचिन्ता नास्ति ~। सर्वदा स्वक्रिडायां रतिः, नान्येषां चिन्ता, नान्यत्र रतिः, तथा योगी सम्मानावमानौ न परिगणयेत्,\footB \  बाह्यविषयकचिन्तनं परित्यजेत्~। सदा परमानन्दस्वरूपे आत्मनि रममाणो भवेत्~।।}
{\fontsize{14}{16}\selectfont \v  एक एव  तपश्चरेत्, नान्येन अन्यैर्वा सह। अनेकेषां सहवासः वार्तालाप कलहादीन् प्रसञ्जयति,उद्देश्यं भञ्जयति। यथा कुमारीकङ्कणानाम्~। 
{\fontsize{14}{16}\selectfont \v  योगी बाह्यविषयेभ्यः मनः प्रत्याहृत्य देहेन्दियादिभ्यो विलक्षणे अत्यन्तसूक्ष्मे आत्मतत्त्वे स्थापयेत् ~। गाढया आत्मतत्त्वबुभुत्सया ऎकाग्र्येण च मनः आत्मनि निरुद्धं कुर्यात्, अन्यत् किमपि वस्तु न स्फुरेत् ~। यथा इषुकारस्य शरनिर्माणे अवहितचित्तस्य \footB \ }
{\fontsize{14}{16}\selectfont \v  यस्य मनः स्वाधीनं प्रशान्तं न भवति, तस्य गृहनिर्माणादिकार्यं मनसि अधिकं विक्षेपम् जनयति, मोहं जनयति, परिरक्षणादौ मनः व्यापारयति ~। इत्थं व्युत्थितम् अशान्तं मनः विषयेषु सञ्चरद्  दुःखं जनयति ~। अतः अवशीकृतचित्तो जनः आत्मतत्त्वबुभुत्सुः गृहनिर्माणप्रभृतिभ्यः बाह्यव्यापारेभ्यः विरतो भवेत्,परगृहे स्वकार्यं साधयेत्~। यथा सर्पः अन्यनिर्मितबिले वसति ~।।}
{\fontsize{14}{16}\selectfont \v  यथा ऊर्णनाभिः हृदयात् ऊर्णां मुखद्वारा बहिरानीय तयैव विहृत्य पश्चात् तामूर्णां ग्रसति ~। तथा ईश्वरः मायया  प्रपञ्चं सृष्ट्वा तेन क्रीडति~, पश्चात्  स्वस्मिन्नेव प्रपञ्चं विलापयति ~। योगी अपि आज्ञानेन कल्पितम् अनात्मप्रपञ्चं  स्वात्मनि विलापयेत ~। }
{\fontsize{14}{16}\selectfont \v  यो जीवी यं यं विषयं जनं वा स्नेहवशात् भयवशात् द्वेषवशात् वा बुद्धिपूर्वकं मनसा निरन्तरं धारयति स  जीवी तत्तत्स्वरूपं प्राप्नोति ~। यथा कीटः पेशस्कृतं\footB \  ध्यायन् तद्रूपं प्राप्नोति ~।}
{\fontsize{14}{16}\selectfont \v मन्दमतिभिः भृत्यादिभिरपि साधयितुं योग्ये कर्मणि तीक्ष्णमतिं शिष्यं नियोज्य, मेधासामर्थ्येन बुद्धिकौशलेन साधनीये कर्मणि मन्दमतिं शिष्यं नियोज्य गुरुः परीक्षेत । धनत्यागेन प्राणत्यगेन च साधनीयेषु कर्मसु समानप्रतिभान् समानबुद्धिसामर्थ्यान् शिष्यान् नियोज्य परीक्षेत । एवं विनम्रतां गुर्वाज्ञापालनाञ्च परीक्षेत ।}
{\fontsize{14}{16}\selectfont \v शिष्यस्य चेष्टासु वाचि च क्रौर्यं कपटं च गुरुणा पुनःपुनः परीक्षणीयम् । समदृष्ट्या निर्वहणीयेषु कर्मसु  शिष्यस्य पक्षपातदृष्टिः, सावधानं निर्वोढव्येषु कर्मसु  शिष्यस्य औदासीन्यञ्च पुनःपुनः परीक्षणीयम् ।}
{\fontsize{14}{16}\selectfont \v  ब्रह्मचारी सायं प्रातः समिद्धोमं, भिक्षासमूहाहरणम्, अखट्वाशयनम् \footB \ ~, गुरोः उदकुम्भाद्याहरणम्~, समावर्तनपर्यन्तं कुर्यात्~।}
{\fontsize{14}{16}\selectfont \v  गन्धं - कर्पूरचन्दनकस्तूरिकादि वर्जयेत् ~। उद्रिक्तरसान् गुडादीन् अपि न खादेत् ~। शुक्तानि यानि स्वभावतो मधुरादिरसानि, कालवशेन उदकवासादिना चाम्लयन्ति, तानि न खादेत् ~।}
{\fontsize{14}{16}\selectfont \v  जनवादं - जनैः सह निरर्थकवाक्कलहं, परिवादं - परस्य दोषवादम्, उपघातम् - अपकारं, वर्जयेत्~~।}
{\fontsize{14}{16}\selectfont \v  ब्रह्मचारी इच्छया रेतः पातयन् स्वकीयव्रतं  नाशयति~~। व्रतलोपे च अवकीर्णप्रायश्चित्तं कुर्यात्~~।}
{\fontsize{14}{16}\selectfont \v  आचार्यस्य यावद्भिः वस्तुभिः प्रयोजनानि सिद्ध्यन्ति तावन्ति आचार्यार्थम्  आहरेत् ~।}
{\fontsize{14}{16}\selectfont \v  ब्रह्मचारी शुचिः मौनी भिक्षेत ~। अभिशस्तान् महापातकयुक्तान् गृहिणः त्यजेत् ~।}
{\fontsize{14}{16}\selectfont \v  ब्रह्मारम्भे - वेदाध्ययनस्यारम्भे ~। हस्तौ संहृत्य – हस्तौ संश्लिष्टौ कृत्वा~।}
{\fontsize{14}{16}\selectfont \v   शिष्यः सव्येन पाणिना गुरोः सव्यं पादं स्पृशेत्, दक्षिणेन पाणिना गुरोः दक्षिणपादं स्पृशेत्~~।}
{\fontsize{14}{16}\selectfont \v  आचान्तः - कृताचमनः~~।}
{\fontsize{14}{16}\selectfont \v  वेदाध्ययनस्य आरम्भे समाप्तौ च प्रणवोच्चारणं कर्तव्यम्~~।}
{\fontsize{14}{16}\selectfont \v  गुरुसन्निधौ योगिसन्निधौ महासिद्धिपीठेषु च  पादप्रक्षालनादीनि न आचरणीयानि~~।}
{\fontsize{14}{16}\selectfont \v  विद्याद्यधिकेन गुरुणा च साधारण्येन स्वीकृते शयने आसने च तत्कालमपि शिष्यः न उपविशेत् ~। स्वयञ्च शय्यासनस्थः शिष्यः गुरौ आगते उत्थाय अभिवादनं कुर्यात्~।}
{\fontsize{14}{16}\selectfont \v  गुरुगृहे वसन् शिष्यः आचार्यात्  अनुमतिं स्वीकृत्यैव स्वान्   मातापितृपितृव्यादीन् अभिवादयेत् ~।}
{\fontsize{14}{16}\selectfont \v  वयोविद्यादिना वृद्धे आगच्छति सति अल्पवयसः जनस्य प्राणाः देशाद् बहिर्निर्गन्तुमिच्छन्ति~~। वृद्धस्य प्रत्युत्थानाभिवादाभ्यां ते प्राणाः पुनः सुस्थाः भवन्ति ~। }
{\fontsize{14}{16}\selectfont \v \v  गुरोः आज्ञास्वीकारं, गुरुणा सम्भाषणञ्च शिष्यः शय्यायां सुप्तः, आसनोपविष्टः भुञ्जानः~, तिष्ठन् विमुखश्च न कुर्यात्~~।}
{\fontsize{14}{16}\selectfont \v  आदिशतः पराङ्मुखस्य गुरोः शिष्यः सम्मुखस्थः भूत्वा~, दूरस्थस्य गुरोः समीपमागत्य, शयानस्य गुरोः प्रह्वो भूत्वा, निकटे अवस्थितस्य गुरोः प्रह्वीभूयैव शिष्यः आज्ञास्वीकृतिं सम्भाषणञ्च कुर्यात् ~।}
{\fontsize{14}{16}\selectfont \v  यत्र वातः गुरुदेशात् शिष्यदेशमागच्छति~, यत्र च शिष्यदेशात् गुरुदेशमागच्छति तत्र शिष्यः गुरुणा समं न आसीत~। एवं    शिष्यः स्वोच्चारितपदानां श्रवणे गुरोः काठिन्यं न जनयेत्~।}
{\fontsize{14}{16}\selectfont \v  सुरच्छाया - देवताच्छाया~। शक्तिः - श्रीचक्राराधनायां पूज्यमाना स्त्री, गुरुपत्नी~।}
{\fontsize{14}{16}\selectfont \v कायेन धनधान्यादिना सम्मानेन सद्भावनया चेति चतुर्विधा गुरुशुश्रूषा सम्भवति ।}
{\fontsize{14}{16}\selectfont \v  शिष्याद् अन्यस्मै अपृष्टवते~, अन्याप्येन पृष्टवतेऽपि जनाय तत्त्वं न वक्तव्यम् ~। शिष्याय तु अपृच्छतेऽपि वक्तव्यम् ~। \footB \ }
{\fontsize{14}{16}\selectfont \v  यस्मिन् शिष्ये अध्यापिते धर्मः अर्थो \footB \  वा न भवति ~। अध्ययनानुरूपा परिचर्यापि न भवति~~। तत्र विद्या न अर्पणीया~।}
{\fontsize{14}{16}\selectfont \v  अपात्राय शिष्याय विद्यार्पणमपेक्ष्य  विद्याम् अप्रदाय मरणमेव वेदाध्यापकस्य वरम् ~। अत्र छान्दोग्यब्राह्मणम् - ‘विद्या ह वै ब्राह्मणाम् अाजगाम तवाहमस्मि त्वं मां पालयानर्हते मानिने नैव मा दा गोपाय मा श्रेयसी तथाहमस्मि' इति ~। विद्याधिष्ठत्री देवता ब्राह्मणमेत्य एवम् आदिशतीति ~।}
{\fontsize{14}{16}\selectfont \v शास्त्रेण विहितानि यानि नित्यनैमित्तिककाम्यकर्माणि तदनुष्ठानावसरे मन्त्रलोप-यन्त्रलोप-क्रियलोपादिहेतुना  उत्पन्ना ये दोषाः, अयोग्यानां पाशबद्धानां दुस्संगत्या  उत्पन्ना ये दोषाः, मन्त्राणां साङ्कर्यात् उत्पन्ना ये दोषाः , प्रकटीकृताः अप्रकटीकृताः ये दोषाः, ज्ञात्वा आचरिताः अज्ञात्वा  आचरिताश्च ये दोषाः, तेषां दोषाणां गुरुलघुभावं मन्दतीव्रभावम् अनुसृत्य गुरुः शिष्याय प्रायश्चित्तं  दद्यात् ।}
{\fontsize{14}{16}\selectfont \v  क्रियादीक्षा वर्णदीक्षा कलादीक्षा स्पर्शदीक्षा वाग्दीक्षा दृग्दीक्षा मानसदीक्षा इति सप्तविधा दीक्षा~~। तत्र पुनः क्रियादीक्षा अष्टधा, वर्णदीक्षा त्रिधा, कलादीक्षा त्रिधा, मनोदीक्षा द्विधा~~।}
