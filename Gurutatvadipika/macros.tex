\usepackage{fontspec,graphicx}
%\usepackage{graphicx}
%\usepackage{gastex}
\graphicspath{ {images/} }
\usepackage{indentfirst}
\parindent 1.5cm
\usepackage{setspace}
%\usepackage{tikz}
%\usetikzlibrary{arrows}
%\usepackage{tikz-qtree}
%\usepackage{wallpaper}
%\usepackage{float}
\usepackage{polyglossia}
\usepackage[utf8]{inputenc}
\usepackage{imakeidx}
\makeindex
\usepackage{multicol}
\setlength{\columnseprule}{-2cm}
\usepackage{microtype}
\usepackage{parcolumns}

\usepackage{verse}
\settowidth{\versewidth}{समासीनं व्यासं सततं पुण्यचरितम्समासीनं व्यासं सततं पुण्यचरितम्समासीनं }

\usepackage{xstring}
\usepackage{footmisc}
\usepackage[robust]{bigfoot}
\usepackage{fancyhdr}
\usepackage[normalem]{ulem}

\DeclareNewFootnote{A}
\renewcommand{\thefootnoteA}{}
\SelectFootnoteRule[1]{default}

\DeclareNewFootnote{B}
\renewcommand{\thefootnoteB}{\devanagarinumeral{footnoteB}}
\MakeSortedPerPage{footnoteB}
\SelectFootnoteRule[1]{default}

\renewcommand\footnoterule{%
\vskip 3pt
  \hrule
\vskip 3pt}

\makeatletter

\setlength{\footnotemargin}{1em}   

\newread\fntA
\openin\fntA=vyakhya %%%%%%%%%For ---- for Anandagirikiriti

\newread\fntB
\openin\fntB=footnote %%%%%%%%%For regular footnotes


\newif\iffntArem \fntAremtrue 
\newif\iffntBrem \fntBremtrue 

\def\footA{\read\fntA to\datafromA%
\footnoteA{\datafromA}% 
}
\def\footB{\read\fntB to\datafromB%
\footnoteB{\small \datafromB}% 
}

\makeatother

\let\footnotesize\normalfont




\usepackage{hyperref}
\hypersetup{
    colorlinks=true,
    linkcolor=black,      
    urlcolor=cyan,
    pdftitle={mainfile},
}
%\usepackage{tabu}

\usepackage{titlesec}

\titleformat{\chapter}
 {\normalfont\fontsize{20}{16}\b \bfseries\centering}{\thechapter}{1em}{}
\titlespacing{\chapter}
{0cm}{-2em}{0.8em}%\titlespacing*{<command>}{<left>}{<before-sep>}{<after-sep>}
 
\titleformat{\section}
  {\normalfont\fontsize{17}{15}\b \bfseries\centering}{\thesection}{1em}{}



\titleformat{\subsection}
  {\normalfont\fontsize{15}{15.3}\bfseries\centering}{\thesubsection}{-12cm}{}
\titlespacing*{\subsection}
  {0pt}{0\baselineskip}{0\baselineskip}

%\usepackage[hang]{footmisc}
%\renewcommand*{\footnotesize}{\fontsize{14}{16}\selectfont}
%\renewcommand{\footnotelayout}{\setstretch{1.3}}

\usepackage{metalogo}
\usepackage{perpage}
\MakePerPage{footnote}
 
 \newcommand\blfootnote[1]{%
  \begingroup
  \renewcommand\thefootnote{}\footnote{#1}%
  \addtocounter{footnote}{-1}%
  \endgroup
}
%\usepackage{arydshln}
 
%\usepackage{thispagestyle}

%\usepackage{sectsty}
%\sectionfont{\fontfamily{s}\fontseries{Lohit Devanagari}\fontsize{25pt}{35pt}\selectfont}
%\subsectionfont{\fontfamily{phv}\fontseries{b}\fontsize{11pt}{20pt}\selectfont}
%\subsubsectionfont{\fontfamily{phv}\fontseries{b}\fontsize{11pt}{20pt}\selectfont}
 
\setmainfont[WordSpace=1.4, Ligatures=TeX,AutoFakeBold=2,AutoFakeSlant=0.2]{Sanskrit2003}
\setdefaultlanguage{sanskrit}
\newfontfamily\b[Ligatures=TeX,WordSpace=1.4,AutoFakeBold=6.5,AutoFakeSlant]{Sanskrit2003}
\newfontfamily\v[Ligatures=TeX,WordSpace=1.4,AutoFakeBold=2.5,AutoFakeSlant]{Sanskrit2003}
\newfontfamily\anuswara[Ligatures=TeX,AutoFakeBold=3.5,AutoFakeSlant]{Sanskrit2003_Aux}
\newfontfamily\eng{Gentium Plus}
\newfontfamily\engt{Times New Roman}


%\def\punct{\fontsize{11pt}{13pt}\eng\selectfont}
%\def\punctt{\fontsize{14pt}{13pt}\anuswara\selectfont}
\newfontfamily{\englishfont}{Times New Roman}
\usepackage{newunicodechar}
\newunicodechar{‘}{{\large\englishfont ‘}}
\newunicodechar{’}{{\large\englishfont ’}}

%\catcode`\,=\active\def,{{\punctt\char"002C}}

\usepackage{pdfpages}
\usepackage{unicode-math}
\usepackage{pdflscape}



\newcommand{\devanagarinumeral}[1]{%
  \devanagaridigits{\number\csname c@#1\endcsname}}
\renewcommand{\thechapter}{\Roman{chapter}}
\renewcommand{\thesection}{\devanagarinumeral{section}}
\renewcommand{\thesubsection}{\roman{subsection}}
\renewcommand{\thepage}{\devanagarinumeral{page}}
\renewcommand{\theenumi}{\devanagarinumeral{enumi}}
\renewcommand{\thefootnote}{\devanagarinumeral{footnote}}

%\usepackage{enumitem}
%\usepackage{blindtext}
\usepackage{multicol}
\setlength{\columnsep}{10cm}
\usepackage{fancyhdr}
\usepackage{xltxtra}
\usepackage{lipsum}
\usepackage{titlesec}
\titlespacing*{\section}{0pt}{0.5\baselineskip}{0.5\baselineskip}

\renewcommand*\contentsname {विषयानुक्रमणिका}
\makeatletter
\newcommand*{\toccontents}{\@starttoc{toc}}
\makeatother

\usepackage{marginnote}

\let\cleardoublepage\clearpage

\usepackage[papersize={145mm,220mm},textwidth=125mm,
textheight=190mm,headheight=6mm,headsep=5mm,topmargin=10mm,botmargin=11mm,leftmargin=16mm,rightmargin=11mm,cropmarks]{zwpagelayout}

\let\olditemize=\itemize
\def\itemize{
\olditemize
\setlength{\itemsep}{-1ex}
}

\let\oldenumerate=\enumerate
\def\enumerate{
\oldenumerate
\setlength{\itemsep}{-2ex}
}
\linespread{1.6}


\usepackage{fancyhdr}
\pagestyle{fancy}
\renewcommand{\headrulewidth}{0pt}
\fancyhead[CE]{ गुरुतत्त्वदीपिका}
\fancyhead[CO]{\rightmark}
\fancyhead[LE,LO]{}
\fancyhead[RE,RO]{}
\fancyhead[LE,RO]{\thepage}
%\rhead{\thepage}
\cfoot{}
%\renewcommand{\chaptermark}[1]{\markboth{#1}{}}
\renewcommand{\sectionmark}[1]{\markright{#1}{}}

\pretolerance=9990
