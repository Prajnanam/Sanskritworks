\clearpage
\thispagestyle{empty}
\fontsize{10}{10}\selectfont
\section{आकरग्रन्थसूची}
%\begin{multicols}{2}
%\begin{itemize}
%\setlength\itemsep{-0.4em}
\begin{raggedright}
\begin{parcolumns}[colwidths={1=.55\textwidth,2=.55\textwidth}]{2}
\colchunk{% left column
  \sloppy
{\large अ}\\
अक्रियैव परा पूजा	– 	कुलार्णवतन्त्रम्  – ९-३८\\
अक्षरत्वाद्वरेण्यत्वा	– 	कुलार्णवतन्त्रम्  – १७-२४\\
अखिलाघप्रशमना	– 	कुलार्णवतन्त्रम्  – १७-९५\\
अग्नीन्धनं भैक्षचर्या	– 	मनुस्मृतिः  – २-१०८\\
अग्नौ तिष्ठति विप्राणां	– 	कुलार्णवतन्त्रम्  – ९-४४\\
अणुभ्यश्च महभ्द्यश्च	– 	पुराणसारः   – ३३-३९\\
अत एव शिवस्साक्षात्	–    कुलार्णवतन्त्रम्  – १३-४२\\
अतः परं समालक्ष्य	– 	पुराणसारः   – ४०-६८\\
अतिह्रस्वो व्याधिहेतु	– 	कुलार्णवतन्त्रम्  – १५-५५\\
अत्याहारः– 	कुलार्णवतन्त्रम्  – १५-१०७\\
अदीक्षिता ये– 	कुलार्णवतन्त्रम्  – १४-९६\\
अद्वैतन्तु शिवेनोक्तं	– 	कुलार्णवतन्त्रम्  – १-१०८\\
अधर्मेण च	– 	मनुस्मृतिः  – २-१११\\
अधःस्थिते– 	कुलार्णवतन्त्रम्  – १२-१०१\\
अध्येष्यमाण– 	मनुस्मृतिः  – २-७०\\
अनन्तपारो ह्यक्षोभ्यो	– 	पुराणसारः   – ३३-३६\\	
अनन्तफलदानाच्च	– 	कुलार्णवतन्त्रम्  – १७-५८\\
अनभिज्ञं गुरुं– 	कुलार्णवतन्त्रम्  – १३-१३२\\
अनर्हे मन्त्रविज्ञानं	– 	कुलार्णवतन्त्रम्  – १४-१४\\
अनेककोटिमन्त्राणि	– 	कुलार्णवतन्त्रम्  – १५-१९\\
अन्नदानात् कुलेशानि	– 	कुलार्णवतन्त्रम्  – १७-७५\\	
अन्योन्यसम्मुखाकारः	– 	कुलार्णवतन्त्रम्  – १७-९१\\
अभिज्ञश्चोद्धरेन्मूर्खं	– 	कुलार्णवतन्त्रम्  – १३-१२५\\
अभिमानो न कर्तव्यो	– 	कुलार्णवतन्त्रम्  – १२-९४\\
अभिरूप्याच्च– 	कुलार्णवतन्त्रम्  – ६-७६\\
अभिवादनशीलस्य	– 	मनुस्मृतिः  – २-१२१\\
अभ्यङ्गमञ्जनं– 	मनुस्मृतिः  – २-१७८\\
अमेध्येन तु– 	कुलार्णवतन्त्रम्  – १५-१०३\\
अल्पं वा बहु वा– 	मनुस्मृतिः  – २-१४९\\
अशुभानि च कर्माणि	– 	कुलार्णवतन्त्रम्  – १२-४१\\
अश्वमेधायुतेनापि– 	कुलार्णवतन्त्रम्  – ९-१३२\\
असच्छिष्येष्वभक्तेषु	– 	कुलार्णवतन्त्रम्  – १४-१७\\
असंस्कृतोपदेशञ्च	– 	कुलार्णवतन्त्रम्  – १४-१३\\
असाध्यसाधकं– 	कुलार्णवतन्त्रम्  – १३-२८\\
अहिंसका दयावन्तो	–   पुराणसारः   – ४०-५०\\
{\large आ}\\
आकाशे पक्षिणां– 	कुलार्णवतन्त्रम्  – ९-६९\\
आकृष्टस्ताडितो– 	कुलार्णवतन्त्रम्  – १४-२२\\
आगमोक्तेनमर्गेण	– 	कुलार्णवतन्त्रम्  – १५-४५\\
आगमोत्थं विवेकोत्थं	– 	कुलार्णवतन्त्रम्  – १-१०१\\
आचारकथनाद्दिव्य	– 	कुलार्णवतन्त्रम्  – १७-४३\\
आचार्यपुत्रः– 	मनुस्मृतिः  – २-१०९\\
आज्ञाभङ्गोर्थहरणं	– 	कुलार्णवतन्त्रम्  – १२-७५\\
आणवं कार्मणञ्चैव	– 	कुलार्णवतन्त्रम्  – १३-८३\\
आत्मसिद्धिप्रदानाच्च	– 	कुलार्णवतन्त्रम्  – १७-६२\\
आत्मसिद्धिप्रदानाच्च	– 	कुलार्णवतन्त्रम्  – १७-६२\\
आत्मार्थमानसद्भावैः	– 	कुलार्णवतन्त्रम्  – १२-६४\\
आदित्वात् सर्वमार्गाणां	– 	कुलार्णवतन्त्रम्  – १७-४८\\
आदिमध्यावसानेषु	– 	कुलार्णवतन्त्रम्  – १४-२७\\
आदिश्य देवताभावं	– 	पुराणसारः   – ४०-६१\\
आदौ भक्तिर्भवेद्देवि	– 	कुलार्णवतन्त्रम्  – १४-२८\\
आदौ भक्तिविहीना	– 	कुलार्णवतन्त्रम्  – १४-३०\\
आनन्दकम्परोमाञ्च	– 	कुलार्णवतन्त्रम्  – १४-२४\\
आनन्दश्चैव	– 	कुलार्णवतन्त्रम्  – १४-६४\\
आब्रह्मबीजदोषाश्च	– 	कुलार्णवतन्त्रम्  – १५-६\\	
आभीष्टफलदानाच्च	– 	कुलार्णवतन्त्रम्  – १७-७१\\
आम्नायतत्त्वरूपत्वा	– 	कुलार्णवतन्त्रम्  – १७-५०\\
आरक्तशुक्लमिश्रा– 	कुलार्णवतन्त्रम्  – १३-८४\\
आलस्यं जृम्भणं– 	कुलार्णवतन्त्रम्  – १५-१०६\\
आवाहनादिकर्माणि	– 	कुलार्णवतन्त्रम्  – १७-८८\\
आशा हि परमं– 	पुराणसारः   – ३३-४४\\
आसमाप्तेः शरीरस्य	– 	मनुस्मृतिः  – २-२४४\\
आसीनस्य स्थितः	– 	मनुस्मृतिः  – २-१९६\\
आस्तिकं दानशीलञ्च	– 	कुलार्णवतन्त्रम्  – १३-२७\\
{\large इ}\\
इमं लोकं मातृभक्त्या	– 	मनुस्मृतिः  – २-२३३\\
इष्टधर्मादिकथना– 	 कुलार्णवतन्त्रम्  – १७-४२\\
इह यत् क्रियते कर्म	– 	कुलार्णवतन्त्रम्  – १-५३\\
इहाभीष्टप्रदं यद्वा– 	पुराणसारः   – ३३-१०\\
इहेष्टकामदं यत्तु– 	कुलार्णवतन्त्रम्  – १३-१२९\\
{\large उ}\\
उच्चैर्जपोधमः– 	कुलार्णवतन्त्रम्  – १५-५४\\
उत्तमा तत्त्वचिन्ता	– 	कुलार्णवतन्त्रम्  – ९-३५\\
उत्तमांश्चाधमे– 	कुलार्णवतन्त्रम्  – १४-२०\\
उत्पादकब्रह्मदात्रो – 	मनुस्मृतिः  – 	२-१४५\\
उदकुंभं सुमनसो– 	मनुस्मृतिः  – २-१८२\\
उपदेशस्य सामर्थ्यात्	– 	कुलार्णवतन्त्रम्  – १५-१३\\
उपदेशावशिष्टान्यः	– 	पुराणसारः   – ३३-१३\\
उपनीय गुरुः– 	मनुस्मृतिः  – २-६९ \\
उपनीय तु यः शिष्यं	– मनुस्मृतिः  – २-१४\\
उपपातकलक्षाणि– 	कुलार्णवतन्त्रम्  – १४-८५\\
उपासनाशतेनापि– 	कुलार्णवतन्त्रम्  – १४-८८\\
उल्बणत्वात्– 	 कुलार्णवतन्त्रम्  – १७-५३\\
उष्णीशी कञ्चुकी– 	कुलार्णवतन्त्रम्  – १५-१०८\\
{\large ऋ}\\
ऋजवो मृदवः– 	पुराणसारः   – ४०-५१\\
{\large ए}\\
एकदेशं तु वेदस्य– 	मनुस्मृतिः  – २-१४१\\
एकःशयीत– 	मनुस्मृतिः  – २-१८०\\
एकाब्देन द्विजो– 	कुलार्णवतन्त्रम्  – १४-१०४\\
एकैकमङ्गुलीभिः– 	कुलार्णवतन्त्रम्  – १५-५०\\
एवं कृतसमीपास्ते	– 	पुराणसारः   – ४०-५२	\\
एवं सर्वशरीरस्था– 	कुलार्णवतन्त्रम्  – ६-७८\\
एषु स्थानेषु	– 	कुलार्णवतन्त्रम्  – ६-७४\\
{\large क}\\
कमलासनरूपत्वा	– 	कुलार्णवतन्त्रम्  – १७-६०	\\
करपादोदरास्या– 	कुलार्णवतन्त्रम्  – ९-५\\
कर्मणा मनसा वाचा	– 	कुलार्णवतन्त्रम्  – २-८६\\
कलातत्वं च भुवनम्	– 	पुराणसारः   – ४०-६९	\\
कल्पयेद्भुवनं तत्वं	– 	कुलार्णवतन्त्रम्  – १५-५८\\
कामक्रोधादि– 	कुलार्णवतन्त्रम्  – ६-८६\\
कायक्लेशेन	– 	कुलार्णवतन्त्रम्  – १२-३९\\
किं तीर्थाद्यैर्महायासैः	– 	कुलार्णवतन्त्रम्  – १२-३८\\
कीटः पेशस्कृतं–   पुराणसारः   – ३३-५२\\
कुर्वीत चाङ्गसिध्यर्थं	– 	कुलार्णवतन्त्रम्  – १५-१०\\
कुलशास्त्रपियं देवि	– 	कुलार्णवतन्त्रम्  – १३-३७\\
केवलं गुरुशुश्रूषा– 	कुलार्णवतन्त्रम्  – १२-६६\\
क्रियमाणानि कर्माणी	– 	कुलार्णवतन्त्रम्  – ९-१२७\\
क्षिरोद्धृतं घृतं– 	कुलार्णवतन्त्रम्  – ९-१७\\
क्षीयन्ते सर्वपापानी	– 	कुलार्णवतन्त्रम्  – १२-६७\\
क्षुधितस्य यथा तृप्तिः	– 	कुलार्णवतन्त्रम्  – १२-१०३\\
{\large ग}\\
गकारो ज्ञानसम्मपत्ती	– 	कुलार्णवतन्त्रम्  – १७-९\\
गतं शूद्रस्य	– 	कुलार्णवतन्त्रम्  – १४-९१\\
गम्भीरापारदौर्भाग्य	– 	कुलार्णवतन्त्रम्  – १७-७३	\\
गवां सर्पिः– 	कुलार्णवतन्त्रम्  – ६-७७\\
गवां सर्वाङ्गजं– 	कुलार्णवतन्त्रम्  – ६-७५\\
गुकारःसिद्धदः– 	कुलार्णवतन्त्रम्  – १७-८\\
गुणैर्गुणानुपादत्ते– 	पुराणसारः   – ३३-३२\\
गुप्तप्रकटसम्भूतं– 	कुलार्णवतन्त्रम्  – ११-१०३\\
गुरवो बहवः सन्ति– 	कुलार्णवतन्त्रम्  – १२-१०६\\
गुरवो बहवः सन्ति कुमन्त्र	 – 	कुलार्णवतन्त्रम्  – १२-१०७\\
गुरवो बहवः सन्ति दीप	– 	कुलार्णवतन्त्रम्  – १२-१०४\\
गुरवो बहवः सन्ति वेद	– 	कुलार्णवतन्त्रम्  – १२-१०५\\
गुरवो बहवः सन्ति शिष्य	– 	कुलार्णवतन्त्रम्  – १२-१०८\\
गुरावतुष्टेऽतुष्टाः– 	पुराणसारः   – ३९-६१\\
गुरुं न मर्त्यं बुध्येत	– 	कुलार्णवतन्त्रम्  – १२-४६\\
गुरुकार्ये स्वयं शक्तो	– 	कुलार्णवतन्त्रम्  – १२-९२\\
गुरुकोपाद्विनाशः– 	कुलार्णवतन्त्रम्  – १२-७८\\
गुरुक्तं परुषं वाक्य	– 	कुलार्णवतन्त्रम्  – १२-५४\\
गुरुच्छायां सुरच्छायां	– 	कुलार्णवतन्त्रम्  – १२-१०२\\
गुरुदैवतसम्भक्तं– 	कुलार्णवतन्त्रम्  – १३-३२\\
गुरुःपिता गुरुर्माता	– 	कुलार्णवतन्त्रम्  – १२-४९\\
गुरुभक्तिविहीनस्य	– 	कुलार्णवतन्त्रम्  – १२-२९\\
गुरुभक्त्यग्निना– 	कुलार्णवतन्त्रम्  – १२-३०\\
गुरुभक्त्या यथा देवि	– 	कुलार्णवतन्त्रम्  – १२-३६\\
गुरुमित्रसुह्रद्दासी– 	कुलार्णवतन्त्रम्  – १२-८१\\
गुरुवाक्यप्रमाणज्ञं– 	कुलार्णवतन्त्रम्  – १३-३४\\
गुरुःशिष्याधिकारार्थं	– 	कुलार्णवतन्त्रम्  – १४-६\\
गुरुशिष्यावुभौ– 	कुलार्णवतन्त्रम्  – १४-११\\
गुरुस्रिवारमाचारं– 	कुलार्णवतन्त्रम्  – ११-१०८\\
गुरूपदिष्टमार्गेण– 	कुलार्णवतन्त्रम्  – १५-५९\\
गुरोः कुले न भिक्षेत	– 	मनुस्मृतिः  – २-१८४\\
गुरोरालोकमात्रेण	– 	कुलार्णवतन्त्रम्  – १४-५६\\
गुरोर्गुरौ सन्निहिते	– 	मनुस्मृतिः  – २-२०५\\
गुरोर्यत्र परीवादो– 	मनुस्मृतिः  – २-२००\\
गुरोर्यस्यैव सम्पर्कात्	– 	कुलार्णवतन्त्रम्  – १३-११०\\
गुरोर्हितं हि कर्तव्यं	– 	कुलार्णवतन्त्रम्  – १२-५०\\
गुरौ मनुष्य बुद्धिञ्च	– 	कुलार्णवतन्त्रम्  – १२-४५\\
गुशब्दस्त्वन्धकारः	– 	कुलार्णवतन्त्रम्  – १७-७\\
गुह्यागमात्मतत्त्वा	– 	कुलार्णवतन्त्रम्  – १७-१०\\
गृहारम्भो हि दुःखाय	– 	पुराणसारः   – ३३-४९	\\
गृहे शतगुणं– 	कुलार्णवतन्त्रम्  – १५-२६\\
गोब्राह्मणवधं कृत्वा	– 	कुलार्णवतन्त्रम्  – १२-९९\\
ग्राम्यगीतं न श्रृणुयात्	– 	पुराणसारः   – ३३-४२	\\
ग्रासं कुमृष्टं विरसं– 	पुराणसारः   – ३३-३४\\
{\large घ}\\
घृणा शङ्का भयं– 	 कुलार्णवतन्त्रम्  – १३-९०\\
{\large च}\\
चराचरसमासन्न– 	कुलार्णवतन्त्रम्  – १७-१२\\
चित्तं तत्त्वे समाधाय	– 	कुलार्णवतन्त्रम्  – १४-५४\\
चिन्मयस्याप्रमेयस्य	– 	कुलार्णवतन्त्रम्  – ६-७२\\
चैतन्यरहिता– 	कुलार्णवतन्त्रम्  – १५-६१\\
चोदितो गुरुणा– 	मनुस्मृतिः  – २-१९१\\
{\large ज}\\
जनः स्वदेहकण्डूतिं	– 	कुलार्णवतन्त्रम्  – ९-२०\\
जन्मान्तर सहस्रेषु	– 	कुलार्णवतन्त्रम्  – २-२८\\
जपध्यानं विनागर्भः	– 	कुलार्णवतन्त्रम्  – १५-४०\\
जपात् श्रान्तः– 	कुलार्णवतन्त्रम्  – १५-११४\\
जाग्रत्स्वप्नप्रसुषुप्तिश्च	– 	कुलार्णवतन्त्रम्  – १२-७५\\
जाड्यं दुःखं तृणच्छेदं	– 	कुलार्णवतन्त्रम्  – १५-१०९\\
जातिमानधने– 	कुलार्णवतन्त्रम्  – १३-३५\\
जाम्बूदनस्य– 	कुलार्णवतन्त्रम्  – ११-१०६\\
जिह्वयातिप्रमाथिन्या– 	पुराणसारः   – ३३-४३	\\
जिह्वा दग्धा परान्नेन	– 	कुलार्णवतन्त्रम्  – १५-७७\\
जीवेदग्नि प्रविष्टो वा	– 	कुलार्णवतन्त्रम्  – १२-७९\\
ज्ञानेन क्रियया वापि	– 	कुलार्णवतन्त्रम्  – १४-१९\\
ज्ञानोपदेशसामर्थ्यं	– 	कुलार्णवतन्त्रम्  – १४-२६\\
{\large त}\\
ततो नमेद– 	कुलार्णवतन्त्रम्  – १२-०११\\
तत् कर्म यत्र बन्धाय	– 	कुलार्णवतन्त्रम्  – १-११२\\
तत्त्वज्ञैरुपदिष्टा ये– 	कुलार्णवतन्त्रम्  – १३-१२३\\
तत्त्वस्वरूपमननात्	– 	कुलार्णवतन्त्रम्  – १७-२३\\
तत्त्वहीनं गुरुं– 	कुलार्णवतन्त्रम्  – १३-१२६\\
तत्वात्मकस्य देवस्य	– 	कुलार्णवतन्त्रम्  – १७-७२\\
तन्निष्ठस्तद्गतप्राण– 	कुलार्णवतन्त्रम्  – १५-११३\\
तस्मात् सर्वप्रयत्नेन	– 	कुलार्णवतन्त्रम्  – १४-८\\
तस्मात् सर्वप्रयत्नेन	– 	कुलार्णवतन्त्रम्  – १५-७६\\
तस्मात् सिद्धान्तं– 	कुलार्णवतन्त्रम्  – १४-४\\
तस्माद् यन्त्रं– 	कुलार्णवतन्त्रम्  – ६-८८\\
तस्यार्पिताधिकारस्य	– 	कुलार्णवतन्त्रम्  – १४-७\\
तारतम्यं समालोक्य	– 	पुराणसारः   – ४०-६७	\\
तावदारधयेच्छिष्यः	– 	कुलार्णवतन्त्रम्  – १२-२०\\
तावदारादयेच्छिष्यः	– 	कुलार्णवतन्त्रम्  – १२-२०\\
तावदार्त्तिर्भयं	– कुलार्णवतन्त्रम्  – १२-१५\\
तावद् भ्रमन्ति– 	कुलार्णवतन्त्रम्  – १२-१६\\
तुषेण बद्धो व्रीहिः– 	 कुलार्णवतन्त्रम्  – ९-४३\\
तूलकम्बलवस्त्राणां	– 	कुलार्णवतन्त्रम्  – १५-३३\\
तेजस्वी तपसा– 	पुराणसारः   – ३३-३०\\
तेजोवन्नन्भूमयै– 	पुराणसारः   – ३३-२८\\
त्यक्ताधिव्याधि– 	कुलार्णवतन्त्रम्  – १३-३१\\
त्रिंशद्भिः स्याद्धनं– 	कुलार्णवतन्त्रम्  – १५-५१\\
{\large द}\\
दक्षमल्पाशिनं– 	कुलार्णवतन्त्रम्  – १३-२५\\
दर्शनं धेनुमुद्राया– 	 कुलार्णवतन्त्रम्  – १७-९३\\
दह्यन्ते ध्मार्यमानानां	– 	कुलार्णवतन्त्रम्  – १५-४३\\
दिव्यभावप्रदानाच्च	– 	कुलार्णवतन्त्रम्  – ९-५१\\
दिव्यानामपि	– 	पुराणसारः   – ४०-६५\\
दीक्षा च द्विविधा– 	कुलार्णवतन्त्रम्  – १४-७८\\
दीक्षाग्निदग्धकर्मासौ	– 	कुलार्णवतन्त्रम्  – १४-९०\\
दीक्षापूर्वं	– 	कुलार्णवतन्त्रम्  – १५-१५\\
दीक्षासमयसम्प्राप्ता	– 	कुलार्णवतन्त्रम्  – १४-२९\\
दीपदर्शनमात्रेण	– 	कुलार्णवतन्त्रम्  – 	– 	१२-११६\\
दीर्घाज्ञानमहाध्वान्ता	– 	कुलार्णवतन्त्रम्  – १७-७८	\\
दूरस्थो नार्चयेदेनं– 	मनुस्मृतिः  – २-२०२\\
दूरादाह्रृत्य समिधः	– 	मनुस्मृतिः  – २-१८६\\
दृश्यं विना स्थिरा	– 	कुलार्णवतन्त्रम्  – १३-७०\\
दृश्यन्ते षड्गुणा	– 	कुलार्णवतन्त्रम्  – १४-६५\\
देवं पूजार्थमाह्वान– 	 कुलार्णवतन्त्रम्  – १७-९०\\
देवतागुरु शास्राणां	– 	कुलार्णवतन्त्रम्  – ११-७३\\
देवताङ्गे षडङ्गानां– 	 कुलार्णवतन्त्रम्  – १७-९२\\
देवतारूपधारि– 	कुलार्णवतन्त्रम्  – १७-१४\\
देवताविग्रहो भूत्वा	– 	पुराणसारः   – ४०-६३	\\
देवस्य मन्तरूपस्य	– 	कुलार्णवतन्त्रम्  – ६-८४\\
देवान् गुरून् – 	कुलार्णवतन्त्रम्  – 	–  १२-११३\\
देवास्तमेव– 	कुलार्णवतन्त्रम्  – १४-५\\
देवि तीव्रतरा– 	कुलार्णवतन्त्रम्  – १४-६०\\
देशं कालं वयो वित्तं	– 	कुलार्णवतन्त्रम्  – ११-१०४\\
देहमास्थाय भक्तानां	– 	कुलार्णवतन्त्रम्  – १७-५५\\
देहो देवालयो	– 	 कुलार्णवतन्त्रम्  – ९-४१\\
द्यूतकौतुकमल्लादि	– 	कुलार्णवतन्त्रम्  – १२-८८\\
द्यूतञ्च जनवादं– 	मनुस्मृतिः  – २-१७९\\
द्विजो यो दीक्षितः– 	कुलार्णवतन्त्रम्  – १४-९८\\
द्वीपाद्वीपान्तरं देवि	– 	कुलार्णवतन्त्रम्  – १२-१०२\\
{\large ध}\\
धनार्थं गम्यते– 	कुलार्णवतन्त्रम्  – १५-१०२\\
धनेच्छाभयलोभाद्यै	– 	कुलार्णवतन्त्रम्  – १४-१८\\
धर्मार्थकामैः किन्तस्य	– 	कुलार्णवतन्त्रम्  – १२-३१\\
धर्मार्थौ यत्र न स्यातां	– 	मनुस्मृतिः  – २-११२\\
धूतशेषमहादोष– 	 कुलार्णवतन्त्रम्  – १७-७७\\
ध्यानन्तु द्विविधं– 	कुलार्णवतन्त्रम्  – ९-३\\
ध्यायतां क्षणमात्रं– 	कुलार्णवतन्त्रम्  – ९-३१\\
}

\colchunk{% left column

{\large न}\\
न देयं नोपभोग्यञ्च	– 	पुराणसारः   – ३३-४१	\\
न पद्मासनतो– 	कुलार्णवतन्त्रम्  – ९-३०\\
न मे प्रियश्चतुर्वेदी– 	कुलार्णवतन्त्रम्  – १२-२७\\
न मे मानावमानौ– 	पुराणसारः   – ३३-४६\\
न वियोगं गुरोः– 	कुलार्णवतन्त्रम्  – १२-५७\\
न विशेदासने– 	कुलार्णवतन्त्रम्  – १२-१०७\\
नातिस्नेहःप्रसङ्गो वा	– 	पुराणसारः   – ३३-३३	\\
नापृष्टः कस्यचिद्– 	मनुस्मृतिः  – २-११०\\
निग्रहेनुग्रहे	 –  कुलार्णवतन्त्रम्  – १२-९९\\
नित्यं नैमित्तिकम्	– 	कुलार्णवतन्त्रम्  – ११-२०२\\
नित्यं स्नात्वा शुचिः	– 	मनुस्मृतिः  – २-१७६\\
नित्यमुद्धृतपाणिः– 	मनुस्मृतिः  – २-१९३\\
निमील्य नयने	– 	कुलार्णवतन्त्रम्  – १४-५५\\
निषेकादीनी कर्माणि– 	मनुस्मृतिः  – २-१४२\\
निषेकाद्याः श्मशानान्ताः	 – पुराणसारः   – ३३-३१	\\
निःसङ्गश्च विसङ्गश्च	– 	कुलार्णवतन्त्रम्  – ९-४०\\
निस्स्पृहो नित्यसन्तुष्टः	– 	कुलार्णवतन्त्रम्  – ९-४६\\
नीचं शय्यासनं– 	मनुस्मृतिः  – २-१९८\\
नोदाहरेदस्य– 	मनुस्मृतिः  – २-१९९\\
न्यायोपार्जितवित्ता	– 	कुलार्णवतन्त्रम्  – १७-५६\\
{\large प}\\
पञ्चाङ्गोपासनेनेष्ट– 	कुलार्णवतन्त्रम्  – १७-८७\\
पञ्चैते कार्यभूताः– 	कुलार्णवतन्त्रम्  – १३-१२९\\
पदाऽपि युवतीं– 	पुराणसारः   – ३३-४०\\
पदे पदेश्वमेधस्य	– 	कुलार्णवतन्त्रम्  – १२-६५\\
पद्मस्वस्तिकवीरादि	– 	कुलार्णवतन्त्रम्  – १५-३४\\
पद्मादिचतुरशीति– 	कुलार्णवतन्त्रम्  – १२-८९\\
परब्रह्मकरं– 	पुराणसारः   – ३३-१४\\
पराङ्मुखस्याभिमुखो	– 	मनुस्मृतिः  – २-१९७\\
परिवादात्स्वरो– 	मनुस्मृतिः  – २-२०१\\
पश्चात्पदेन– 	कुलार्णवतन्त्रम्  – १२-१०६\\
पादप्रक्षालनं	–    कुलार्णवतन्त्रम्  – १२-८५\\
पादप्रसारणं वादं	– 	कुलार्णवतन्त्रम्  – १२-८७\\
पारम्पर्यागमाम्नायं	– 	कुलार्णवतन्त्रम्  – ११-४६\\
पालनाद्दुरितच्छेदात्	– 	कुलार्णवतन्त्रम्  – १७-३३	\\
पाशबद्धः पशुर्ज्ञेयः	– 	कुलार्णवतन्त्रम्  – १२-९१\\
पिण्डब्रह्माण्डयोरैक्यं	– 	कुलार्णवतन्त्रम्  – १२-८८\\
पुण्यक्षेत्रं नदीतीरं– 	कुलार्णवतन्त्रम्  – १५-२२\\
पुण्यपापादिकथना	– 	कुलार्णवतन्त्रम्  – १७-३९	\\
पुण्यसंवर्द्धनाच्चापि	– 	कुलार्णवतन्त्रम्  – १७-७६	\\
पूजा त्रैकालिकी– 	कुलार्णवतन्त्रम्  – १५-८\\
पूजाकोटिसमं– 	कुलार्णवतन्त्रम्  – ९-३६\\
पूजाहोमाश्रमाचार	– 	कुलार्णवतन्त्रम्  – १३-११८\\
पूर्वजन्मानुशमना– 	 कुलार्णवतन्त्रम्  – १७-७०\\
पूर्वजन्मानुशमना– 	कुलार्णवतन्त्रम्  – १७-७०\\
पृथिवी वायुराकाश	– 	पुराणसारः   – ३३-२४\\
प्रकाशानन्दजननात्	– 	कुलार्णवतन्त्रम्  – १७-८४	\\
प्रतिवातेनुवाते– 	मनुस्मृतिः  – २-२०३\\
प्रतिश्रवणसंभाषे– 	मनुस्मृतिः  – २-१९५\\
प्राणायामैः विशुद्धात्म	– 	कुलार्णवतन्त्रम्  – १५-४४\\
प्रेरकः सूचकश्चैव– 	कुलार्णवतन्त्रम्  – १३-१२८\\
{\large ब}\\
बन्धनं योनिमुद्राया	– 	कुलार्णवतन्त्रम्  – १३-९२\\
बहुनात्र किमुक्तेन– 	कुलार्णवतन्त्रम्  – ११-१०७\\
बहुप्रकारविचरद्– 	 कुलार्णवतन्त्रम्  – १७-८१\\
बाह्यव्यापारनिर्मुक्तो	– 	कुलार्णवतन्त्रम्  – १५-६१\\
ब्रह्मणःप्रणवं– 	मनुस्मृतिः  – २-७४\\
ब्रह्मविष्णुमहेशादि –  कुलार्णवतन्त्रम्  – १-४६\\
ब्रह्माकारं मनोरूपं	– 	कुलार्णवतन्त्रम्  – १३-१२०\\
ब्रह्माकारं मनोरूपं	– 	कुलार्णवतन्त्रम्  – १३-१२०\\
ब्रह्मारम्भेवसाने– 	मनुस्मृतिः  – २-७१\\
{\large भ}\\
भक्त्या वित्तानुसारेण	– 	कुलार्णवतन्त्रम्  – १२-६९\\
भक्या तुष्टेन– 	कुलार्णवतन्त्रम्  – १४-२३\\
भक्ष्यं हविष्यं	– 	कुलार्णवतन्त्रम्  – १५-७४\\
भजनात् परया– 	 कुलार्णवतन्त्रम्  – १७-२९\\
भाषणं पाठनं गानं	– 	कुलार्णवतन्त्रम्  – १२-१०३\\
भूतैराक्रम्यमाणोपि	– 	पुराणसारः   – ३३-२६	\\
भोगमोक्षार्थिनां– कुलार्णवतन्त्रम्  – १२-४०\\
{\large म}\\
मङ्गलत्वाच्च डाकिन्या	– 	कुलार्णवतन्त्रम्  – १७-५९	\\
मधुपो हारिणो– 	पुराणसारः   – ३३-२५\\
मधुलुब्धो यथा– 	कुलार्णवतन्त्रम्  – १३-१३३\\
मध्वाज्यदधिभिः	– 	कुलार्णवतन्त्रम्  – १७-९७	\\
मननात्तत्वरूपस्य	– 	कुलार्णवतन्त्रम्  – १७-५४\\
मनसापि न काङ्क्षन्ते	– 	कुलार्णवतन्त्रम्  – १२-२१\\
मनो दीक्षा द्विधा	– 	कुलार्णवतन्त्रम्  – १४-५७\\
मनोवाक्तनुभिर्नित्यं	– 	कुलार्णवतन्त्रम्  – १३-३३\\
मनोऽन्यत्र– 	कुलार्णवतन्त्रम्  – १५-१००\\
मन्त्रत्यागाद्भवे– 	कुलार्णवतन्त्रम्  – १२-४७\\
मन्त्रहीनं क्रियाहीनं	– 	कुलार्णवतन्त्रम्  – ६-८०\\
मन्त्रार्थं मन्त्रचैतन्यं	– 	कुलार्णवतन्त्रम्  – १५-५९\\
मन्त्रिदोषश्च– 	कुलार्णवतन्त्रम्  – ११-१०९\\
मन्त्रौषधेः यथा	– 	कुलार्णवतन्त्रम्  – १४-८३\\
मलिनाम्बरकेशा– 	कुलार्णवतन्त्रम्  – १५-१०५\\
महामुद्रां नभोमुद्राम्	– 	कुलार्णवतन्त्रम्  – १२-८५\\
मुक्तिदा गुरुवागेका	– 	कुलार्णवतन्त्रम्  – १-१०७\\
मुदं कुर्वन्ति– 	 कुलार्णवतन्त्रम्  – १७-५७\\
मूलादिब्रह्मरन्ध्रान्त	– 	कुलार्णवतन्त्रम्  – १२-९४\\
मोहाध्वान्तप्रशमनात्	– 	कुलार्णवतन्त्रम्  – १७-७९	\\
{\large य}\\
यः क्षणेनात्मसामर्थ्यं	– 	कुलार्णवतन्त्रम्  – १३-१००\\
यः प्रसन्नः क्षणार्धेन	– 	कुलार्णवतन्त्रम्  – १२-९८\\
य सध्यः प्रत्ययकारं	– 	कुलार्णवतन्त्रम्  – १२-१०१\\
यत्र यत्र मनो देही	– 	पुराणसारः   – ३३-५१\\
यथा कपिश्च– 	कुलार्णवतन्त्रम्  – १४-३२\\
यथा कूर्मः स्वतनयान्	– 	कुलार्णवतन्त्रम्  – १४-३७\\
यथा खनत्खनित्रेण	– 	मनुस्मृतिः  – २-२१८\\
यथा गाढान्धकारस्थो	– 	कुलार्णवतन्त्रम्  – ९-१८\\
यथा ग्रामगतं तोयं	–  कुलार्णवतन्त्रम्  – ९-७८\\
यथा घटश्च– 	कुलार्णवतन्त्रम्  – १३-६४\\
यथा जले जलं– 	कुलार्णवतन्त्रम्  – ९-१५\\
यथा दीप्तानलः	– 	कुलार्णवतन्त्रम्  – 	– १२-११४\\
यथा देवस्तथा– 	कुलार्णवतन्त्रम्  – १३-६५\\
यथा ध्यानस्य– 	कुलार्णवतन्त्रम्  – ९-१६\\
यथा निमीलने– 	कुलार्णवतन्त्रम्  – ९-१९\\
यथा पक्षी स्वपक्षाभ्यां	– 	कुलार्णवतन्त्रम्  – १४-३५\\
यथा पिपीलीका– 	कुलार्णवतन्त्रम्  – १४-३१\\
यथा महानीलोद्धूतं –  कुलार्णवतन्त्रम्  – 	– 	१२-११५\\
यथा वियद्गमः– 	कुलार्णवतन्त्रम्  – १४-३३\\
यथा हसति लोकोदयं	– 	कुलार्णवतन्त्रम्  – ९-७३\\
यथैवमात्मन्यवरूद्धचित्तो	– 	पुराणसारः   – ३३-४८	\\
यथोर्णनाभिर्हृदयादूर्णां	– 	पुराणसारः   – ३३-५०	\\
यदत्र नात्र निर्भासः	– 	कुलार्णवतन्त्रम्  – ९-९\\
यदस्ति वेधकाले– 	कुलार्णवतन्त्रम्  – १४-६३\\
यदा दद्यात् स्वशिष्याय	– 	कुलार्णवतन्त्रम्  – १२-१९\\
यदृच्छया श्रुतं– 	कुलार्णवतन्त्रम्  – १५-२०\\
यद् यदङ्गं विहीयेत	– 	कुलार्णवतन्त्रम्  – १५-९\\
यन्त्रं मन्त्रमयं– 	कुलार्णवतन्त्रम्  – ६-८५\\
यन्मोक्षसुखदं ज्ञानं	– 	पुराणसारः   – ३३-११	\\
यमभूतादिसर्वेभ्यः	– 	कुलार्णवतन्त्रम्  – १७-६१	\\
यया चोन्मीलितात्मानो	– 	कुलार्णवतन्त्रम्  – १४-८६	\\
यःशिवःसर्वर्गः	–   कुलार्णवतन्त्रम्  – १३-४१\\
यस्तत्वविन्महेशानि	– 	कुलार्णवतन्त्रम्  – १३-१२२\\
यस्य देवे परा भक्ति	– 	कुलार्णवतन्त्रम्  – १२-३३\\
यस्य देवे पराभक्तिः	– 	कुलार्णवतन्त्रम्  – १२-३३\\
यस्यानुभवपर्यन्तं– 	कुलार्णवतन्त्रम्  – १२-१११\\
यस्यान्न पानपुष्टाङ्गः	– 	कुलार्णवतन्त्रम्  – १५-७५\\
यावदिन्दियसन्तापं	– 	कुलार्णवतन्त्रम्  – १७-३६	\\
यावदेहाभिमानश्च– 	कुलार्णवतन्त्रम्  – १-११५\\
ये दत्वा सहजानन्दं	– 	कुलार्णवतन्त्रम्  – १२-९७\\
ये वा पराञ्च पश्यन्तीं	– 	कुलार्णवतन्त्रम्  – १२-७७\\
येन वा दर्शिते तत्वे	– 	कुलार्णवतन्त्रम्  – १२-९६\\
यो न्यासकवचच्छन्दो	– 	कुलार्णवतन्त्रम्  – १५-४६\\
योगमार्गेण शिष्यस्य	– 	पुराणसारः   – ४०-६०	\\
योनिमुद्रानुसन्धानात्	– 	कुलार्णवतन्त्रम्  – १७-२१\\
योषिद्धिरण्याभरणा	– 	पुराणसारः   – ३३-३७	\\
यौगिकी मानसी– 	पुराणसारः   – ४०-५९\\
{\large र}\\
रसेन्द्रेण यथा– 	कुलार्णवतन्त्रम्  – १४-८९\\
रिक्तहस्तश्च	– 	कुलार्णवतन्त्रम्  – १२-१२०\\
{\large ल}\\
लिङ्गस्थण्डिल– 	कुलार्णवतन्त्रम्  – ६-७३\\
लौकिकं वैदिकं– 	मनुस्मृतिः  – २-११७\\
{\large व}\\
वर्जयेन्मधु मांसं– 	मनुस्मृतिः  – २-१७७\\
वर्णाश्रमकुलाचार	–   कुलार्णवतन्त्रम्  – १२-१०९\\
वादार्थं पठ्यते– 	कुलार्णवतन्त्रम्  – १५-१०१\\
वासे बहूनां कलहो	– 	पुराणसारः   – ३३-४७	\\
विण्मूत्रत्यागशेषा– 	कुलार्णवतन्त्रम्  – १५-१०४\\
विदिते परमे तत्त्वे	– 	कुलार्णवतन्त्रम्  – ९-२१\\
विद्धस्तु वेधयेद्देवि– 	कुलार्णवतन्त्रम्  – १३-१२४\\
विद्ययैव समं कामं	– 	मनुस्मृतिः  – २-११३\\
विधवायाःसुतादेशात्	– 	कुलार्णवतन्त्रम्  – १४-१०५\\
विना दीक्षां न– 	कुलार्णवतन्त्रम्  – १४-३\\
विनिक्षिप्तां गतायातां	– 	कुलार्णवतन्त्रम्  – १३-९३\\
विप्रः षड्गुणयुक्त– 	कुलार्णवतन्त्रम्  – १२-२८\\
विशिष्टं दीयते	– 	पुराणसारः   – ४०-५८\\
विश्वासाय नमस्तस्मै	– 	कुलार्णवतन्त्रम्  – १२-४२\\
विषयेष्वाविशन्योगी	– 	पुराणसारः   – ३३-२७	\\
वीरासनं सुदुर्वाक्यं	– 	कुलार्णवतन्त्रम्  – १२-८६\\
वेदमनूच्याचार्यो– 	शीक्षावल्ली	– ११. अनुवाक\\
वेदयज्ञैरहीनानां– 	मनुस्मृतिः  – २-१८३\\
वेदिताऽखिल– 	 कुलार्णवतन्त्रम्  – १७-३८\\
वेधदीक्षाकरो– 	कुलार्णवतन्त्रम्  – १४-६६\\
व्यत्यस्तपाणिना	– 	मनुस्मृतिः  – २-७२\\
{\large श}\\
शक्तिपातानुसारेण	– 	कुलार्णवतन्त्रम्  – १४-३८\\
शक्तिसिद्धिसुसिद्ध्यर्थं	– 	कुलार्णवतन्त्रम्  – १४-९\\
शङ्कया भक्षितं सर्वं	– 	कुलार्णवतन्त्रम्  – १२-११२\\
शयीताहानि भूरीणि	– 	पुराणसारः   – ३३-३५	\\
शय्यासने– 	मनुस्मृतिः  – २-११९\\
शरीरं चैव वाचं– 	मनुस्मृतिः  – २-२०५\\
शरीरमर्थं प्राणांश्च	– 	कुलार्णवतन्त्रम्  – १७-३०\\
शरीरमर्थं प्राणांश्च	– 	कुलार्णवतन्त्रम्  – १७-३०\\
शरीरमिव जीवस्य	– 	कुलार्णवतन्त्रम्  – ६-८७\\
शरीरवित्तप्राणैश्च– 	कुलार्णवतन्त्रम्  – १२-४९\\
शरीरस्य न संस्कारो– 	कुलार्णवतन्त्रम्  – १४-८१\\
शान्तःशुचिमिताहारो	– 	कुलार्णवतन्त्रम्  – १५-०११\\
शासनादनिशं– 	 कुलार्णवतन्त्रम्  – १७-४०\\
शिवलिङ्गे शिला– 	कुलार्णवतन्त्रम्  – १४-९२\\
शिवादिक्षितिपर्यन्तं	– 	कुलार्णवतन्त्रम्  – १३-८६\\
शिवादिगुरुपर्यन्तं	– 	कुलार्णवतन्त्रम्  – १२-९५\\
शिवे मन्त्रे गुरौ– 	पुराणसारः   – ३३-१६\\
शिष्येणापि तथा– 	कुलार्णवतन्त्रम्  – १२-२४\\
शिष्योपि गुरुणाज्ञप्तं	– 	कुलार्णवतन्त्रम्  – ११-१०५\\
शिष्योपि लक्षणैरेतैः	– 	कुलार्णवतन्त्रम्  – १४-२५\\
शूद्रसङ्करजाती– 	कुलार्णवतन्त्रम्  – १४-१०३\\
श्रीगुरुं  प्राकृतैः सार्धं	– 	कुलार्णवतन्त्रम्  – १२-४७\\
श्रीगुरुं परमं तत्त्वं –  कुलार्णवतन्त्रम्  – १३-४९\\
श्रीगुरु लक्षणोपेतं	– 	कुलार्णवतन्त्रम्  – १३-१३०\\
श्रीगुरोः स्मरणे– 	कुलार्णवतन्त्रम्  – १४-२३\\
श्रीगुरोःपादुकां– 	कुलार्णवतन्त्रम्  – १२-४५\\
श्रीगुरौ निश्चला– 	कुलार्णवतन्त्रम्  – १२-३७\\
श्रेयोर्थी चेन्नरो– 	पुराणसारः   – ३९-६२\\
{\large स}\\
स शिवो गुरुरूपेण	– 	कुलार्णवतन्त्रम्  – १२-३२\\
सकुण्डमण्डपा– 	पुराणसारः   – ४०-६६\\
सकृदुच्चरितेप्येवं– 	कुलार्णवतन्त्रम्  – १५-६४\\
सङ्गदुःखपरित्यागात्	– 	कुलार्णवतन्त्रम्  – १७-२२\\
सच्छिष्यन्तु कुशानि	– 	कुलार्णवतन्त्रम्  – १३-२३\\
सच्छिष्यायाति– 	कुलार्णवतन्त्रम्  – १४-१६\\
सञ्चित्य हृदये	– 	पुराणसारः   – ४०-६४\\
सद्भक्तरक्षणायैव	  – 	कुलार्णवतन्त्रम्  – १३-५५\\
सम्पूज्य सावृत्तिं– 	 कुलार्णवतन्त्रम्  – १७-८९\\
सर्वं वापि चरेद्ग्रामं	– 	मनुस्मृतिः  – २-१८५\\
सर्वकार्याति कुशलं	– 	कुलार्णवतन्त्रम्  – १३-२६\\
सर्वलक्षणसम्पन्नो– 	कुलार्णवतन्त्रम्  – १३-११७\\
सर्वलक्षणहिनोपि	– 	कुलार्णवतन्त्रम्  – १३-१२१\\
सर्वशोषी यथा–  कुलार्णवतन्त्रम्  – ९-७६\\
सर्वसिद्धिफलोपेतो	– 	कुलार्णवतन्त्रम्  – १२-१७\\
सर्वस्पर्शी यथा– 	कुलार्णवतन्त्रम्  – ९-७७\\
सर्वस्वमपि यो दद्याद्	– 	कुलार्णवतन्त्रम्  – १२-७०\\
संसार मोहनाशाय	– 	कुलार्णवतन्त्रम्  – १-१७\\
संसारभयभीतस्य– 	कुलार्णवतन्त्रम्  – १२-९८\\
संसारविषयेऽत्यर्थं	– 	पुराणसारः   – ३३-१२	\\
संसारे दुःखभूयिष्ठे	– 	कुलार्णवतन्त्रम्  – १५-७\\
साधनेषु प्रशस्तानि	– 	कुलार्णवतन्त्रम्  – १५-२४\\
सामन्यतो निषिद्धं	– 	कुलार्णवतन्त्रम्  – १२-९७\\
सामिषं कुररं– 	पुराणसारः   – ३३-४५\\
सारसंग्रहणाच्चैव– 	 कुलार्णवतन्त्रम्  – १७-२८\\
सिद्धमन्त्राद्	– 	कुलार्णवतन्त्रम्  – १५-१४\\
सिद्धार्थमक्षतञ्चैव– 	 कुलार्णवतन्त्रम्  – १७-९६\\
सूर्यस्याग्नेर्गुरो– 	कुलार्णवतन्त्रम्  – १५-२५\\
सेवतेमांस्तु– 	मनुस्मृतिः  – २-१७५\\
स्तोकं स्तोकं ग्रसेद्भासं	– 	पुराणसारः   – ३३-३८	\\
स्तोकस्तोकेन	– 	 कुलार्णवतन्त्रम्  – १७-३५\\
स्थानान्तरगतेचार्ये	– 	कुलार्णवतन्त्रम्  – १२-१००\\
स्थिरार्थं मनसः– 	कुलार्णवतन्त्रम्  – ९-४\\
स्पर्शाख्या देवि– 	 कुलार्णवतन्त्रम्  – १४-३४\\
स्मरणोत्सुकनिष्ठानां	– 	कुलार्णवतन्त्रम्  – १७-४१	\\
स्याद्वेदाध्ययने– 	कुलार्णवतन्त्रम्  – १४-१०६\\
स्रीद्विष्टं गुरुभिः– 	कुलार्णवतन्त्रम्  – १२-११४\\
स्वच्छः प्रकृतितः– 	पुराणसारः   – ३३-२९	\\
स्वच्छदेहाम्बरं – कुलार्णवतन्त्रम्  – १३-२४\\
स्वयं वेद्ये परे तत्त्वे	– 	कुलार्णवतन्त्रम्  – १३-११९\\
स्वयमाचरते शिष्यान्	– 	कुलार्णवतन्त्रम्  – १७-११	\\
स्वस्तुतौ परनिन्दायां	– 	कुलार्णवतन्त्रम्  – १३-३०\\
स्वान्तश्शान्ति– 	कुलार्णवतन्त्रम्  – १७-१५\\
स्वापत्यानि यथा– 	कुलार्णवतन्त्रम्  – १४-३६\\
{\large ह}\\
हस्ते शिवं गुरुं– 	कुलार्णवतन्त्रम्  – १४-५३\\
हार्द्रमन्त्रमयं– 	पुराणसारः   – ४०-६२\\
हितसत्यमितस्मेर	– 	कुलार्णवतन्त्रम्  – १३-२९\\
हीनान्नवस्रवेषः– 	मनुस्मृतिः  – २-१९४\\
ह्रत्कण्ठग्रन्थिभेदश्च	– 	कुलार्णवतन्त्रम्  – १५-६३
}
\end{parcolumns}
\end{raggedright}
\begin{center}\includegraphics[scale=0.08]{end}\end{center}
