%\bfseries
\thispagestyle{empty}
\fontsize{14}{16}\selectfont
\section{उपोद्घातः}

	द्विविधो हि शास्त्रधर्मः प्रवर्तनं वस्तुस्वरूपान्वाख्यानञ्च~। आद्यं वेदस्मृतिधर्मशास्त्रागमादीनाम्~। पुरुषस्य हिते प्रवृत्तिम् अहितान्निवृत्तिं वा बोधयत् शास्त्रं प्रवर्तनपरं भवति~।\footnote{\small तद्यथा - ‘सत्यं वद धर्मं चर’ इत्यादिश्रुतिः प्रवृत्तिम् उपदिशति, ‘मा हिंस्यात् सर्वा भूतानि’ इत्यादिश्च  अनिष्टात् हिंसनात् निवृत्तिं बोधयति~। एवं ‘ब्राह्मे मुहूर्ते बुद्ध्येत’ इत्यादिस्मृतिः प्रवृत्तिपरा, ‘न दत्त्वा परिकीर्तयेत्’ इत्यादिः निवृत्तिपरा~।}  सोऽयं प्रवर्तनलक्षणः शास्त्रधर्मः मुख्यः। अत एवोक्तं शास्त्रलक्षणम् -

\begin{verse}[\versewidth]
		प्रवृत्तिर्वा निवृत्तिर्वा नित्येन कृतकेन वा~।\\[-6pt]
		पुंसां येनोपदिश्येत तच्छास्त्रमभिधीयते~॥इति~।
\end{verse}

	द्वितीयः शास्त्रधर्मः वस्तुस्वरूपान्वाख्यानं पूर्वोत्तरमीमांसान्यायवैशेषिकादीनाम्~। प्रवर्तनशास्त्रविषयीभूतानां तत्सम्बद्धानाञ्च वस्तूनां तत्त्वं विवेचयत्  शास्त्रं वस्तुस्वरूपान्वाख्यानपरं भवति~।\footnote{\small तद्यथा ‘धर्मेण पापमपनुदति’ इत्यादिश्रुतिविहितधर्मस्य स्वरूपविवेचनपरम् ‘अथातो धर्मजिज्ञासा’ इत्यादिसूत्रजालघटितं पूर्वमीमांसाशास्त्रं, ‘यतोऽभ्युदयनिश्श्रेयससिद्धिः स धर्मः’ इत्यादिसूत्रघटितं वैशेषिकदर्शनं, ‘ब्रह्मविद् ब्रह्मैव भवति’ ‘आत्मा वाऽरे द्रष्टव्यः’ इत्यादिश्रुतिप्रतिपाद्यस्य ब्रह्मणः आत्मनः स्वरूपविवेचनपरम् ‘अथातो ब्रह्मजिज्ञासा’ इत्यादिसूत्रसन्दर्भघटितम् उत्तरमीमांसादर्शनम्, ‘इच्छाद्वेषप्रयत्नसुखदुःखज्ञानान्यात्मनो लिङ्गम्’ इत्यादिसूत्रघटितं न्यायदर्शनम्~। ‘योगी युञ्जीत सततम्’ इत्यादिस्मृतिविहितयोगस्य तत्त्वविवेचकं ‘योगश्चित्तवृत्तिनिरोधः’ इत्यादिसूत्रोपबृंहितं योगदर्शनम्~। ‘प्रकृतिं पुरुषं चैव विद्ध्यनादी उभावपि’ इत्यादिस्मृतिप्रतिपाद्यायाः प्रकृतेः तत्त्वविवेचकं  ‘मूलप्रकृतिरविकृतिः’ इत्यादिकारिकोपबृंहितं साङ्ख्यदर्शनञ्च  वस्तुतत्त्वान्वाख्यानपराणि भवन्ति~।} क्वचिच्च शास्त्रे तत्त्वान्वाख्यानं प्रवर्तनमुभयं भवति~। यथा आयुर्वेदे~। 

	शिक्षाव्याकरणच्छन्दांसि शब्दतत्त्वान्वाख्यानपराणि~। वस्तुतत्त्वान्वाख्यानञ्च हिताहितसाधनतानिर्णयस्य दार्ढ्यसम्पादने उपयुज्यते, न ततः ऊर्ध्वम्~। अत्र एवोक्तम्~-
\begin{verse}[\versewidth]
		ग्रन्थमभ्यस्य मेधावी ज्ञानविज्ञानतत्परः~।\\[-6pt]
		पलालमिव धान्यार्थी त्यजेद् ग्रन्थम् अशेषतः~॥ इति~।
\end{verse}

	वस्तूनां तत्त्वनिर्णयेन पुरुषः हिताहितसाधनतानिर्णयं दृढीकृत्य प्रवर्तनलक्षणशास्त्रेण ज्ञापितानां हितसाधनानाम् अनुसरणे अहितसाधनानां निवारणे च प्रवर्तेत~। ततश्च हितं प्राप्नुयात् , अहितञ्च निवारयेदिति~। इत्थञ्च हिताहितप्राप्तिपरिहाररूपं  प्रयोजनं वस्तुतत्त्वान्वाख्यानपरशास्त्राणां प्रवर्तनपरशास्त्रं द्वारीकृत्यैव सम्भवति, न तु साक्षात्~। 

	वस्तुतत्त्वान्वाख्यानपरशास्त्राणां पुनः समाने विषये बहुधा मतभेदो दृश्यते~।\footnote{\small तद्यथा जगदुत्पत्तिविषये साङ्ख्यानां परिणामवादः नैय्यायिकानाम् आरम्भवादः वेदान्तिनां विवर्तवादः~। ख्यातिविषये नैय्यायिकानाम् अन्यथाख्यातिवादः प्राभाकरमीमांसकानाम् अख्यातिवादः वेदान्तिनाम् अनिर्वचनीयख्यातिवादः इत्येवम्~।\clearpage } स च मतभेदः मन्दमध्यमोत्तमाधिकारिणाम् इव संस्कारवैचित्र्येण पुनरवान्तरभेदमापन्नानां सर्वेषां लोकानां सङ्ग्राहकतया औचित्यमेव आवहति~। द्रष्टॄणां संस्कारवैचित्र्येण दृष्टीनां वैचित्र्यं लोकप्रसिद्धम्~। तद्यथा - 
\begin{verse}[\versewidth]
		परिव्राट्‌कामुकशुनाम् एकस्यां प्रमदातनौ~।\\[-6pt]
		कुणपः कामिनी भक्ष्यम् इति तिस्रो विकल्पनाः~॥ इति~।
\end{verse}

	शास्त्रेऽपि प्रसिद्धम्, एकस्मिन्नेव वस्तुनि द्रष्टॄणां दृष्टिवैचित्र्यम्~। तद्यथा ‘एकं सद् विप्रा बहुधा वदन्त्यग्निं यमं मातरिश्वानमाहुः’(ऋग्वेदः-१.१०४,६४) इति~। यत् तत्त्वान्वाख्यानं यदीयसंस्कारानुरूपं स पुरुषः तदेव तत्त्वान्वाख्यानमनुसरेत्~। हिताहितसाधनतानिर्णये दार्ढ्योत्पादनेन उपकुर्वतां तत्त्वान्वाख्यानानां परस्परविरोधश्च उपेक्षणार्हः~। तत्परीक्षणस्य काकदन्तपरीक्षावत्\footnote{\small काकस्य कति वा दन्ताः मेषस्याण्डं कियत्~। का वार्ता सिन्धुसौवीरेष्वेषा मूर्खविचारणा॥}  व्यर्थत्वात्~। शास्त्रवासनारूपतया परमप्रयोजनावाप्तौ प्रतिबन्धकत्वाच्च~। तद्यथा -
\begin{verse}[\versewidth]
		लोकवासनया जन्तोः शास्त्रवासनयापि च~।\\[-6pt]
		देहवासनया ज्ञानं यथावन्नैव जायते~॥ इति ।\\ \hfill (मुक्तिकोपनिषत् २.२) 
\end{verse}

  विरोधादिपरीक्षणाय प्रवृत्ता व्याख्योपव्याख्यानपरम्परा हि तत्तद्व्याख्यातृसंस्कारानुसारिणी प्रत्यक्षनिर्णयमलभमानस्य लोकस्य विप्रतिपत्तिं शमयितुं नाऽलम्~। तद्यथा -
\begin{verse}[\versewidth]
		अन्यथा परमं तत्त्वं जनाः क्लिश्यन्ति चान्यथा~।\\[-6pt]
		अन्यथा शास्त्रसद्भावो व्याख्यां कुर्वन्ति चान्यथा~॥\\ \hfill (कु.त.१.१२) 
\end{verse}

	प्रवर्तनलक्षणशास्त्रेषु वेदस्य इव आगमस्यापि अपौरुषेयत्वम् ईश्वरोक्तत्त्वं वा आगमसम्प्रदायप्रवर्तकैः अङ्गीक्रियते~। आगमशब्दः प्रायः तन्त्रशब्दपर्यायतया उपयुज्यते~। तन्त्रप्रस्थानं स्वतन्त्रं  वेदादिनिरपेक्षं सदपि वेदप्रामाण्याभ्युपगन्तृभिः प्रमाणत्वेन अनुस्रियते~। प्रवर्तनलक्षणशास्त्रप्रस्थानेषु एषु प्रत्यक्षानुभवेनैव शास्त्राधीतेः कृतार्थता, न तु शाब्दबोधमात्रेण~। 
\begin{verse}[\versewidth]
		कथयन्त्युन्मनीभावं स्वयं नानुभवन्ति हि~। \\[-6pt]
		अहङ्कारहताः केचिद् उपदेशविवर्जिताः~॥\\
		संसारमोहनाशाय शाब्दबोधो न हि क्षमः~।\\[-6pt]
		न निवर्तेत तिमिरं कदाचिद्दीपवार्तया~॥\\ \hfill  (कु.त -९३,९७) 
\end{verse}

	तन्त्रशास्त्रेषु देवतोपासनादीक्षामन्त्रयन्त्रगुरुशिष्यादिविचाराः  उपलभ्यन्ते~। षट्‌चक्र-वाक्‌चतुष्टयादिविचारेषु तन्त्राणां योगशास्त्रव्याकरणस्मृत्यादिना संवादो दृश्यते~। 

	ईश्वरोपदिष्टतया प्रसिद्धे कुलार्णवतन्त्रे गुरोः शिष्यस्य च लक्षणानि दीक्षोपदेशः उपासना मन्त्रसिद्ध्यादयश्च विषयाः विस्तरेण प्रतिपाद्यन्ते~। तान् विचारान् तत्सम्बद्धान् अन्यत्र स्मृत्यादिषु उपलभ्यमानान्  विचारानपि सङ्गृह्य प्रकृतग्रन्थोऽयं सज्जीकृतः~। गुरुपादुकास्तोत्रं गुर्वष्टकं गुरुगीता प्रश्नोत्तरमालिकासङ्ग्रहः अपि अध्येतॄणां सौकर्याय  हिताय च अत्र ग्रन्थे औचित्यपूर्णत्वात् संयोजिताः विद्यन्ते~। तेषु गुर्वष्टकप्रश्नोत्तरमालिके  श्रीशङ्करभगवत्पादप्रणीते~। गुरुपादुकास्तोत्रं तावत् शृङ्गेरीश्रीशारदापीठस्य त्रयस्त्रिंशत्तमा\-धीश्वरैः श्रीसच्चिदानन्दशिवाभिनवनृसिंहभारतीस्वामिभिः विरचितम्~। गुरुगीता तु स्कान्दपुराणान्तर्गते उत्तरकाण्डे  उमामहेश्वरसंवादे वर्तत इति श्रूयते~।

	‘संस्कृतसंवर्धनप्रतिष्ठानं’ ग्रन्थस्यास्य मुद्रणभारं वहति~। प्रतिष्ठानायास्मै सादरं कार्तज्ञ्यं विनिवेद्यते~। ग्रन्थस्यास्य सज्जीकरणे विहितोद्यमाः आयुष्मान् राघवेन्द्र. पि. आरोल्लि, कुमारी दीपा हेगडे, अन्ये च छात्राः साधुवादेन अभिनन्द्यन्ते~। अल्पीयसा कालेन सामर्थ्येन च निर्वर्तितेषु टिप्पणीरचनासम्पादनादिकर्मसु सम्भाव्यमानान् अशेषान् दोषान्  सज्जनाः विबुधाः क्षमन्त इति दृढो मे विस्रम्भः~। सत्त्वपूर्णत्वेऽपि सम्प्रति कुण्ठितप्रचारेभ्यः प्रवर्तनपरशास्त्रग्रन्थेभ्यः सारः सङ्गृह्य जिज्ञासूनां संस्कृतलोकानाम् उपयोगाय आविष्क्रियते~। जगन्मातुः प्रेरणया सद्गुरुकृपया समारब्धस्यास्य महोद्यमस्य परिणामत्वेन अद्य विकसितमेतत् प्रथमं प्रज्ञानप्रसूनं श्रीसद्गुरुपादुकयोः समर्प्य विरमति~।
%\bigskip 

{\b 
\begin{center}
दुर्मुख-मार्गशीर्ष-कृष्ण-तृतीया \hfill 	वि. नवीन होळ्ळः
\end{center}
\vskip -0.5cm

शुक्रवासरः \hfil  शृङ्गेरी
\vskip -1mm
१६.१२.२०१६ 
}

