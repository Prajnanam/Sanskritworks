%\chapter{शिष्यशिक्षा}

\section{शिष्यस्य नियमाः ।}

अग्नीन्धनं भैक्षचर्यामधःशय्यां गुरोर्हितम् ।\\[-2mm]
आसमावर्तनात्कुर्यात्कृतोपनयनो द्विजः ॥१०८॥\\
सेवेतेमांस्तु नियमान्ब्रह्मचारी गुरौ वसन् ।\\[-2mm]
सन्नियम्येन्द्रिग्रामं तपोवृद्ध्यर्थमात्मनः ॥१७५॥\\
नित्यं स्नात्वा शुचिः कुर्याद्देवर्षिपितृतर्पणम् ।\\[-2mm]
देवताभ्यर्चनं चैव समिदाधानमेव च ॥१७६॥\\
वर्जयेन्मधु मांसं च गन्धं माल्यं रसान्स्त्रियः।\\[-2mm]
शुक्तानि यानि सर्वाणि प्राणिनां चैव हिंसनम् ॥१७६॥\\
अभ्यङ्गमञ्जनं चाक्ष्णोरुपानच्छत्रधारणम् ।\\[-2mm]
कामं क्रोधं च लोभं च नर्तनं गीतवादनम्॥१७७॥\\
द्यूतञ्च जनवादं च परिवादं तथानृतम् ।\\[-2mm]
स्त्रीणां च प्रेक्षणालम्भमुपघातं परस्य च ॥१७९॥\\
एकः शयीत सर्वत्र न रेतः स्कन्दयेत्क्वचित् ।\\[-2mm]
कामाद्धि स्कन्दयन्रेतो हिनस्ति व्रतमात्मनः ॥१८०॥\\
उदकुंभं सुमनसो गोशकृन्मृत्तिकाकुशान् ।\\[-2mm]
आहरेद्यावदर्थानि भैक्षं चाहरहश्चरेत् ॥१८२॥\\
वेदयज्ञैरहीनानां प्रशस्तानां स्वकर्मसु ।\\[-2mm]
ब्रह्मचार्याहरेद्भैक्षं गृहेभ्यःप्रयतोऽन्वहम् ॥१८३॥\\
गुरोः कुले न भिक्षेत न ज्ञातिकुलबंधुषु ।\\[-2mm]
अलाभे त्वन्यगेहानां पूर्वं पूर्वं विवर्जयेत् ॥१८४॥\\
सर्वं वापि चरेद्ग्रामं पूर्वोक्तानामसम्भवे ।\\[-2mm]
नियम्य प्रयतो वाचमभिशस्तांस्तु वर्जयेत् ॥१८५॥\\
दूरादाह्त्य समिधः संनिदध्याद्विहायसि ।\\[-2mm]
सायंप्रातश्च जुहुयात्ताभिरग्निमतन्द्रितः ॥१८६॥\\
चोदितो गुरुणा नित्यमप्रचोदित एव वा ।\\[-2mm]
कुर्यादध्ययने यत्नमाचार्यस्य हितेषु च ॥१९१॥\\
ब्रह्मारम्भेऽवसाने च पादौ ग्राह्यौ गुरोः सदा ।\\[-2mm]
संहत्य हस्तावध्येयं स हि ब्रह्मांजलिः स्मृतः ॥७१॥\\
व्यत्यस्तपाणिना कार्यमुपसंग्रहणं गुरोः ।\\[-2mm]
 सव्येन सव्यः स्प्रष्टव्यो दक्षिणेन च दक्षिणः ॥७२॥\\
अभिवादनशीलस्य नित्यं वृद्धोपसेविनः ।\\[-2mm]
चत्वारि तस्य वर्धन्ते आयुर्विद्या यशो बलम् ॥१२१॥\\
लौकिकं वैदिकं वापि तथाध्यात्मिकमेव च ।\\[-2mm]
आददीत यतो ज्ञानं तं पूर्वमभिवादयेत् ॥११७॥\\
अध्येष्यमाणस्त्वाचान्तो यथाशास्त्रमुदङ्मुखः ।\\[-2mm]
ब्रह्मांजलिकृतोऽध्याप्यो लघुवासा जितेन्द्रियः ॥७०॥\\
ब्रह्मणः प्रणवं कुर्यादादावन्ते च सर्वदा ।\\[-2mm]
स्रवत्यनोङ्कृतं पूर्वं पुरस्ताच्च विशीर्यति ॥७४॥\\
पादप्रक्षालनं स्नानमभ्यङ्गं दन्तधावनम् ।\\[-2mm]
मूत्रं निष्ठीवनं क्षौरं शयनं स्त्रीनिषेवनम् ॥८५॥\\
वीरासनं सुदुर्वाक्यं शासनं हास्यरोदनम् ।\\[-2mm]
केशमोचनमुष्णीषं कञ्चुकं नग्नतां तथा ॥८६॥\\
पादप्रसारणं वादं कलहं दूषणं प्रिये ।\\[-2mm]
अङ्गभङ्गाङ्गवाद्यादिकरास्फालनधूननम् ॥८७॥\\
द्यूतकौतुकमल्लादियुद्धतृत्यादि चाम्बिके ।\\[-2mm]
गुरुयोगिमहासिद्धिपीठक्षेत्राश्रमेषु च ।\\[-2mm]
नाचरेदाचरन्मोहाद्देवताशापमाप्नुयात् ॥८८॥\\
स्थानान्तरगताचार्ये व्यसने विषमे स्तिथे ।\\[-2mm]
श्रीगुरुं न त्यजेत् क्वापि तदादिष्टो व्रजेत् प्रिये ॥१००॥\\
पश्चात्पदेन निर्गच्छेन्नमस्कृत्य गुरोर्गृहात् ।\\[-2mm]
एकासने नोपविशेद् गुरुणा तत्समैः सह ॥१०६॥\\
न विशेदासने देवि देवतागुरुसन्निधौ ।\\[-2mm]
गुरोः सिंहासनं देयं ज्येष्ठानामुत्तमासनम् ।\\[-2mm]
देश्यासनं कनिष्ठानामितरेशां समासनम् ॥१०७॥\\
रिक्तहस्तश्च नोपेयाद्राजानं देवतां गुरुम् ।\\[-2mm]
फलपुष्पाम्बरादीनि यथाशक्त्या समर्पयेत् ॥१२०॥\\ 
नीचं शय्यासनं चास्य सर्वदा गुरुसन्निधौ ।\\[-2mm]
गुरोस्तु चक्षुर्विषये न यथेष्टासनो भवेत् ॥१९८॥\\
नोदाहरेदस्य नाम परोक्षमपि केवलम् ।\\[-2mm]
न चैवास्यानुकुर्वीत गतिभाषितचेष्टितम् ॥१९९॥\\
शय्यासनेऽध्याचरिते श्रेयसा न समाविशेत् ।\\[-2mm]
शय्यासनस्थश्चैवैनं प्रत्युत्थायाभिवादयेत् ॥११९॥\\
गुरोर्यत्र परीवादो निन्दा वापि प्रवर्तते ।\\[-2mm]
कर्णौ तत्र पिधातव्यौ गन्तव्यं वा ततोऽन्यतः ॥२००॥\\
परीवादात्खरो भवति श्वा वै भवति निन्दकः ।\\[-2mm]
परिभोक्ता कृमिर्भवति कीटो भवति मत्सरी ॥२०१॥\\
दूरस्थो नार्चयेदेनं न क्रुद्धो नान्तिके स्त्रियाः ।\\[-2mm]
यानासनस्थश्चैवैनमवरुह्याभिवादयेत् ॥२०२॥\\
गुरोर्गुरौ सन्निहिते गुरुवद्वृत्तिमाचरेत् ।\\[-2mm]
न चानिसृष्टो गुरुणा स्वान्गुरूनभिवादयेत् ॥२०५॥\\

\section{गुरुशुश्रूषणविधिः ।}

शरीरं चैव वाचं च बुद्धीन्द्रियमनांसि च ।\\[-2mm]
नियम्य प्राञ्जलिस्तिष्ठेद्वीक्षमाणो गुरोर्मुखम् ॥१९२॥\\
नित्यमुद्धृतपाणिः स्यात्साध्वाचारः सुसंयुतः ।\\[-2mm]
आस्यतामिति चोक्तः सन्नासीताभिमुखं गुरोः ॥१९३॥\\
हीनान्नवस्त्रवेषः स्यात्सर्वदा गुरुसन्निधौ ।\\[-2mm]
उत्तिष्ठेत्प्रथमं चास्य चरमं चैव संविशेत् ॥१९४॥\\
प्रतिश्रवणसंभाषे शयानो न समाचरेत् ।\\[-2mm]
नासीनो न च भुञ्जानो न तिष्ठन्न पराङ्मुखः ॥१९५॥\\
आसीनस्य स्थितः कुर्यादभिगच्छंस्तु तिष्ठतः ।\\[-2mm]
प्रत्युद्गम्य त्वाव्रजतः पश्चाद्धावंस्तु धावतः ॥१९६॥\\
पराङ्मुखस्याभिमुखो दूरस्थस्यैत्य चान्तिकम् ।\\[-2mm]
प्रणम्य तु शयानस्य निदेशे चैव तिष्ठतः ॥१९७॥\\
प्रतिवातेऽनुवाते च नासीत गुरुणा सह ।\\[-2mm]
असंश्रवे चैव गुरोर्न किञ्चिदपि कीर्तयेत् ॥२०३॥\\
भाषणं पाठनं गानं भोजनं शयनादिकम् ।\\[-2mm]
अनादिष्टो न कुर्वीत न चावन्दनपूर्वकम् ॥\\
यथा खनन्खनित्रेण नरो वार्यधिगच्छति ।\\[-2mm]
तथा गुरुगतां विद्यां शुश्रूषुरधिगच्छति॥२१८॥\\
गुरूक्तं परुषं वाक्यमाशिषं परिचिन्तयेत् ।\\[-2mm]
तेन सन्ताडितो वापि प्रसादमिति संस्मरेत् ॥५४॥\\
गुरुकार्ये स्वयं शक्तो नापरं प्रेषयेत् प्रिये ।\\[-2mm]
बहुभृत्यपरैर्भृत्यैः सहितोऽप्यतिभक्तिमान् ॥९२॥\\
अभिमानो न कर्तव्यो जातिविद्याधनादिभिः ।\\[-2mm]
सर्वदा सेवयेत् नित्यं शिष्यः श्रीगुरुसन्निधौ ॥९४॥\\
सामान्यतो निषिञ्च तद्गुरोर्यदि सन्निधौ ।\\[-2mm]
आचरेत्तस्य सर्वस्य दोषः कोटिगुणो भवेत् ॥९७॥\\
गोब्राह्मणवधं कृत्वा यत् पापं समवाप्नुयात् ।\\[-2mm]
तत् पापं समवाप्नोति गुर्वग्रेऽनृतभाषणात् ॥९९॥\\
शक्तिच्छायां सुरच्छायां गुरुच्छायां न लङ्घयेत् ।\\[-2mm]
न तेषु कुर्यात् स्वच्छायां न स्वपेद्गुरु सन्निधौ ॥१०२॥\\
आसमाप्तेः शरीरस्य यस्तु शुश्रूषते गुरुम्।\\[-2mm]
स गच्छत्यञ्जसा विप्रो ब्रह्मणः सद्म शाश्वतम् ॥२४४॥\\
कर्मणा मनसा वाचा नित्यमाराधयेद्गुरुम् ।\\[-2mm]
दीर्घदण्डं नमस्कुर्यान्निर्लज्जो गुरुसन्निधौ ॥\\
शरीरमर्थं प्राणांश्च सद्गुरुभ्यो निवेदयेत् ।\\[-2mm]
आत्मानमपि दास्याय वैदेहो जनको यथा ॥\\
अप्रियस्य हास्यस्य नावकाशो  गुरोः परः ।\\[-2mm]
न नियोग परं ब्रूयात् गुरोराज्ञां विभावयेत् ॥\\
मनसा कर्मणा वाचा गुरोः क्रोधं न कारयेत् ॥\\
तस्य क्रोधेन दह्यन्ते ह्यायुः श्रीर्ज्ञानसत्क्रियाः ।\\[-2mm]
श्रेयोर्थी चेन्नरो धीमान्न मिथ्याचारमाचरेत् ॥\\
गुरोर्हितं प्रियं कुर्याद् आदिष्टो वा न वा सदा ॥\\
सर्वदेवात्मकश्चासौ सर्वमन्त्रमयो गुरुः ।\\[-2mm]
तस्मात्सर्वप्रयत्नेन तस्याज्ञां शिरसा वहेत् ॥\\
गुरावतुष्टेऽतुष्टाः स्युस्सर्वे देवा द्विजोत्तमाः ।\\[-2mm]
तुष्टे तुष्टा यतस्तस्मात्सर्वदेवमयो गुरुः ॥\\

\section{शिष्यलक्षणम् ।}

विप्रः षड्गुणयुक्तश्चेदभक्तो न प्रशस्यते ।\\[-2mm]
म्लेच्छोऽपि गुणहीनोऽपि भक्तिमान् शिष्य उच्यते ॥२८॥\\
शरीरमर्थं प्राणांश्च सद्गुरुभ्यो निवेद्य यः ।\\[-2mm]
गुरुभ्यः शिक्षयते योगं शिष्य इत्यभिधीयते ॥\\
गुरुभक्तिविहीनस्य तपो विद्या कुलं व्रतम् ।\\[-2mm]
सर्वं नश्यति तत्रैव भूषणं लोकरञ्जनम् ॥२९॥\\
गुरुभक्त्यग्निना सम्यग्दग्धदुर्मतिकल्मषः ।\\[-2mm]
श्वपचोऽपि परैः पूज्यो विद्वानपि न नास्तिकः ॥३०॥\\
इमं लोकं मातृभक्त्या पितृभक्त्या तु मध्यमम् ।\\[-2mm]
गुरुशुश्रूषया त्वेवं ब्रह्मलोकं समश्नुते ॥२३३॥\\
अहिंसका दयावन्तो नित्यमुद्युक्तचेतसः ।\\[-2mm]
अमानिनो बुद्धिमन्तस्त्यक्तस्पर्धाः प्रियंवदाः ॥\\
ऋजवो मृदवः स्वच्छा विनीताः स्थिरचेतसः ।\\[-2mm]
शौचाचारसमायुक्ता गुरुभक्ता द्विजातयः ॥\\
एवं कृतसमीपेता वाङ्मनः कार्यकर्मभिः ।\\[-2mm]
शोध्या बोध्या यथान्यायम् इति शास्त्रेषुनिर्णयः ॥\\

\section{गुरुभक्तिः ।}

धर्मार्थकामैः किन्तस्य मोक्ष एव करे स्थितः ।\\[-2mm]
सर्वार्थैः श्रीगुरो देवि यस्य भक्तिः सदा स्थिरा ॥३१॥\\
यस्य देवे परा भक्तिर्यथा  देवे तथा गुरौ ।\\[-2mm]
तस्य ते कथिता ह्यर्थाः प्रकाशन्ते महात्मनः ॥३३॥\\
गुरुभक्त्या यथा देवि प्राप्यन्ते सर्वसिद्धयः ।\\[-2mm]
यज्ञदानतपस्तीर्थव्रताद्यैर्न तथा प्रिये ॥३६॥\\
भोगमोक्षार्थिनां ब्रह्मविष्णवीशपदकाङ्क्षिणाम्।\\[-2mm]
भक्तिरेव गुरौ देवि नान्यः पन्था इति श्रुतिः ॥४०॥\\
गुरुं न मर्त्यं बुध्येत यदि बुध्येत तस्य हि ।\\[-2mm]
न कदाचिद्भवेत् सिद्धिर्मन्त्रैर्वा देवतार्चनैः ॥४६॥\\
तावद् भ्रमन्ति संसारे सर्वदुःखमलीमसाः ।\\[-2mm]
न भवेत् सद्गुरौ भक्तिर्यावद्देवेशि देहिनाम् ॥१६॥\\
तावदाराधयेच्छिष्यः प्रसन्नोऽसौ यदा भवेत् ।\\[-2mm]
गुरौ प्रसन्ने शिष्यस्य सद्यः पापक्षयो भवेत् ॥२०॥\\
क्षीयन्ते सर्वपापानि वर्धन्ते पुण्यराशयः ।\\[-2mm]
सिध्यन्ति सर्वकार्याणि गुरुशुश्रूषया प्रिये ॥६७॥\\
भक्त्या वित्तानुसारेण गुरुमुद्दिश्य यत्कृतम् ।\\[-2mm]
अल्पे महति वा तुल्यं पुण्यमाढ्यदरिद्रयोः ॥६९॥\\
सर्वस्वमपि यो दद्याद् गुरौ भक्तिविवर्जितः ।\\[-2mm]
शिष्यो न फलमाप्नोति भक्तिरेव हि कारणम् ॥७०॥\\
भक्त्या तुष्टेन गुरुणा यः प्रदिष्टः कृपालुना ।\\[-2mm]
कर्ममुक्तो भवेच्छिष्यो भुक्तिमुक्त्योः स भाजनम् ॥२३॥\\
शिष्येणापि तथा कार्यं यथा सन्तोषितो गुरुः ।\\[-2mm]
प्रियं कुर्याच्च देवेशि मनोवाक्कायकर्मभिः ॥२४॥\\
मुनिभ्यः पन्नगेभ्यश्च सुरेभ्यः शापतोऽपि च ।\\[-2mm]
कालमृत्युभयाद्वापि गुरू रक्षति पार्वति ॥\\
धन्या माता पिता धन्यो धन्या वंशा जना अपि ।\\[-2mm]
धन्या च वसुधा यत्र गुरुभक्तः प्रजायते ॥\\

\section{शिष्यपरीक्षा ।}

ज्ञानेन क्रियया वापि गुरुः शिष्यं परीक्षयेत् ।\\[-2mm]
संवत्सरं तदर्धं वा तदर्धं वा प्रयत्नतः ॥१९॥\\
उत्तमांश्चाधमे कुर्यान्नीचानुत्तमकर्मणि ।\\[-2mm]
प्राणद्रव्यप्रणामाद्यैरादेशैश्च स्वयं समैः ॥२०॥\\
आकृष्टस्ताडितो वापि यो विषादं न याति च ।\\[-2mm]
गुरुः कृपां करोतीति मुदा सञ्चिन्तयेत् सदा ॥२२॥\\
श्रीगुरोः स्मरणे चापि कीर्तने दर्शनेऽपि च ।\\[-2mm]
वन्दने परिचर्यायामाह्वाने प्रेषणे प्रिये ॥२३॥\\
आनन्दकम्परोमाञ्चस्वरनेत्रादिविक्रियाः ।\\[-2mm]
येषां स्युस्तेऽत्र योग्याश्च दीक्षासंस्कारकर्मणि ॥२४॥\\
आदिमध्यावसानेषु योग्याः शक्तिनिपातिताः।\\[-2mm]
अधमा मध्यमाः श्रेष्ठाः शिष्या देवि प्रकीर्तिताः ॥२७॥\\
आदौ भक्तिर्भवेद्देवि दीक्षार्थं समुदन्ति ये ।\\[-2mm]
पुनर्विपुलहृष्टास्ते आदियोग्या इतीरिताः ॥२८॥\\
दीक्षासमयसम्प्राप्ता ज्ञानविज्ञानवर्जिताः ।\\[-2mm]
भक्त्या प्रध्वस्तपर्याया मध्ययोग्याश्च ते स्मृताः ॥२९॥\\
आदौ भक्तिविहीना ये मध्यभक्तास्तु ये नराः ।\\[-2mm]
अन्तप्रवृद्धभक्ताश्च अन्तयोग्या भवन्ति ते ।\\[-2mm]
उत्तमज्ञानसंज्ञांश्चेत्युपदेशस्त्रिधा प्रिये ॥३०॥\\

\section{गुरोर्नियमाः ।}

उपनीय गुरुः शिष्यं शिक्षयेच्छौचमादितः ।\\[-2mm]
आचारमग्निकार्यं च संध्योपासनमेव च ॥६९॥\\
अध्येष्यमाणं तु गुरुर्नित्यकालमतन्द्रितः ।\\[-2mm]
अधीष्व भो इति ब्रूयाद्विरामोऽस्त्विति चारभेत् ॥७३॥\\
आचार्यपुत्रः शुश्रूषुर्ज्ञानदो धार्मिकः शुचिः ।\\[-2mm]
आप्तः शक्तोर्थदः साधुः स्वोऽध्याप्या दश धर्मतः ॥१०९॥\\
नापृष्टः कस्यचिद् ब्रूयान्न चान्यायेन पृच्छतः ।\\[-2mm]
जानन्नपि हि  मेधावी जडवल्लोक ते न आचरेत् ॥११०॥\\
अधर्मेण च यः प्राह यश्चाधर्मेण पृच्छति ।\\[-2mm]
तयोरन्यतरः प्रैति विद्वेषं वाधिगच्छति ॥१११॥\\
सच्छिष्यायातिभक्ताय यज्ज्ञानमुपदिश्यते ।\\[-2mm]
तज्ज्ञानं तत्तु शास्त्रार्थं तद्विदध्यादखण्डितम्॥१६॥\\
असच्छिष्येष्वभक्तेषु यज्ज्ञानमुपदिश्यते ।\\[-2mm]
तत् प्रयात्यपवित्रत्वं गोक्षीरं श्वघृतादिव ॥१७॥\\
धर्मार्थो यत्र न स्यातां शुश्रूषा वापि तद्विधा ।\\[-2mm]
तत्र विद्या न वक्तव्या शुभं बीजमिवोषरे ॥\\
विद्ययैव समं कामं भर्तव्यं ब्रह्मवादिना।\\[-2mm]
आपद्यपि हि घोरायां न त्वेनामिरिणो वपेत् ॥\\
उपनीय तु यः शिष्यं वेदमध्यापयेद्द्विजः ।\\[-2mm]
सकल्पं सरहस्यं च तमाचार्यं प्रचक्षते ॥१४०॥\\
स्वयमाचरते शिष्यान् आचारे स्थापयत्यपि ।\\[-2mm]
आचिनोतीह शास्त्रार्थान् आचार्यस्तेन कथ्यते
एकदेशं तु वेदस्य वेदाङ्गान्यपि वा पुनः ।\\[-2mm]
योऽध्यापयति वृत्त्यर्थमपाध्यायः स उच्यते ॥१४१॥\\
निषेकादीनि कर्माणि यः करोति यथाविधि ।\\[-2mm]
संभावयति चान्नेन स विप्रो गुरुरुच्य्ते ॥१४२॥\\
अल्पं वा बहु वा यस्य श्रुतस्योपकरोति यः ।\\[-2mm]
तमपीह गुरुं विद्याच्छ्रुतोपक्रियया तया ॥१४९॥\\
धनेच्छाभयलोभाद्यैरयोग्यं यदि दीक्षयेत् ।\\[-2mm]
देवताशापमाप्नोति कृतञ्च निष्फलं भवेत् ॥१८॥\\

\section{गुरुपरीक्षा ।}

स्वयं वेद्ये परे तत्त्वे स्वात्मानं वेत्ति निश्चलः ।\\[-2mm]
आत्मनोऽनुग्रहो नास्ति परस्यानुग्रहः कथम् ॥११९॥\\
ब्रह्माकारं मनोरूपं प्रत्यक्षं स्वतनुस्थितम् ।\\[-2mm]
यो न जानाति चान्यस्य कथं मोक्षं ददात्यसौ ॥१२०॥\\
विद्धस्तु वेधयेद्देवि नाविद्धो वेधको भवेत् ।\\[-2mm]
मुक्तस्तु मोचयेद् बद्धं न मुक्तो मोचकः कथम् ॥१२४॥\\
अभिज्ञश्चोद्धरेन्मूर्खं न मूर्खो मूर्खमुद्धरेत् ।\\[-2mm]
शिलां सन्तारयेन्नौर्हि किं शिला तारयेच्छिलाम् ॥१२५॥\\
तत्त्वहीनं गुरुं लब्ध्वा केवलं भवतत्परः ।\\[-2mm]
इहामुत्र फलं किञ्चित् स नरो नाप्नुयात् प्रिये ॥१२६॥\\
श्रीगुरुं लक्षणोपेतं संशयच्छेदकारकम् ।\\[-2mm]
लब्ध्वा ज्ञानप्रदं देवि न गुर्वन्तरमाश्रयेत् ॥१३०॥\\
अनभिज्ञं गुरुं प्राप्य सदा संशयकारकम् ।\\[-2mm]
गुर्वन्तरन्तु गत्वा स नैतद्दोषेण लिप्यते ॥१३१॥\\
मधुलुब्धो यथा भृङ्गः पुष्पात् पुष्पान्तरं व्रजेत् ।\\[-2mm]
ज्ञानलुब्धस्तथा शिष्यः गुरोगुर्वन्तरं व्रजेत् ॥१३२॥\\
पञ्चैते कार्यभूताः स्युः कारणं बोधको भवेत् ।\\[-2mm]
पूर्णाभिषेककर्ता यो गुरुस्तस्यैव पादुका ।\\[-2mm]
पूजनीया महेशानि बहुत्वेऽपि न संशयः॥१२९॥\\
शिष्योऽपि लक्षणैरेतैः कुर्याद् गुरुपरीक्षणम् ।\\[-2mm]
आनन्दाद्यैर्जपस्तोत्रध्यानहोमार्चनादिषु ॥२५॥\\
ज्ञानोपदेशसामर्थ्यं मन्त्रसिद्धिमपीश्वरि ।\\[-2mm]
वेधकत्वं परिज्ञाय शिष्यो भूयान्न चान्यथा ॥२६॥\\
ज्ञानहीनो गुरुंमन्यो मिथ्यावादी विडम्बकः ।\\[-2mm]
स्वविश्रान्तिं न जानाति परशान्तिं करोति किम् ॥\\
स्वयं तरितुमक्षमः परान्निस्तारयेत्कथम् ।\\[-2mm]
दूरे तं वर्जयेत् प्राज्ञो धीरमेव समाश्रयेत् ॥	

\section{गुरुलक्षणानि ।}

गुरवो बहवः सन्ति वेदशास्त्रादिपारगाः ।\\[-2mm]
दुर्लभोऽयं गुरुर्देवि परतत्त्वार्थपारगः ॥१०५॥\\
गुरवो बहवः सन्ति आत्मनोऽन्यप्रदा भुवि ।\\[-2mm]
दुर्लभोऽयं गुरुर्देवि लोकेष्वात्मप्रकाशकः ॥१०६॥\\
गुरवो बहवः सन्ति कुमन्त्रौषधिवेदिनः ।\\[-2mm]
निगमागमशास्त्रोक्तमन्त्रज्ञो  दुर्लभो भुवि ॥१०७॥\\
गुरवो बहवः सन्ति शिष्यवित्तापहारकाः ।\\[-2mm]
दुर्लभोऽयं गुरुर्देवि शिष्यदुःखापहारकः ॥१०८॥\\
वर्णाश्रमकुलाचारनिरता बहवो भुवि ।\\[-2mm]
सर्वसङ्कल्पहीनो यः स गुरुर्देवि दुर्लभः ॥१०९॥\\

पाशबद्धः पशुर्ज्ञेयः पाशमुक्तो महेश्वरः ।\\[-2mm]
तस्मात् पाशहरो यस्तु स गुरुः परमो मतः ॥११॥\\ 
मूलादिब्रह्मरन्ध्रान्तसप्ताम्भोजदलेषु च ।\\[-2mm]
जीवाचारफलं वेत्ति स गुरुर्नापरः प्रिये ॥१४॥\\
शिवादिगुरुपर्यन्तं पारम्पर्यक्रमेण यः ।\\[-2mm]
अवाप्ततत्त्वसम्भारः स गुरुर्नापरः प्रिये ॥१५॥\\
येन वा दर्शिते तत्त्वे तत्क्षणात्तन्मयो भवेत् ।\\[-2mm]
मन्यते मुक्तमात्मानं स गुरुर्नापरः प्रिये ॥१६॥\\
ये दत्वा सहजानन्दं दरन्तीन्द्रियजं सुखम् ।\\[-2mm]
सेव्यास्ते गुरवः शिष्यैरन्ये त्याज्याः प्रतारकाः ॥१७॥\\
संसारभयभीतस्य शिष्यस्य गुरुरादरात् ।\\[-2mm]
व्रतोपवासनियमैर्नियन्ता स गुरुर्मतः ॥१८॥\\
यः प्रसन्नः क्षणार्धेन मोक्षलक्ष्मीं प्रयच्छति ।\\[-2mm]
दुर्लभं तं विजानीयाद् गुरुं संसारतारकम् ॥१९॥\\
यः क्षणेनात्मसामर्थ्यं स्वशिष्याय ददाति हि ।\\[-2mm]
क्रियायासादिरहितं स गुरुर्देवदुर्लभः ॥१००॥\\
यः सद्यः प्रत्ययकरं सुलभञ्चात्मसौख्यदम् ।\\[-2mm]
ज्ञानोपदेशं कुरुते स गुरुर्देवदुर्लभः ॥१०१॥\\
द्वीपाद्द्वीपान्तरं देवि सञ्चरेद्यथा तथा ।\\[-2mm]
यो दद्यात्स गुरुर्ज्ञानमभ्यासादिविवर्जितम् ॥१०२॥\\
क्षुधितस्य यथा तृप्तिराहारादाशु जायते ।\\[-2mm]
तथोपदेशमात्रेण ज्ञानदो दुर्लभो गुरुः ॥१०३॥\\
गुरवो बहवः सन्ति दीपवच्च गृहे गृहे ।\\[-2mm]
दुर्लभोऽयं गुरुर्देवि सूर्यवत् सर्वदीपकः ॥१०४॥\\
गुरोर्यस्यैव सम्पर्कात् परानन्दोभिजायते ।\\[-2mm]
गुरुं तमेव वृणुयान्नापरं मतिमान्नरः ॥११०॥\\
यस्यानुभवपर्यन्तं बुद्धिस्तत्र प्रवर्तते ।\\[-2mm]
यस्यालोकनमात्रेण मुच्यते नात्र संशयः ॥१११॥\\
शङ्कया भक्षितं सर्वं त्रैलोक्यं सचराचरम् ।\\[-2mm]
सा शङ्का भक्षिता येन स गुरुर्देवि दुर्लभः ॥११२॥\\
दृश्यं विना स्थिरा दृष्टिर्मनश्चालम्बनं विना ।\\[-2mm]
विनायासं स्थिरो वायुर्यस्य स्यात् स गुरुः प्रिये ॥७०॥\\
जाग्रत्स्वप्नसुषुप्तिश्च तुरीयं तदतीतकम् ।\\[-2mm]
यो वेत्ति पञ्चकं देवि स गुरुः कथितः प्रिये ॥७५॥\\
यो वा पराञ्च पश्यन्तीं मध्यमां वैखरीमपि ।\\[-2mm]
चतुष्टयं विजानाति स गुरुः कथितः प्रिये ॥७७॥\\
महामुद्रां नभोमुद्राम् उड्डीयानां जलन्धरम् ।\\[-2mm]
मूलबन्धञ्च यो वेत्ति स गुरुः परमो मतः ॥८५॥\\
पिण्डब्रह्माण्डयोरैक्यं स्थितिं यो वेत्ति तत्त्वतः ।\\[-2mm]
शिरास्थिरोमसंख्यादि स गुरुर्नापरः प्रिये ॥८८॥\\
पद्मादिचतुरशीतिनानासनविचक्षणः ।\\[-2mm]
यमाद्यष्टाङ्गयोगज्ञः स गुरुः परमो मतः ॥८९॥\\
प्रेरकः सूचकश्चैव वाचको दर्शकस्तथा ।\\[-2mm]
शिक्षको बोधकश्चैव षडेते गुरवः स्मृताः ॥१२८॥\\
 रूढाविद्या जगन्माया देहेस्ति ध्वान्तरूपिणी ।\\[-2mm]
तद्वारकः प्रकाशश्च गुरुशब्देन कथ्यते ॥\\
गुकारश्चान्धकारो हि रुकारस्तेज उच्यते ।\\[-2mm]
अज्ञानग्रासकं ब्रह्म गुरुरेव न संशयः ॥\\
गुकारश्चान्धकारस्तु  रुकारस्तन्निरोधकः ।\\[-2mm]
अन्धकारविनाशित्वात् गुरुरित्यभिधीयते ॥\\
गुकारः स्याद्गुणातीतो रूपातीतो रुकारकः ।\\[-2mm]
गुणरूपविहीनत्वाद्गुरुरित्यभिधीयते ॥\\
गुकारः प्रथमो वर्णो मायादिगुणभासकः ।\\[-2mm]
रुकारोऽस्ति परं ब्रह्म मायाभ्रान्तिविमोचकम् ॥\\
यस्यानुग्रहमात्रेण हृदि ह्युत्पद्यते क्षणात् ।\\[-2mm]
ज्ञानं च परमानन्दः सद्गुरुः शिव एव सः ॥\\
नित्यं ब्रह्म निराकारं येन प्राप्तं स वै गुरुः ।\\[-2mm]
स शिष्यं प्रापयेत् प्राप्यं दीपो दीपान्तरं यथा ॥\\
 आचिनोति च शास्त्रार्थमाचारे स्थापयत्यपि ।\\[-2mm]
स्वयमाचरते यस्मादाचार्यस्तेन चोच्यते ॥\\

\section{चतुर्विंशतिगुरवः}

पृथिवी वायुराकाशमापोऽग्निश्चन्द्रमा रविः ।\\[-2mm]
कपोतोऽजगरः सिन्धुः पतङ्गो मधुकृद् गजः ॥\\
मधुपो हरिणो मीनः पिङ्गला कुररोऽर्भकः ।\\[-2mm]
कौमारी शरकृत्सर्प ऊर्णनाभिःसुपेशकृत्
एते हि गुरवो विप्राश्चतुर्विंशतिरीरिता ॥\\

\subsection{पृथिवीगुरुलक्षणम् ।}

भूतैराक्रम्यमाणोऽपि धीरो दैववशानुगैः ।\\[-2mm]
तद्विद्वान्न चलेन्मार्गाह् अनुशिक्षान् क्षितेर्व्रतम् ॥\\

\subsection{वायुगुरुलक्षणम् ।}

विषयेष्वाविशन्षन्योगी नाना धर्मेषु सर्वतः ।\\[-2mm]
गुणदोषव्यपेतात्मा न विषज्येत वायुवत् ॥\\

\subsection{आकाशगुरुलक्षणम् ।}

तेजोवान्भूमयैर्भावैमेधर्द्यैर्वायुनेरितैः ।\\[-2mm]
न स्पृश्यते नभस्तद्वत्कालसृष्टैर्गुणैर्भवेत् ॥\\

\subsection{अब् गुरुलक्षणम् ।}

स्वच्छः प्रकृतितः स्निग्धो मधुरस्तीर्थभूर्नणाम् ।\\[-2mm]
मुनिः पुनात्याप इव दीक्षोपस्पर्शकीर्तनैः ॥\\

\subsection{अग्निगुरुलक्षणम्}

तेजस्वी तपसा दीप्तो धुर्धर्षश्चापरैरपि ।\\[-2mm]
सर्वभक्षोपि युक्तात्मा नादत्ते मलमग्निवत् ॥\\

\subsection{चन्द्रगुरुलक्षणम् ।}

निषेकाद्याः श्मशानान्ताः भावा देहस्य नात्मनः।\\[-2mm]
कलानामिव चन्द्रस्य कालेनाव्यक्तवर्त्मना ॥\\

\subsection{रविगुरुलक्षणम् ।}

गुणैर्गुणानुपादत्ते यथाकाले विमुञ्चति ।\\[-2mm]
न तेषु युज्यते योगी गोभिर्गा इव गोपतिः ॥\\

\subsection{कपोतगुरुलक्षणम् ।}

नातिस्नेहःप्रसङ्गो वा कर्तव्यः क्वापि केनचित् ।\\[-2mm]
कुर्वन्विन्देत  सन्ताप कपोत इव दीनधीः ॥\\

\subsection{अजगरगुरुलक्षणम् ।}

ग्रासं कुमृष्टं विरसं महीयः स्वल्पमेव वा ।\\[-2mm]
यदृच्छयैवापतितं ग्रसेदजगरक्रियः ॥\\
शयिताहानि भूरीणि निराहारोऽनुपक्रमः ।\\[-2mm]
यदि नोपनमेद्ग्रासो महाहिरिव दिष्टभुक् ॥\\

\subsection{सिन्धुगुरुलक्षणम् ।}

अनन्तपारो ह्यक्षोभ्यो दुर्विगाह्यो दुरत्ययः ।\\[-2mm]
समृद्धकामो हीनो  वा नारायणपरो मुनिः ।\\[-2mm]
नोत्सर्पेन्न शुष्येत सरिद्भिरिव सागरः ॥\\

\subsection{पतङ्गगुरुलक्षणम्}

योषिद्धिरण्याभरणाम्बरादि 
द्रव्येषु मायारचितेषु मूढः ।\\[-2mm]
प्रलोभितात्माभ्युपभोगबुध्या
पतङ्गवन्नश्यति नष्ट दृष्टिः ॥\\

\subsection{मधुकृदगुरुलक्षणम्}

स्तोकं स्तोकं ग्रसेद्भासं देहो वर्तेत यावता ।\\[-2mm]
गृहानहिंसन्नातिष्ठेद्वृत्तिं माधुकरीं मुनिः ॥\\
अणुद्भ्यश्च महद्भ्यश्च शास्त्रेभ्यः कुशलो नरः ।\\[-2mm]
सर्वतस्सारमादद्यात्पुष्पेभ्य इव षट् पदः ॥\\

\subsection{गजगुरुलक्षणम् ।}

पदापि युवतीं भिक्षुर्न स्पृशोद्दारवीमपि ।\\[-2mm]
स्पृशन् करीव बद्ध्येत करिण्या अङ्गसद्गतः ॥\\

\subsection{मधुपगुरुलक्षणम् ।}

न देयं नोपभोग्यञ्च लुब्धै र्यंद्दुःखसञ्चितम् ।\\[-2mm]
भुङ्क्ते तदपि तत्रान्यो मधुपेवार्थविन्मधु ॥\\

\subsection{हरिगुरुलक्षणम्}

ग्राम्यगीतं न शृणुयात् यतिर्वनचरः क्वचित् ।\\[-2mm]
शिक्षेत हरिणाद्बुद्धिं मृगयागीतमोहितात् ॥\\

\subsection{मीनगुरुलक्षणम् ।}

जिह्वयातिप्रमाथिन्या जनो रसविमोहितः ।\\[-2mm]
मृत्युमृच्छत्यसद्बुद्धिर्मीनस्तु बडिशैर्यथा ॥\\

\subsection{पिङ्गलगुरुलक्षणम्}

आशा हि परमं दुःखं नैराश्यं परमं सुखम् । 
यथा संच्छिद्य कान्ता सा सुखं सुष्वाप पिङ्गला ॥\\

\subsection{कुरगुरुलक्षणम् ।}

सामिषं कुररं जघ्नुर्बलिनो ये निरामिषाः ।\\[-2mm]
तदामिषं परित्त्यज्य स सुखं समविन्दत ।\\[-2mm]
तथैव सुखमाप्नोति तद्विद्वान्यस्त्वकिञ्चनः ॥\\

\subsection{अर्भकगुरुलक्षणम् ।}
 
न मे मानावमानौ स्तो न चिन्तागेहपुत्रिणाम् ।\\[-2mm]
आत्मक्रीडो ह्यात्मरतो विचरे द्बालवत्सदा ॥\\

\subsection{कुमारीगुरुलक्षणम् ।}

वासो बहूनां कलहो भवेद्वार्ता द्वयोरपि ।\\[-2mm]
एक एव चरेत्तस्मात्कुमार्या इव कङ्कणः ॥\\

\subsection{शरकृद्गुरुलक्षणम् ।}

यथैवमात्मव्यवरुद्धचित्तो 
न वेद किञ्चिद्बहिरन्तरं वा ।\\[-2mm]
तथेषुकारो नृपतिं वृजन्त
मिषौ गतात्मा न ददर्श पार्श्वे ॥\\

\subsection{सर्पगुरुलक्षणम् ।}

गृहारम्भो हि दुःखाय विफलश्चाधृतात्मनः ।\\[-2mm]
सर्पः परकृतं वेश्म प्रविश्य सुखमेधते ॥\\

\subsection{ऊर्णनाभिगुरुलक्षणम् ।}

यथोर्णनाभिर्हृदयादूर्णां संत्यज्य वक्रतः ।\\[-2mm]
तया विहृत्य भूयस्तां ग्रसत्येवं महेश्वरः ॥\\

\subsection{पेशस्कृद् गुरुलक्षणम् ।}

यत्र यत्र मनो देही धारयेत् सततं धिया ।\\[-2mm]
स्नेहाद्द्वेषाद्भयाद्वापि याति तत्तत्स्वरूपताम्  ॥\\
कीटः पेशस्कृतं ध्यायन्कुड्यं तेन प्रवेशितः ।\\[-2mm]
याति तत्तुल्यतां विप्राः पूर्वरूपमसंत्त्यजन् ॥\\

\section{गुरुवैविध्यम्}

इहेष्टकामदं यत्तु परस्मिन्नरकप्रदम् ।\\[-2mm]
यः प्रयच्छति तज्ज्ञानं स निषिद्धगुरुर्भवेत् ॥\\
इहाभीष्टप्रदं यद्वा परत्र स्वर्गभोगदम् ।\\[-2mm]
स काम्यगुरुरित्याहुः तज्ज्ञानं यः प्रयच्छति ॥\\
यन्मोक्षसुखदं ज्ञानं यो ददाति दयापरः ।\\[-2mm]
तं प्राहुर्विहितगुरुं सात्त्विकं निरपेक्षकम् ॥\\
संसारविषयेऽत्यर्थं विरक्तिं वक्ति सर्वदा ।\\[-2mm]
भक्तिमध्यात्मविषये स गुरुर्वाचकस्स्मृतः ॥\\
उपदेशावशिष्टान्यः सूक्ष्मार्थान् सूचयत्यलम् ।\\[-2mm]
यमादिसद्गुणकरान् स भवेत्सूचको गुरुः ॥\\
परब्रह्मकरं ज्ञानमुपदिश्य यथाविधि ।\\[-2mm]
करोति शिष्यं ब्रह्मैक्यं स कारकगुरुर्भवेत् ॥\\

\section{गुरुमहिमा ।}

यथा वह्निसमीपस्थं नवनीतं विलीयते । 
तथा पापं विलीयते सदाचार्यसमीपतः ॥११३॥\\
यदा दीप्तानलः काष्ठं शुष्कमार्द्रञ्च निर्दहेत् ।\\[-2mm]
तथा गुरुकटाक्षस्तु शिष्यपापं दहेत् क्षणात् ॥११४॥\\
यथा महानिलोद्धूतं तूलं दशदिशो व्रजेत् ।\\[-2mm]
तथैव गुरुकारुण्यात् पापराशिः पलायते ॥११५॥\\
दीपदर्शनमात्रेण प्रणश्यति तमो यथा ।\\[-2mm]
सद्गुरोर्दशनाद्देवि तथाज्ञानं विनश्यति ॥११६॥\\
उत्पादकब्रह्मदात्रोर्गरीयान् ब्रह्मदः पिता ।\\[-2mm]
ब्रह्मजन्म हि विप्रस्य प्रेत्य चेह च शाश्वतम् ॥१४६॥\\
ध्यानमूलं गुरोर्मूतिः पूजामूलं गुरोः पदम् ।\\[-2mm]
मन्त्रमूलं गुरोर्वाक्यं मोक्षमूलं गुरोः कृपा ॥१३॥\\
तावदार्त्तिर्भयं शोको लोभमोहभ्रमादयः ।\\[-2mm]
यावन्नायाति शरणं श्रीगुरुं भक्तवत्सलम् ॥१५॥\\
ब्रह्मविष्णुमहेशादिदेवतामुनियोगिनः ।\\[-2mm]
कुर्वन्त्यनुग्रहं तुष्टा गुरौ तुष्टे न संशयः ॥२२॥\\
निग्रहेऽनुग्रहे वापि गुरुः सर्वस्य कारणम् ।\\[-2mm]
निर्गतं यद्गुरोर्वक्त्रात् सर्वं शास्त्रं तदुच्यते ॥९१॥\\
मनुष्यचर्मणा बद्धः साक्षात्परशिवः स्वयम् ।\\[-2mm]
सच्छिष्यानुग्रहार्थाय गूढं पर्यटति क्षितौ ॥५४॥\\
नरवद्दृश्यते लोके श्रीगुरुः पापकर्मणा ।\\[-2mm]
शिववद्दृश्यते लोके भवानि पुण्यकर्मणा ॥५८॥\\
श्रीगुरुं परमं तत्त्वं तिष्ठन्तं चक्षुरग्रतः ।\\[-2mm]
मन्दभाग्या न पश्यन्ति ह्यन्धाः सूर्यमिवोदितम् ॥५९॥\\ 

\section{गुरुगीता ।}

यो गुरुः स शिवः प्रोक्तो यः शिवः स गुरुः स्मृतः ।\\[-2mm]
विकल्पं यस्तु कुर्वीत स भवेत् पातकी गुरौ ॥\\
यदङ्घ्रिकमल द्वन्द्वं द्वन्द्वतापनिवारकम् ।\\[-2mm]
तारकं भवसिन्धोश्च तं गुरुं प्रणमाम्यहम् ॥\\
गुरुरेव जगत्सर्वं ब्रह्मविष्णुशिवात्मकम् ।\\[-2mm]
गुरोः परतरं नास्ति तस्मात्तं पूजयेद्गुरुम् ॥\\
संसारवृक्षमारूढाः पतन्तो नरकार्णवे ।\\[-2mm]
सर्वे येनोद्धृता लोकास्तस्मै श्री गुरवे नमः ॥\\
गुरुर्ब्रह्मा गुरुर्विष्णुः गुरुर्देवो महेश्वरः ।\\[-2mm]
गुरुरेव परं ब्रह्म तस्मै श्री गुरवे नमः ॥\\
अखण्डमण्डलाकारं व्याप्तं येन चराचरम् ।\\[-2mm]
तत्पदं दर्शितं येन तस्मै श्रीगुरवे नमः ॥\\
देहे जीवत्वमापन्नः चैतन्यं निष्फलं परम् ।\\[-2mm]
त्वं पदं दर्शितं येन तस्मै श्री गुरवे नमः ॥\\
अखण्डं परमार्थं सत् ऐक्यं च त्वंतदोः शुभम् ।\\[-2mm]
असिना दर्शितं येन तस्मै श्री गुरवे नमः ॥\\
सर्वश्रुतिशिरोरत्ननीराजितपदाम्बुजम् ।\\[-2mm]
वेदान्ताम्बुजसूर्याभं श्रीगुरुं शरणं व्रजेत् ॥\\
चैतन्यं शाश्वतं शान्तं मायातीतं निरञ्जनम् ।\\[-2mm]
नादबिन्दुकलातीतं तस्मै श्री गुरवे नमः ॥\\
स्थावरं जङ्गमं चेति यत्किञ्चिज्जगतीतले ।\\[-2mm]
व्याप्तं यस्य चिता सर्वं तस्मै श्रीगुरवे नमः ॥\\
त्वं पिता त्वं च मे माता त्वं बन्धुस्त्वञ्च दैवतम् ।\\[-2mm]
संसारप्रीतिभङ्गाय तुभ्यं श्रीगुरवे नमः ॥\\
यत्सत्तया जगत्सत्वं यत्प्रकाशेन भायुतम् ।\\[-2mm]
नन्दनं च यदानन्दात्तस्मै श्रीगुरवे नमः ॥\\
येन चेतयताऽऽपूर्यं चित्तं चेतयते नरः ।\\[-2mm]
जाग्रत् स्वप्नसुषुप्त्यादौ तस्मै श्रीगुरवे नमः ॥\\
यस्य ज्ञानादिदं विश्वमदृश्यं भेदभेदतः ।\\[-2mm]
यत्स्वरूपावशेषं च तस्मै श्रीगुरवे नमः ॥\\
य एव कार्यरूपेण कारणेनापि भाति च ।\\[-2mm]
कार्यकारणनिर्मुक्तस्तस्मै श्रीगुरवे नमः ॥\\
ज्ञानशक्तिस्वरूपाय कामितार्थप्रदायिने ।\\[-2mm]
भुक्तिमुक्तिप्रदात्रे च तस्मै श्रीगुरवे नमः ॥\\
अनेकजन्मसम्प्राप्तकर्मकोटिविदाहिने ।\\[-2mm]
ज्ञानानलप्रभावेन तस्मै श्रीगुरवे नमः ॥\\
न गुरोरधिकं तत्त्वं न गुरोरधिकं तपः ।\\[-2mm]
न गुरोरधिकं ज्ञानं तस्मै श्रीगुरवे नमः ॥\\
मन्नाथः श्रीजगन्नाथो मद्गुरुः श्रीजगद्गुरुः ।\\[-2mm]
ममात्मा सर्वभूतात्मा तस्मै श्रीगुरवे नमः ॥\\
गुरुरादिरनादिश्च गुरुः परम दैवतम् ।\\[-2mm]
गुरोः समानः को वास्ति तस्मै श्रीगुरवे नमः ॥\\
एक एव परो बन्धुर्विषमे समुपस्थिते ।\\[-2mm]
निस्पृहः करुणासिन्धुस्तस्मै श्रीगुरवे नमः ॥\\
गुरुमध्ये स्थितं विश्वं विश्वमध्ये स्थितो गुरुः ।\\[-2mm]
विश्वरूपो विरूपोऽसौ तस्मै श्रीगुरवे नमः ॥\\
भवारण्यप्रविष्टस्य दिङ्मोहभ्रातचेतसः ।\\[-2mm]
येन संदर्शितः पन्थास्तस्मै श्रीगुरवे नमः ॥\\
तापत्रयाग्नितप्तानां श्रान्तानां प्राणिनां मुमे ।\\[-2mm]
गुरुरेव परा गङ्गा तस्मै श्रीगुरवे नमः ॥\\
हेतवे सर्वजगतां संसारार्णवसेतवे ।\\[-2mm]
प्रभवे सर्वविद्यानां शंभवे गुरवे नमः ॥\\
ध्यानमूलं गुरोर्मूर्तिः पूजामूलं गुरोःपदम् ।\\[-2mm]
मन्त्रमूलं गुरोर्वाक्यं मोक्षमूलं गुरोःकृपा ॥\\
हरणं भवरोगस्य तरणं क्लेशवारिधेः ।\\[-2mm]
भरणं सर्वलोकस्य शरणं चरणं गुरोः ॥\\
शिवे रुष्टे गुरुस्त्राता गुरै रुष्टे न कश्चन ।\\[-2mm]
तस्मात्परगुरुं लब्ध्वा तमेव शरणं व्रजेत् ॥\\
अत्रिनेत्रः शिवः साक्षात् द्विभुजश्चापरो हरिः ।\\[-2mm]
योऽचतुर्वदनो ब्रह्मा श्रीगुरुः कथितः प्रिये ॥\\
नित्याय निर्विकाराय निरवद्याय योगिने ।\\[-2mm]
निष्कलाय निरीहाय शिवाय गुरवे नमः ॥\\
शिष्यहृत्पद्मसूर्याय सत्याय ज्ञानरूपिणे ।\\[-2mm]
वेदान्तवाक्यवेद्याय शिवाय गुरवे नमः ॥\\
उपायोपेयरूपाय सदुपायप्रदर्शिने ।\\[-2mm]
अनिर्वाच्याय वाच्याय शिवाय गुरवे नमः ॥\\
कार्यकारणरूपाय रूपारूपाय ते सदा ।\\[-2mm]
अप्रमेयस्वरूपाय शिवाय गुरवे नमः ॥\\
दृग्दृश्यद्रष्टृरूपाय निष्पन्ननिजरूपिणे ।\\[-2mm]
अपारायाद्वितीयाय शिवाय गुरवे नमः ॥\\
 गुणधाराय गुणिने गुणस्वरूपिणे ।\\[-2mm]
जन्मिने जन्महीनाय शिवाय गुरवे नमः ॥\\
अनाद्यायाखिलाद्याय मायिने गतमायिने ।\\[-2mm]
अरूपाय स्वरूपाय शिवाय गुरवे नमः ॥\\
सर्वमन्त्रस्वरूपाय सर्वतन्त्र स्वरूपिणे ॥\\
सर्वगाय समस्ताय शिवाय गुरवे नमः ॥\\
मनुष्यचर्मणाऽऽबद्धः साक्षात्परशिवः स्वयम् ।\\[-2mm]
गुरुरित्यभिधां गृह्णन् गूढः पर्यटति क्षितौ ॥\\
गुरोः कृपा प्रसादेन ब्रह्मविष्णुमहेश्वराः ।\\[-2mm]
समर्था तत्प्रसादो हि केवलं गुरु सेवया ॥\\
श्रीमत्परब्रह्म गुरुं स्मरामि श्रीमत्परब्रह्मगुरुं भजामि ।\\[-2mm]
श्रीमत्परब्रह्मगुरुं वदामि श्रीमत्परब्रह्मगुरुं नमामि ॥\\
एकं नित्यं विमलमचलं सर्वधीसाक्षिभूतम् ।\\[-2mm]
भावातीतं त्रिगुणरहितं सद्गुरुं तं नमामि ॥\\
आनन्दमानन्दकरं प्रसन्नं 
ज्ञानस्वरूपं निजबोधयुक्तम् ।\\[-2mm]
योगीन्द्रमीड्यं भवरोगवेद्यं 
श्रीमद्गुरुं नित्यमहं नमामि ॥\\
नित्यं शुद्धं निराभासं निराकारं निरञ्जनम् ।\\[-2mm]
नित्यबोधचिदानन्दं गुरुं ब्रह्म नभाम्यहम् ॥\\
सच्चिदानन्दरूपाय व्यापिने परमात्मने ।\\[-2mm]
नमः श्रीगुरुनाथाय प्रकाशानन्दमूर्तये ॥\\
सच्चिदानन्दरूपाय कृष्णाय क्लेशहारिणे ।\\[-2mm]
नमो वेदान्तवेद्याय गुरवे बुद्धिसक्षिणे ॥\\
यस्य प्रसादादहमेव विष्णुर्मय्येव सर्वं परिकल्पितं च ।\\[-2mm]
इत्थं विजानामि सदात्मतत्वं तस्याङ्घ्रिपद्मं प्रणतोऽस्मि नित्यम् ॥\\
नाकारं नो विकारं नहि जनि मरणं नैव पुण्यं न पापम् ।\\[-2mm]
नो तत्वं तत्वमेकं सहजसमरसं सद्गुरुं तं नमामि ॥\\

\section{दीक्षोपदेशः}

\subsection{उपदेशत्रैविध्यम् ।}

यथा पिपीलिका मन्दमन्दं वृक्षाग्रगं फलम् ।\\[-2mm]
चिरेणाप्नोति कर्मोपदेशश्चापि तथा स्मृतः ॥३१॥\\
यथा कपिश्च शाखायां शाखामुल्लंघ्य यत्नतः ।\\[-2mm]
फलं प्राप्नोति धर्मस्य चोपदेशस्तथा प्रिये ॥३२॥\\
यथा वियद्गमः शीघ्रं फल एव निषीदति ।\\[-2mm]
तथा ज्ञानोपदेशश्च कथितः कुलनायिके ॥३३॥\\

\subsection{दीक्षात्रैविध्यम् ।}

दिव्यभावप्रदानाच्च क्षालनात् कल्मषस्य च । 
दीक्षेति कथिता सद्भि र्भवबन्ध विमोचनात् ॥\\
यया चोन्मीलितात्मानो भवन्ति पशवः शिवाः ।\\[-2mm]
सा दीक्षा ह्युदिता देवि पशुपाशविमोचिका ॥\\
मन्त्रौषधेः यथा हन्याद् विषशक्तिं कुलेश्वरि । 
पशुपाशं तथा छिन्द्यात् दीक्षया मन्त्रवित् क्षणात् ॥ 
स्पर्शाख्या देवि दृक्संज्ञा मानसाख्या महेश्वरि ।\\[-2mm]
क्रियायासादिरहिता देवि दीक्षा त्रिधा स्मृता ॥३४॥\\  
यथा पक्षी स्वपक्षभ्यां शिशून् संवर्द्धयेच्छनैः ।\\[-2mm]
स्पर्शदीक्षोपदेशश्च तादृशः कथितः प्रिये ॥३५॥\\
स्वापत्यानि यथा मत्स्यो वीक्षणेनैव पोषयेत् ।\\[-2mm]
दृग्भ्यां दीक्षोपदेशश्च तादृशः परमेश्वरि ॥३६॥\\
यथा कूर्मः स्वतनयान् ध्यानमात्रेण पोषयेत् ।\\[-2mm]
वेधदीक्षोपदेशश्च मानसः स्यात् तथाविधः ॥३७॥\\
शक्तिपातानुसारेण शिष्योऽनुग्रहमर्हति ।\\[-2mm]
यत्र शक्तिर्न पतति तत्र सिद्धिर्न जायते ॥३८॥\\
उपपातकलक्षाणि महापातककोटिशः ।\\[-2mm]
क्षणाद्दहति देवेशि दीक्षा हि विधिना कृता ॥८५॥\\
उपासनशतेनापि यां विना नैव सिध्यति ।\\[-2mm]
तां दीक्षामाश्रयेद् यत्नात् श्रीगुरोर्मन्त्रसिद्धये ॥८८॥\\
रसेन्द्रेण यथा विद्धमयः सुवर्णतां व्रजेत् ।\\[-2mm]
दीक्षाविद्धस्तथा ह्यात्मा शिवत्वं लभते प्रिये ॥८९॥\\
विधवायाः सुतादेशात् कन्यायाः पितुराज्ञया ।\\[-2mm]
नाधिकारः स्वतो नार्याः भार्यायाः भर्तुराज्ञया ॥\\
स्याद्वेदाध्ययने शूद्रो नाधिकारी यथा प्रिये ।\\[-2mm]
तथैवादीक्षितश्चापि नाधिकारी कुलेश्वरि ॥\\
जन्मान्तर सहस्रेषु कृतपाप प्रणाशनात् ।\\[-2mm]
परदेवप्रकाशाच्च जप इत्यभिधीयते ॥\\
मननात् तत्वरूप्स्य देवस्यामिततेजसः ।\\[-2mm]
त्रायते सर्वभयतस्तस्मान्मन्त्र शरीरितः ॥\\
देहमास्थाय भक्तानां व्रदनाच्च पार्वति ।\\[-2mm]
तापत्रयादिशमनाद्देवता परिकीर्तिता ॥\\
पूर्वजन्मानुशमनाज्जन्ममृत्युनिवारणात् ।\\[-2mm]
सम्पूर्णफलदानाच्च पूजेति कथिता प्रिये ॥\\
तत्वात्मकस्य देवस्य परिवारवृतस्य च ।\\[-2mm]
नवानन्दप्रजननात्तर्पणं समुदाहृतम् ॥\\
पञ्चाङ्गोपासनेनेष्टदेवता प्रीतिदानतः ।\\[-2mm]
पुरश्चरति भक्तस्य तत्पुरश्चरणं प्रिये ॥\\
आत्मसिद्धिप्रदानाच्च सर्वरोगनिवारणात् ।\\[-2mm]
नवसिद्धिप्रदानाच्च आसनं कथितं प्रिये ॥\\
दह्यन्ते ध्यायमानानां धातूनाञ्च यथा मलम् ।\\[-2mm]
तथेऽन्द्रियाणां दह्यन्ते दोषाः प्राणस्य संयमात् ॥  
जपध्यानं विनाऽगर्भः सगर्भ्स्तत्द्विपर्ययात् ।\\[-2mm]
अगर्भाद् गर्भसंयुक्तः प्राणायमः शताधिकः ॥\\
प्राणायामैः विशुद्धात्मा यद्यत् कर्म करोति हि ।\\[-2mm]
तत्तत् फलत्यसंदेहस्वप्रयत्नेन वा कृतम् ॥\\
यः शिवः स्वर्गः सूक्ष्मश्चोन्मना निष्फलोऽव्ययः । 
व्योमकारो ह्यजोऽनन्तः स कथं पूज्यते प्रिये ॥\\
अत एव शिवस्साक्षात् गुरुरूपं समाश्रितः ।\\[-2mm]
भक्त्या संपूजितो देवि भुक्तिं मुक्तिं प्रयच्छति ॥\\
सद्भक्तरक्षणायैव निराकारोऽपि साकृतिः ।\\[-2mm]
शिवः कृपानिधिर्लोके संसारीव हि चेष्टते ॥\\
विशिष्टं दीयते ज्ञानं क्षीयते पापसञ्चयः ।\\[-2mm]
मायाकर्ममलोद्भूता यतो दीक्षेति सा स्मृता ॥\\
	
\subsection{दीक्षासप्तविधत्वम् ।}

यौगिकी मानसी चान्या स्पार्शिकी चाक्षुषी तथा ।\\[-2mm]
वाचिकी तान्त्रिकी हौत्री दीक्षेयं सप्तधा मता ॥\\
योगमार्गेण शिष्यस्य देहमाविश्य यद्गुरुः ।\\[-2mm]
बोधयेद्देवतातत्त्वं सा दीक्षा यौगिकी मता ॥\\
आदिश्य देवताभावं तेजसा यद्दयालुना ।\\[-2mm]
शिष्यं गुरुरनुध्यायेत् सा दीक्षा मानसी भवेत् ॥\\
हार्द्रमन्त्रमयं ज्योतिर्हस्ते सञ्चित्य यद्गुरुः ।\\[-2mm]
ब्रह्मरन्ध्रे स्पृशेच्छिष्यं सा दीक्षा स्पार्शिकी भवेत् ॥\\
देवताविग्रहो भूत्वा यच्छिष्यं दयया गुरुः ।\\[-2mm]
प्रसन्नया दृशा पश्येत् सा दीक्षा चाक्षुषी स्मृता ॥\\
सञ्चित्य हृदये तत्त्वं कर्णे शिष्यस्य यद्गुरुः ।\\[-2mm]
मन्त्रमुच्चारयेत्सम्यक् सा दीक्षा मान्त्रिकी भवेत् ॥\\
दिव्यानामपि मन्त्राणां यद्व्याख्यानेन देशिकः ।\\[-2mm]
शिष्यायोपदिशेत्तत्वं सा दीक्षा तान्त्रिकी स्मृता ॥\\
सकुण्डमण्डपा होमक्रियाकौशलशालिनी ।\\[-2mm]
अभिषेकादि सम्पन्ना हौत्री दीक्षेयमुच्यते ॥\\
तारतम्यं समालोक्य दीक्षा कार्या विपश्चिता ।\\[-2mm]
शक्त्तिपातानुसारेण शिष्योऽनुग्रहमर्हति ॥\\
अतः परं समालक्ष्य गुरुश्शिष्यस्य योग्यताम् ।\\[-2mm]
षडध्वशुद्धिं कुर्वीत सर्वबन्धविमुक्तये ॥\\
कलातत्वं च भुवनम् वर्णः पदमतः परम् ।\\[-2mm]
मन्त्रं चेति समासेन षडध्वा परिपठ्यते ॥\\

\subsection{मन्त्रसिद्धिः ।}

आब्रह्मबीजदोषाश्च नियमातिक्रमोद्भवाः ।\\[-2mm]
ज्ञानाज्ञानकृतां सर्वे प्रणश्यन्ति जपात् प्रिये ॥६॥\\
संसारे दुःखभूयिष्ठे यदीच्छेत् सिद्धिमात्मनः ।\\[-2mm]
पञ्चाङ्गोपासनेनैव मन्त्रजापी व्रजेत् सुखम् ॥७॥\\ 
पूजा त्रैकालिकी नित्यं जपस्तर्पणमेव च ।\\[-2mm]
होमो ब्राह्मणभुक्तिश्च पुरश्चरणमुच्यते ॥८॥\\
यद् यदङ्गं विहीयेत तत्संख्याद्विगुणो जपः ।\\[-2mm]
कुर्याद् द्वित्रिचतुःपञ्चसंख्यां वा साधकः प्रिये ॥९॥\\
कुर्वीत चाङ्गसिध्यर्थं तदशक्तौ स भक्तितः ।\\[-2mm]
तच्चेदङ्गं विहीयेत मन्त्री नेष्टमवाप्नुयात् ॥१०॥\\
उपदेशस्य सामर्थ्यात् श्रीगुरोश्च प्रसादतः ।\\[-2mm]
मन्त्रप्रभावाद्भक्त्या च मन्त्रसिद्धिः प्रजायते ॥१३॥\\
अनेककोटिमन्त्राणि चित्ताकुलकराणि च ।\\[-2mm]
मन्त्रं गुरुकृपाप्राप्तमेकं स्यात् सर्वसिद्धिदम् ॥१९॥\\
यथा घटश्च कलशः कुम्भश्चैकार्थवाचकः ।\\[-2mm]
तथा देवश्च मन्त्रश्च गुरुश्चैकार्थ उच्यते ॥६४॥\\
यथा देवस्तथा मन्त्रो यथा मन्त्रस्तथा गुरुः ।\\[-2mm]
देवमन्त्रगुरूणाञ्च पूजया सदृशं फलम् ॥६५॥\\
मन्त्रार्थं मन्त्रचैतन्यं योनिमुद्रां न वेत्ति यः ।\\[-2mm]
शतकोटिजपेनापि तस्य सिद्धिर्न जायते ॥५९॥\\
चैतन्यरहिता मन्त्राः प्रोक्ता वर्णास्तु केवलम् ।\\[-2mm]
फलं नैव प्रयच्छन्ति लक्षकोटिजपादपि ॥६१॥\\
हृत्कण्ठग्रन्थिभेदश्च सर्वावयववर्धनम् ।\\[-2mm]
आनन्दाश्रु च पुलको देहावेशः कुलेश्वरि ।\\[-2mm]
गद्गदोक्तिश्च सहसा जायते नात्र संशयः ॥६३॥\\
सकृदुच्चरितेऽप्येवं मन्त्रे चैतन्यसंयुते ।\\[-2mm]
दृश्यन्ते प्रत्यया यत्र पारम्पर्य तदुच्यते ॥६४॥\\
शिवे मन्त्रे गुरौ यस्य भावना सदृशी भवेत् ।\\[-2mm]
भोगो मोक्षश्च सिद्धिश्च शीघ्रं तस्य भवेद्ध्रुवम् ॥\\

\section{जपविधिः ।}

सूर्यस्याग्नेर्गुरोरिन्दोर्दीपस्य च जलस्य च ।\\[-2mm]
गोविप्रकुलवृक्षाणां सन्निधौ शस्यते जपः ॥२५॥\\
गृहे शतगुणं विद्याद् गोष्ठे लक्षगुणं भवेत् ।\\[-2mm]
कोटिर्देवालये पुण्यमनन्तं शिवसन्निधौ ॥२६॥\\
तूलकम्बलवस्त्राणां सिंहव्याघ्रमृगाजिनम् ।\\[-2mm]
कल्पयेदासनं धीमान् सौभाग्यज्ञानवृद्धिदम् ॥३३॥\\
पद्मस्वस्तिकवीरादिष्वासनेषूपविश्य च ।\\[-2mm]
जपार्चनादिकं कुर्यादन्यथा निष्फलं भवेत् ॥३४॥\\
एकैकमङ्गुलीभिः स्याद्रेखाभिर्दशधा फलम् ।\\[-2mm]
मणिभिः शतसाहस्रं माणिक्याऽनन्तमुच्यते ॥५०॥\\
त्रिंशद्भिः स्याद्धनं पुष्टिः सप्तविंशतिभिर्भवेत् ।\\[-2mm]
पञ्चविंशतिभिर्मोक्षं पञ्चदश्याभिचारके ।\\[-2mm]
पञ्चाशद्भिः कुलेशानि सर्वसिद्धिरुदीरिता ॥५१॥\\
उच्चैर्जपोऽधमः प्रोक्त उपांशुर्मध्यमः स्मृतः ।\\[-2mm]
उत्तमो मानसो देवि त्रिविधः कथितो जपः ॥५४॥\\

भक्ष्यं हविष्यं शाकादि विहितानि फलान्यपि ।\\[-2mm]
मूलं शक्तु यवानाञ्च शस्तान्येतानि मन्त्रिणाम् ॥७४॥\\
यस्यान्न पानपुष्टाङ्गः कुरुते धर्मसञ्चयम् ।\\[-2mm]
अन्नदातुः फलं चार्द्धं कर्तुश्चार्द्धं न संशयः ॥७५॥\\
तस्मात् सर्वप्रयत्नेन परान्नं वर्जयेत् सुधीः ।\\[-2mm]
पुरश्चरणकाले च काम्यकर्मस्वपीश्वरि ॥७६॥\\
जिह्वा दग्धा परान्नेन करौ दग्धौ प्रतिग्रहात् ।\\[-2mm]
मनो दग्धं परस्त्रीभिः कार्यसिद्धिः कथं भवेत् ॥७७॥\\
शान्तः शुचिमिताहारो भूशायी भक्तिमान् वशी ।\\[-2mm]
निर्द्वन्द्वः स्थिरधीर्मौनी संयतात्मा जपेत् प्रिये ॥११०॥\\
तन्निष्ठस्तद्गतप्राणस्तच्चित्तस्तत्परायणः ।\\[-2mm]
तत्पदार्थानुसन्धानं  कुर्वन् मन्त्रं जपेत् प्रिये ॥११३॥\\
जपात् श्रान्तः पुनर्ध्यायेद्ध्यानात् श्रान्तः पुनर्जपेत् ।\\[-2mm]
जपध्यानादियुक्तस्य क्षिप्रं मन्त्रः प्रसिध्यति ॥११४॥\\
यदृच्छया श्रुतं मन्त्रं दृष्टेनापि छलेन च ।\\[-2mm]
पत्रे स्थितं वा चाध्याप्य तज्जपः स्यादनर्थकृत्॥२०॥\\

\section{जपकाले विवर्जनीयानि ।}

जाड्यं दुःखं तृणच्छेदं विवादं वा मनोरथम् ।\\[-2mm]
बहिस्तु देहवायुञ्च जपकाले विवर्जयेत् ॥१०९॥\\
उष्णीशी कञ्चुकी नग्नो मुक्तकेशो गणावृतः।\\[-2mm]
अपवित्रोत्तरीयश्चाशुचिर्गच्छंश्च नो जपेत् ॥१०८॥\\
आलस्यं जृम्भणं निद्रां क्षुतं निष्ठीवनं भयम् ।\\[-2mm]
नीचाङ्गस्पर्शनं कोपं जपकाले विवर्जयेत् ॥१०६॥\\
मलिनाम्बरकेशादिमुखदौर्गन्धसंयुतः ।\\[-2mm]
यो जपेत्तं दहत्याशु देवता सुजुगुप्सिता ॥१०५॥\\
विण्मूत्रत्यागशेषादियुक्तः कर्म करोति यः ।\\[-2mm]
जपार्चनादिकं सर्वमपवित्रं भवेत् प्रिये ॥१०४॥\\
अमेध्येन तु देहेन न्यासं देवार्चनं जपम् ।\\[-2mm]
होमं कुर्वन्ति चेन्मूढास्तत् सर्वं निष्फलं भवेत् ॥१०३॥\\
अत्याहारः प्रलापश्च प्रजल्पो नियमाग्रहः ।\\[-2mm]
अन्यासङ्गश्च लौल्यञ्च षड्भिर्मन्त्रो न सिध्यति ॥१०८॥\\
मनोऽन्यत्र शिवोऽन्यत्र शक्तिरन्यत्र मारुतः ।\\[-2mm]
न सिध्यति वरारोहे लक्षकोटिजपादपि ॥१००॥\\
धनार्थं गम्यते तीर्थं दम्भार्थं क्रियते तपः ।\\[-2mm]
ख्यात्यर्थं दीयते दानं कथं सिद्धिर्वरानने ॥१०२॥\\


\section{आत्मविद्या}

गुरोः कॄपा प्रसादेन ब्रह्माहमिति भावयेत् ।\\[-2mm]
अनेन मुक्तिमार्गेण ह्यात्मज्ञानं प्रकाशते ॥\\
श्रीगुरुं सच्चिदानन्दं भावातीतं विभाव्य च ।\\[-2mm]
तन्निदर्शितमार्गेण ध्यानमग्नो भवेत्सुधीः ॥\\
परात्परतरं ध्यायेच्छुद्धस्फटिकसन्निभम् ।\\[-2mm]
हृदयाकाशमध्यस्थं स्वाङ्गुष्ठपर्रिमाणकम् ॥\\
अङ्गुष्ठमात्रं पुरुषं ध्यायतश्चिन्मयं हृदि ।\\[-2mm]
तत्र स्फुरति यो भावः शृणु तत्कथयामि ते ॥\\
यथा निजस्वभावेन केयूरकटकादयः ।\\[-2mm]
सुवर्णत्वेन तिष्ठन्ति तथाऽहं ब्रह्म शाश्वतम् ॥\\
एवं ध्यायन् परंब्रह्म स्थातव्यं यत्र कुत्रचित् ।\\[-2mm]
कीटो भृङ्ग इव ध्यानात् ब्रह्मैव भवति स्वयम् ॥\\
यदृच्चया चोपपन्नं ह्यल्पं बहुळमेव वा ।\\[-2mm]
नीरागेणैव भुञ्जीत स्वाभ्याससमये मुदा ॥\\
एकमेवाद्वितीयोऽहम् गुरुवाक्यात्सुनिश्चितम् ।\\[-2mm]
एवमभ्यस्यतो नित्यं न सेव्यं वै वनान्तरम् ॥\\
अभ्यासान्निमिषेणैव समाधिमधिगच्छति ।\\[-2mm]
जन्मकोटिकृतं पापं तत्क्षणादेव नश्यति ॥\\
न तत्सुखं सुरेन्द्रस्य न सुखं चक्रवर्तिनाम् ।\\[-2mm]
यत्सुखं वीतरागस्य सदा संतुष्टचेतसः ॥\\
रसं ब्रह्म पिबेद्यश्च तृप्तो यः परमात्मनि ।\\[-2mm]
इन्द्रं च मनुते रङ्कं नृपाणां तत्र का कथा ॥\\
देशः पूतो जनाः पूतास्तादृशो यत्र तिष्ठति ।\\[-2mm]
तत्कटाक्षोऽथ संसर्गः परस्मै श्रेयसेप्यलम्  ॥\\
देही ब्रह्म भवेदेवं प्रसादाद्ध्यानतो गुरोः ।\\[-2mm]
 नराणां च फलप्राप्तौ भक्तिरेव हि कारणम् ॥\\

\section{अथ गुर्वष्टकम्}

शरीरं सुरूपं तथा वा कलत्रं यशश्चारु चित्रं धनं मेरुतुल्यम् ।\\
मनश्चेन्न लग्नं गुरोरंघ्रिपद्मे ततः किं ततः किं ततः किं ततः किम् ॥१॥\\
कलत्रं धनं पुत्रपौत्रादि सर्वं गृहं बान्धवाः सर्वमेतद्धि जातम् ।\\
मनश्चेन्न लग्नं गुरोरङ्घ्रिपद्मे ततः किं ततः किं ततः किं ततः किम् ॥२॥\\
षडङ्गादिवेदो मुखे शास्त्रविद्या कवित्वादि गद्यं सुपद्यं करोति ।\\
मनश्चेन्न लग्नं गुरोरङ्घ्रिपद्मे ततः किं ततः किं ततः किं ततः किम् ॥३॥\\
विदेशेषु मान्यः स्वदेशेषु धन्यः सदाचारवृत्तेषु मत्तो न चान्यः ।\\
मनश्चेन्न लग्नं गुरोरङ्घ्रिपद्मे ततः किं ततः किं ततः किं ततः किम् ॥४॥\\
क्षमामण्डले भूपभूपालवृन्दैः सदा सेवितं यस्य पादारविन्दम् ।\\
मनश्चेन्न लग्नं गुरोरङ्घ्रिपद्मे ततः किं ततः किं ततः किं ततः किम् ॥५॥\\
यशो मे गतं दिक्षु दानप्रतापाज्जगद्वस्तु सर्वं करे यत्प्रसादात् ।\\
मनश्चेन्न लग्नं गुरोरङ्घ्रिपद्मे ततः किं ततः किं ततः किं ततः किम् ॥६॥\\
न भोगे न योगे न वा वाजिराजौ न कान्तामुखे नैव वित्तेषु चित्तम् ।\\
मनश्चेन्न लग्नं गुरोरङ्घ्रिपद्मे ततः किं ततः किं ततः किं ततः किम् ॥७॥\\
अरण्ये न वा स्वस्य गेहे न कार्ये न देहे मनो वर्तते मे त्वनर्घ्ये ।\\
मनश्चेन्न लग्नं गुरोरङ्घ्रिपद्मे ततः किं ततः किं ततः किं ततः किम् ॥८॥\\
गुरुरष्टकं यः पठेत्पुण्यदेही यतिर्भूपतिर्ब्रह्मचारी च गेही ।\\
लभेद्वाञ्छितार्थं पदं ब्रह्मसंज्ञं गुरोरुक्तवाक्ये मनो यस्य लग्नम् ॥९॥\\

\section{अथ  गुरुपादुकास्तोत्रम्}

अनन्तसंसारसमुद्रतार नौकायिताभ्यां गुरुपादुकाभ्याम् । \\
वैराग्यसाम्राज्यदपूजनाभ्यां नमो नमः श्रीगुरुपादुकाभ्याम् ॥१॥\\
कवित्ववाराशिनिशाकराभ्यां दौर्भाग्यदावाम्बुदमालिकाभ्याम् ।\\
दूरिकृतानम्रविपत्ततिभ्यां नमो नमः श्रीगुरुपादुकाभ्याम् ॥२॥\\
नता ययोः श्रीपतितां समीयुः कदाचिदप्याशु दरिद्रवर्याः ।\\
मूकाश्च वाचस्पतितां हि ताभ्यां नमो नमः श्रीगुरुपादुकाभ्याम् ॥३॥\\
नालीकनीकाशपदाहृताभ्यां नानाविमोहादिनिवारिकाभ्यां ।\\
नमज्जनाभीष्टततिप्रदाभ्यां नमो नमः श्रीगुरुपादुकाभ्याम् ॥४॥\\
नृपालिमौलिव्रजरत्नकान्तिसरिद्विराजत् झषकन्यकाभ्याम् ।\\
नृपत्वदाभ्यां नतलोकपङ्केः नमो नमः श्रीगुरुपादुकाभ्याम् ॥५॥\\
पापान्धकारार्कपरम्पराभ्यां तापत्रयाहीन्द्रखगेश्वराभ्याम् ।\\
जाड्याब्धिसंशोषणवाडवाभ्यां नमो नमः श्रीगुरुपादुकाभ्याम् ॥६॥\\
शमादिषट्कप्रदवैभवाभ्यां समाधिदानव्रतदीक्षिताभ्याम् ।\\
रमाधवाङ्घ्रिस्थिरभक्तिदाभ्यां नमो नमः श्रीगुरुपादुकाभ्याम् ॥७॥\\
स्वार्चापराणामखिलेष्टदाभ्यां स्वाहासहायाक्षधुरन्धराभ्याम् ।\\
स्वान्ताच्छ भावप्रदपूजनाभ्यां नमो नमः श्रीगुरुपादुकाभ्याम् ॥८॥\\
कामादिसर्पव्रजगारुडाभ्यां विवेकवैराग्यनिधिप्रदाभ्याम् ।\\
बोधप्रदाभ्यां द्रुतमोक्षदाभ्यां नमो नमः श्रीगुरुपादुकाभ्याम् ॥९॥\\
नमो नमस्तेऽस्तु सहस्रकृत्वः पुनश्चभूयोऽपि नमो नमस्ते ।\\
नमः पुरस्थादथ पृष्ठतस्ते नमोऽस्तु ते सर्वत एव सर्व ॥१०॥\\
त्वमेव माता च पिता त्वमेव त्वमेव बन्धुश्च सखा त्वमेव ।\\
त्वमेव विद्या द्रविणं त्वमेव त्वमेव सर्वं मम देवदेव ॥११॥

\section{अथ श्रीगुरुगीता}

\begin{center} \begin{small} श्रीगणेशाय नमः श्री दत्तप्रभवे नमः\end{small}\end{center}

हंसाभ्यां परिवृत्तहार्दकमले शुद्धे जगत्कारणं\\
विश्वाकारमनेकदेहनिलयं स्वच्छन्दमानन्दकम् ।\\
सर्वाधारमखण्डचिद्घनरसं पूर्णं ह्यनन्तं शुभं\\
प्रत्यक्षाक्षरविग्रहं गुरुवरं ध्यायेद्विभुं शाश्वतम् ॥\\
नमामि सद्गुरुं शान्तं प्रत्यक्षं शिवरूपिणम् ।\\
शिरसा योगपीठस्थं मुक्तिकाम्यार्थसिद्धये ॥\\
ऋषय ऊचुः –\\
गुह्यात् गुह्यतरा सारा गुरुगीता विशेषतः ।\\
तव प्रसादाच्छ्रोतव्या तां सर्वां ब्रूहि सूत नः ॥१॥\\
सूत उवाच –\\
कैलासशिखरे रम्ये भक्तानुग्रहतत्परम् ।\\
प्रणम्य पार्वती भक्त्या शङ्करं परिपृच्छति ॥२॥\\
पार्वत्युवाच –\\
नमस्ते देवदेवेश परात्पर जगद्गुरो ।\\
सदशिव महादेव गुरुदीक्षां प्रयच्छ मे ॥३॥\\
भगवन् सर्वधर्मज्ञ व्रतानामुत्तमोत्तमम् ।\\
ब्रूहि मे कृपया शम्भो गुरुमाहात्म्यमद्भुतम् ॥४॥\\
केन मार्गेण भो स्वामिन् देही ब्रह्ममयो भवेत् ।\\
कुरुमेवानुग्रहं देव नमामि चरणौ तव ॥५॥\\
ईश्वर उवाच – \\
मम रूपासि देवि त्वं त्वद्भक्त्या तद्वदाम्यहम् ।\\
लोकोपकारकः प्रश्नः कृतः केनापि नो पुरा ॥६॥\\
यो गुरुः स शिवः प्रोक्तो यः शिवः स गुरुः स्मृतः ।\\
विकल्पं यस्तु कुर्वीत स भवेत् पातकी गुरौ ॥७॥\\
दुर्लभं त्रिषु लोकेषु तच्छृणु प्रवदाम्यहम् ।\\
गुरुं ब्रह्म विना नान्यत् सत्यं सत्यं वरानने ॥८॥\\
वेदशास्त्रपुराणानि सेतिहासाकानि च ।\\
मन्त्रतन्त्रादिविद्याश्च स्मृतिरुच्चाटनादिकम् ॥९॥\\
शैवशाक्तागमादीनि ह्यन्ये च बहवो मताः ।\\
भ्रामकाः सर्व एवैते जीवानामल्पचेतसाम् ॥१०॥\\
यज्ञो व्रतं तपोदानं जपस्तीर्थं तथैव च ।\\
सर्वेषामेव जन्तूनां सर्वे मार्गः प्रतारकाः ॥११॥\\
जपस्तपोव्रतं तीर्थं यज्ञोदानं तथैव च ।\\
गुरुतत्त्वमविज्ञाय सर्वं व्यर्थं भवेत्प्रिये ॥१२॥\\
गुरोर्विद्यात्मनो नान्यत् सत्यं सत्यं न संशयः ।\\
तल्लाभार्थं प्रयत्नस्तु कर्तव्यो हि मनीषिभिः ॥१३॥\\
रूढाविद्याजगन्माया देहेस्तिध्वान्तरूपिणे ।\\
तद्वारकः प्रकाशश्च गुरुशब्देन कथ्यते ॥१४॥\\
यदङ्घ्रिकमलद्वन्द्वं द्वन्द्वतापनिवारकम् ।\\
तारकं भवसिन्धोश्च तं गुरुं प्रणमाम्यहम् ॥१५॥\\
देही ब्रह्म भवेद्यस्मात् तदिदानीं प्रकाशये ।\\
सर्वपापविशुद्धात्मा श्रीगुरोः पादसेवनम् ॥१६॥\\
गुरोः पादोदकं पीत्वा धृत्वा शिरसि पावनम् ।\\
सर्वतीर्थावगाहस्य सम्प्राप्नोति फलं नरः ॥१७॥\\
शोषणं पापपङ्कस्य दीपनं ज्ञानतेजसः ।\\
गुरोः पादोदकं देवि संसारार्णवतारकम् ॥१८॥\\
अविद्यामूलनाशाय जन्मकर्मनिवृत्तये ।\\
ज्ञानवैराग्यसिद्ध्यर्थं गुरुपादोदकं पिबेत् ॥१९॥\\
गुरुपादोदकं पानं गुरोरुच्छिष्टभोजनम् ।\\
गुरुमूर्तेः सदाध्यानं गुरुस्तोत्रं परो जपः ॥२०॥\\
स्वदेशिकस्यैव च नामकीर्तनं भवेदनन्तस्य शिवस्य कीर्तनम् ।\\
स्वदेशिकस्यैव च रूपचिन्तनं भवेदनन्तस्य शिवस्य चिन्तनम् ॥२१॥\\
यत्पादपांसवः सन्तः केऽपि संसारवारिधेः ।\\
सेतुबन्धाय कल्पन्ते देशिकं तमुपास्महे ॥२२॥\\
काशीक्षेत्रं निवासश्च जाह्नवी चरणोदकम् ।\\
गुरुर्विश्वेश्वरः साक्षात् तारकं ब्रह्मनिश्चितम् ॥२३॥\\
गुरोः पादाङ्कितं यत्र गया साऽधोक्षजोद्भवा ।\\
तीर्थराजः प्रयागोऽसौ गुरुमूर्त्यै नमो नमः ॥२४॥\\
गुरुमूर्तिं स्मरेन्नित्यं गुरोर्नाम सदा जपेत् ।\\
गुरोराज्ञां प्रकुर्वीत गुरोरन्यं न भावयेत् ॥२५॥\\
गुरुवक्त्रे स्थिता  विद्या प्राप्यते तत्प्रसादतः ।\\
तस्मात्तं देशिकं ध्यायेद्यथा योषित्प्रियं स्वकम् ॥२६॥\\
स्वाश्रमं च स्वजातिं च स्वकीर्तिं पुष्टिवर्धनम् ।\\
एतत्सर्वं परित्यज्य गुरुमेव समाश्रयेत् ॥२७॥\\
अनन्याश्चिन्तयन्तो ये ध्रुवं तेषां परं पदम् ।\\
तस्मात्सर्वप्रयत्नेन गुरोराराधनं कुरु ॥२८॥\\
गुरोर्मुखाच्च सम्प्राप्य देवि ब्रह्मात्मसंविदम् ।\\
त्रैलोक्यस्फुटवक्तारो देवर्षिपितृमानवाः ॥२९॥\\
गुकारश्चान्धकारो हि रुकारस्तेज उच्यते ।\\
अज्ञानग्रासकं ब्रह्म गुरुरेव न संशयः ॥३०॥\\
गुकारश्चान्धकारस्तु रुकारस्तन्निरोधकः ।\\
अन्धकारविनाशित्वात् गुरुरित्यभिधीयते ॥३१॥\\
गुकारः स्याद्गुणातीतो रूपातीतो रुकारकः ।\\
गुणरूपविहीनत्वाद्गुरुरित्यभिधीयते ॥३२॥\\
गुकारः प्रथमो वर्णो मायादिगुणभासकः ।\\
रुकारोऽस्ति परंब्रह्म  मायाभ्रान्तिविमोचकम् ॥३३॥\\
एवं गुरुपदं श्रेष्ठं देवानामपि दुर्लभम् ।\\
हाहाहूहूगणैश्चैव गन्धर्वैरपि पूजितम् ॥३४॥\\
ध्रुवं तेषां च सर्वेषां नास्ति तत्त्वं गुरोः परम् ।\\
गुरोराराधानं कार्यं स्वजीवत्वं निवेदयेत् ॥३५॥\\
आसनं शयनं वस्त्रं वाहनं भूषणादिकम् ।\\
साधकेन प्रदातव्यं गुरोः सन्तोषकारणम् ॥३६॥\\
कर्मणा मनसा वाचा नित्यमाराधयेद्गुरुम् ।\\
दीर्घदण्डं नमस्कुर्यान्निर्लज्जो गुरुसन्निधौ ॥३७॥\\
शरीरमर्थं प्राणांश्च सद्गुरुभ्यो निवेदयेत् ।\\
आत्मानमपि दास्याय वैदेहो जनको यथा ॥३८॥\\
गुरुरेव जगत्सर्वं ब्रह्मविष्णुशिवात्मकम् ।\\
गुरोः परतरं नास्ति तस्मात्तं पूजयेद्गुरुम् ॥३९॥\\
यस्यानुग्रहमात्रेण हृदि ह्युत्पद्यते क्षणात् ।\\
ज्ञानं च परमानन्दः सद्गुरुः शिव एव सः ॥४०॥\\
भस्मकीटविडं तं हि देहं स्थूलं वरानने ।\\
त्वङ्‍मूत्ररुधिरास्थीनि मलमांसादिभाजनम् ॥४१॥\\
संसारवृक्षमारूढाः पतन्तो नरकार्णवे ।\\
सर्वे येनोद्धृता लोकास्तस्मै श्री गुरवे नमः ॥४२॥\\
गुरुर्ब्रह्मागुरुर्विष्णुर्गुरुर्देवो महेश्वरः ।\\
गुरुरेव परं ब्रह्म तस्मै श्री गुरवे नमः ॥४३॥\\
अखण्डमण्डलाकारं व्याप्तं येन चराचरम् ।\\
तत्पदं दर्शितं येन तस्मै श्रीगुरवे नमः ॥४४॥\\
देहे जीवत्वमापन्नं चैतन्यं निष्कलं परम् ।\\
त्वं पदं दर्शितं येन तस्मै श्री गुरवे नमः ॥४५॥\\
अखण्डं परमार्थं सत् ऐक्यं च त्वं तदोः शुभम् ।\\
असिना दर्शितं येन तस्मै श्री गुरवे नमः ॥४६॥\\
सर्वश्रुतिशिरोरत्ननीराजितपदाम्बुजम् ।\\
वेदान्ताम्बुजसूर्याभं श्रीगुरुं शरणं व्रजेत् ॥४७॥\\
चैतन्यं शाश्वतं शान्तं मायातीतं निरञ्जनम् ।\\
नादबिन्दुकलातीतं तस्मै श्री गुरवे नमः ॥४८॥\\
स्थावरं जङ्गमं चेति यत्किञ्चिज्जगतीतले ।\\
व्याप्तं यस्य चिता सर्वं तस्मै श्री गुरवे नमः ॥४९॥\\
त्वं पिता त्वं च मे माता त्वं बन्धुस्त्वं च दैवतम् ।\\
संसारप्रीतिभङ्गाय तुभ्यं श्री गुरवे नमः ॥५०॥\\
यत्सत्तया जगत्सत्वं यत्प्रकाशेन भायुतम् ।\\
नन्दनं च यदानन्दात्तस्मै श्री गुरवे नमः ॥५१॥\\
येन चेतयताऽऽपूर्यं चित्तं चेतयते नरः ।\\
जाग्रत्स्वप्नसुषुप्त्यादौ तस्मै श्री गुरवे नमः ॥५२॥\\
यस्य ज्ञानादिदं विश्वमदृश्यं भेदभेदतः ।\\
सत्स्वरूपावशेषं च तस्मै श्री गुरवे नमः ॥५३॥\\
य एव कार्यरूपेण कारणेनापि भाति च ।\\
कार्यकारणनिर्मुक्तस्तस्मै श्री गुरवे नमः ॥५४॥\\
ज्ञानशक्तिस्वरूपाय कामितार्थप्रदायिने ।\\
भुक्तिमुक्तिप्रदात्रे च तस्मै श्री गुरवे नमः ॥५५॥\\
अनेकजन्मसम्प्राप्तकर्मकोटिविदाहिने ।\\
ज्ञानानलप्रभावेन तस्मै श्री गुरवे नमः ॥५६॥\\
न गुरोरधिकं तत्त्वं न गुरोरधिकं तपः ।\\
न गुरोरधिकं ज्ञानं तस्मै श्री गुरवे नमः ॥५७॥\\
मन्नाथः श्रीजगन्नाथो मद्गुरुः श्रीजगद्गुरुः ।\\
ममात्मा सर्वभूतात्मा तस्मै श्री गुरवे नमः ॥५८॥\\
गुरुरादिरनादिश्च गुरुः परमदैवतम् ।\\
गुरोः समानः कोवास्ति तस्मै श्री गुरवे नमः ॥५९॥\\
एक एव परो बन्धुर्विषमो समुपस्थिते ।\\
निस्स्पृहः करुणासिन्धुस्तस्मै श्री गुरवे नमः ॥६०॥\\
गुरुमध्ये स्थितं विश्वं विश्वमध्ये स्थितो गुरुः ।\\
विश्वरूपो विरूपोऽसौ तस्मै श्री गुरवे नमः ॥६१॥\\
भवारण्यप्रविष्टस्य दिङ्‍मोहभ्रान्तचेतसः ।\\
येन सन्दर्शिनः पन्थास्तस्मै श्री गुरवे नमः ॥६२॥\\
तापत्रयाग्नितप्तानां श्रान्तानां प्राणिनामुमे ।\\
गुरुरेव परागङ्गा तस्मै श्री गुरवे नमः ॥६३॥\\
हेतवे सर्वजगतां संसारार्णवसेतवे ।\\
प्रभवे सर्वविद्यानां शम्भवे गुरवे नमः ॥६४॥\\
ध्यानमूलं गुरोर्मूर्तिः पूजामूलं गुरोः पदम् ।\\
मन्त्रमूलं गुरोर्वाक्यं मोक्षमूलं गुरोः कृपा ॥६५॥\\
हरणं भवरोगस्य तरणं क्लेशवारिधे  ।\\
भरणं सर्वलोकस्य शरणं चरणं गुरोः ॥६६॥\\
शिवे रुष्टे गुरुस्त्राता गुरौ रुष्टे न कश्चन ।\\
तस्मात् परगुरुं लब्ध्वा तमेव शरणं व्रजेत् ॥६७॥\\
अत्रिणेत्रः शिवःसाक्षात् द्विभुजश्चापरो हरिः ।\\
योऽचतुर्वदनो ब्रह्मा श्री गुरुः कथितः प्रिये ॥६८॥\\
नित्याय निर्विकाराय निरवद्याय योगिने ।\\
निष्कलाय निरीहाय शिवाय गुरवे नमः ॥६९॥\\
शिष्यहृत्पद्मसूर्याय सत्याय ज्ञानस्वरूपिणे ।\\
वेदान्तवाक्यवेद्याय शिवाय गुरवे नमः ॥७०॥\\
उपायोपेयरूपाय सदुपायप्रदर्शिने ।\\
अनिर्वाच्याय वाच्याय शिवाय गुरवे नमः ॥७१॥\\
कार्यकारणरूपाय रूपारूपाय ते सदा ।\\
अप्रमेयस्वरूपाय शिवाय गुरवे नमः ॥७२॥\\
दृग्दृश्यद्रष्टृरूपाय निष्पन्ननिजरूपिणे ।\\
अपारायाद्वितीयाय शिवाय गुरवे नमः ॥७३॥\\
गुणाधाराय गुणिने गुणरूपस्वरूपिणे ।\\
जन्मिने जन्महीनाय शिवाय गुरवे नमः ॥७४॥\\
अनाद्यायाखिलाद्याय मायिने गतमायिने ।\\
अरूपाय स्वरूपाय शिवाय गुरवे नमः ॥७५॥\\
सर्वमन्त्रस्वरूपाय सर्वतन्त्रस्वरूपिणे ।\\
सर्वगाय समस्ताय शिवाय गुरवे नमः ॥७६॥\\
मनुष्यचर्मणाऽऽबद्धः साक्षात् परशिवः स्वयम् ।\\
गुरुरित्यभिधां गृह्णन् गूढः पर्यटति क्षितौ ॥७७॥\\
शिववद्दृश्यते साक्षात् श्रीगुरुः पुण्यकर्मणाम् ।\\
नरवद्दृश्यते सैव श्रीगुरुः पापकर्मणाम् ॥७८॥\\
श्रीनाथचरणद्वन्द्वं यस्यां दिशि विराजते ।\\
तस्यै दिशे नमस्कुर्यात्  भक्त्या प्रतिदिनं प्रिये ॥७९॥\\
तस्यै दिशे सततमञ्जलिरेष नित्यं\\
प्रक्षिप्यते मुखरितालियुतप्रसूनैः ।\\
जागर्ति यत्र भगवान् गुरुचक्रवर्ती\\
विश्वस्थितिप्रलयनाटकविश्वसाक्षी ॥८०॥\\
उरसा शिरसा चैव मनसा वचसा दृशा ।\\
पद्‍भ्यां कराभ्यां कर्णाभ्यां प्रणामोऽष्टाङ्ग उच्यते ॥८१॥\\
गुरोः कृपाप्रसादेन ब्रह्मविष्णुमहेश्वराः ।\\
समर्थास्तत्प्रसादो हि केवलं गुरुसेवया ॥८२॥\\
देवकिन्नरगन्धर्वाः पितरो यक्षचारणाः ।\\
मुनयो नैव जानन्ति गुरुशुश्रूषणे विधिम् ॥८३॥\\
मदाहङ्कारगर्वेण तपोविद्याबलान्विताः ।\\
संसारकुहरावर्ते पतिता घटयन्त्रवत् ॥८४॥\\
ध्यानं श्रुणु महादेवि श्रीगुरोः कथयामि ते ।\\
सर्वसौख्यकरं तद्वद्भुक्तिमुक्तिप्रदायकम् ॥८५॥\\
श्रीमत्परब्रह्मगुरुं स्मरामि श्रीमत्परब्रह्मगुरुं भजामि ।\\
श्रीमत्परब्रह्मगुरुं वदामि श्रीमत्परब्रह्मगुरुं नमामि ॥८६॥\\
ब्रह्मानन्दं परमसुखदं केवलं ज्ञानमूर्तिं\\
द्वन्द्वातीतं गगनसदृशं तत्त्वमस्यादिलक्ष्यम् ।\\
एकं नित्यं विमलमचलं सर्वधीसाक्षिभूतं\\
भावातीतं त्रिगुणरहितं सद्गुरुं तं नमामि ॥८७॥\\
आनन्दमानन्दकरं प्रसन्नं\\
ज्ञानस्वरूपं निजबोधयुक्तम् ।\\
योगीन्द्रमीड्यं भवरोगवैद्यं\\
श्रीमद्गुरुं नित्यमहं नमामि ॥८८॥\\
नित्यं शुद्धं निराभासं निराकारं निरञ्जनम् ।\\
नित्यबोधचिदानन्दं गुरुं ब्रह्मनमाम्यहम् ॥८९॥\\
हृदम्बुजे कर्णिकमध्यसंस्थे\\
सिंहासने संस्थितदिव्यमूर्तिम् ।\\
ध्यायेद्गुरुं चन्द्रकलाप्रकाशं\\
सच्चित्सुखाभीष्टवरं ददानम् ॥९०॥\\
श्वेताम्बरं श्वेतविलेपपुष्पं\\
मुक्ताविभूषं मुदितं त्रिणेत्रम् ।\\
वामाङ्कपीठस्थितदिव्यशक्तिं\\
मन्दस्मितं सान्द्रकृपानिधानम् ॥९१॥\\
यस्मिन् सृष्टिस्थितिध्वंसनिग्रहानुग्रहात्मकम् ।\\
कृत्यं पञ्चविधं शश्वद्भासते तं गुरुं भजेत् ॥९२॥\\
न गुरोरधिकं न गुरोरधिकं\\
न गुरोरधिकं न गुरोरधिकं ।\\
शिवशासनतः शिवशासनतः\\
शिवशासनतः शिवशासनतः ॥९३ ॥\\
ज्ञेयं सर्वं विलाप्येत विशुद्धज्ञानयोगतः ।\\
ज्ञातृत्वमपि चिन्मात्रे नान्यः पन्था द्वितीयकः ॥९४॥\\
यावत्तिष्ठति देहेऽसौ तावद्देवि गुरुं स्मरेत् ।\\
गुरुलोपो न कर्तव्यो निष्टितोप्यद्वये परे ॥९५॥\\
हुंकारेण न वक्तव्यं प्राज्ञैः शिष्यैः कदाचन ।\\
गुरोरग्रे न वक्तव्यमसत्यं च कदाचन ॥९६॥\\ 
गुरुं त्वं कृत्य हुं कृत्य गुर्जित्य वादतः ।\\
अरण्ये निर्जले घोरे स भवेद्‍ब्रह्मराक्षसः ॥९७॥\\
उपभुञ्जीत नो वस्तु गुरोः किञ्चिदपि स्वयम् ।\\
दत्तं ग्राह्यं  प्रसादेति प्रायो ह्येतन्न लभ्यते ॥९८॥\\
पादुकासनशय्यादि गुरुणा यदधिष्ठितम् ।\\
नमस्कुर्वीत तत्सर्वं पादाभ्यां न स्पृशेत्क्वचित् ॥९९॥\\
गच्छतः पृष्ठतो गच्छेत् गुरुपादौ न लङ्घयेत् ।\\
नोल्बणं धारयेद्वेषं नालाङ्कारास्तथोल्बणान् ॥१००॥\\
गुरुनिन्दापरं दृष्ट्वा धावयेदथ वारयेत् ।\\
स्थानं वा तत् परित्याज्यं जिह्वा च्छेदाक्षमो यदि ॥१०१॥\\
अप्रियस्य हास्यस्य नावकाशो गुरोः पुरः ।\\
न नियोग परं ब्रूयात् गुरोराज्ञां विभावयेत् ॥१०२॥\\
मुनिभ्यः पन्नगेभ्यश्च सुरेभ्यः शापतोऽपि च ।\\
कालमृत्युभयाद्वापि गुरू रक्षति पार्वति ॥१०३॥\\
नित्यं ब्रह्म निराकारं येन प्राप्तं स वै गुरुः ।\\
स शिष्यं प्रापयेत् प्राप्यं दीपो दीपान्तरं यथा ॥१०४॥\\
गुरोः कृपाप्रसादेन ब्रह्माहमिति भावयेत् ।\\
अनेन मुक्तिमार्गेण ह्यात्मज्ञानं प्रकाशते ॥१०५॥\\
संपश्येच्छ्रीगुरुं शान्तं परमात्मस्वरूपिणम् ।\\
स्थावरे जङ्गमे चैव सर्वत्र जगतीतले ॥१०६॥\\
श्रीगुरुं सच्चिदानन्दं भावातीतं विभाव्य च ।\\
तन्निदर्शितमार्गेण ध्यानमग्नो भवेत्सुधिः ॥१०७॥\\
परात्परतरं ध्यायेच्छुद्धस्फटिकसन्निभम् ।\\
हृदयाकाशमध्यस्थं स्वाङ्गुष्ठपरिमाणकम् ॥१०८॥\\
अङ्गुष्ठमात्रं पुरुषं ध्यायतश्चिन्मयं हृदि ।\\
तत्र स्फुरति यो भावः शृणु तत्कथयामि ते ॥१०९॥\\
{ विरजं परमाकाशं  ध्रुवमानन्दमव्ययम् ।\\
अगोचरं तथाऽऽगम्यं नामरूपविवर्जितम् ॥११०॥\\
तदहं ब्रह्म कैवल्यमिति बोधः प्रजायते । }\\
यथा निजस्वभावेन केयूरकटकादयः ।\\
सुवर्णत्वेन तिष्ठन्ति तथाऽहं ब्रह्मशाश्वतम् ॥१११॥\\
एवं ध्यायन् परंब्रह्म स्थातव्यं यत्र कुत्रचित् ।\\
कीटो भृङ्ग इव ध्यानात् ब्रह्मैव भवति स्वयम् ॥११२॥\\
यदृच्छया चोपपन्नं ह्यल्पं बहुळमेव वा ।\\
नीरागेणैव भुञ्जीत स्वाभ्याससमये मुदा ॥११३॥\\
एकमेवाद्वितीयोऽहं गुरुवाक्यात्सुनिश्चितम् ।\\
एवमभ्यस्यतो नित्यं न् असेव्यं वै वनान्तरम् ॥११४॥\\
अभ्यासान्निमिषेणैव समाधिमधिगच्छति ।\\
जन्मकोटिकृतं पापं तत्क्षणादेव नश्यति ॥११५॥\\
न तत्सुखं सुरेन्द्रस्य न सुखं चक्रवर्तिनाम् ।\\
यत्सुखं वीतरागस्य सदा सन्तुष्टचेतसः ॥११६॥\\
रसं ब्रह्म पिबेद्यश्च तृप्तो यः परमात्मनि ।\\
इन्द्रं च मनुते रङ्कं नृपाणां तत्र का कथा ॥११७॥\\
देशः पूतो जनाः पूतास्तादृशो यत्र तिष्ठति ।\\
तत्कटाक्षोऽथ संसर्गः परस्मै श्रेयसेप्यलम् ॥११८॥\\
देही ब्रह्म भवेदेवं प्रसादाद्ध्यानतो गुरोः ।\\
नराणां च फलप्राप्तौ भक्तिरेव हि कारणम् ॥११९॥\\
मुक्तस्य लक्षणं देवि तवाग्रे कथितं मया ।\\
गुरुभक्तिस्थता ध्यानं सकलं तव कीर्तनम् ॥१२०॥\\
गुरुगीतातिगुह्येयं मयाऽस्ति कथिता शुभा ।\\
श्रीगुरुं चिन्मयं ध्यायन् यामहं कलये सदा ॥१२१॥\\
गुरुगीतामिमां देवि शुद्धतत्त्वं मयोदितम् ।\\
गुरुं मां ध्यायती प्रेम्णा हृदि नित्यं विभावय ॥१२२॥\\
इयं चेद्भक्तिभावेन पठ्यते श्रूयतेऽथवा ।\\
लिख्यते दीयते पुम्भिर्भवेद्भवविनाशिनी ॥१२३॥\\
अनन्तफलमाप्नोति गुरुगीतजपेन तु ।\\
अस्याश्च विविधा मन्त्राः कलां नार्हन्ति षोडशीम् ॥१२४॥\\
सर्वपापप्रशमनी सर्वसङ्कटनाशिनी ।\\
सर्वसिद्धिकरी चेयं सर्वलोकवशङ्करी ॥१२५॥\\
दुःस्वप्ननाशिनी चेयं सुस्वप्नफलदायिनी ।\\
रिपूणां स्तम्भनी गीता वाचस्पत्यप्रदायिनी ॥१२६॥\\
कामिनां कामधेनुश्च सर्वमङ्गलकारिणी ।\\
चिन्तामणिश्चिन्तितस्य श्लोके श्लोके च सिद्धिदा ॥१२७॥\\
मोक्षकामो जपेन्नित्यं मोक्षश्रियमवाप्नुयात् ।\\
पुत्रकामो लभेत्पुत्रान् श्रीकामश्चामितां श्रियम् ॥१२८॥\\
त्रिवारपठनात्सद्यः कारागाराद्विमुच्यते ।\\
नित्यपाठाद्भवेच्छस्त्री पुत्रिणी सुभगा चिरम् ॥१२९॥\\
अकामतः स्त्री विधवा जपेन्मोक्षमवाप्नुयात् ।\\
अवैधव्यं सकामा चेल्ल्भते चान्यजन्मनि ॥१३०॥\\
जपेच्छाक्ताश्च सौरश्च गाणापत्यश्च वैष्णवः ।\\
शैवश्च सिद्धिदामेताम् सर्वदेवस्वरूपिणीम् ॥१३१॥\\
तीर्थे बिल्वतरोर्मूले वटमूले सरित्तटे ।\\
देवालये च गोष्ठे च मठे बृन्दावने तथा ।\\
पवित्रे निर्मले स्थाने जपः शीघ्रफलप्रदः ॥१३२॥\\
शान्त्यर्थे धारयेच्छुक्लं वस्त्रं वश्येऽथ रक्तिमम् ।\\
आभिचारे नीलवर्णं पीतवर्णं धनागमे ॥१३३॥\\
गुरुभक्तो भवेच्छीघ्रं गुरुगीताजपेन तु ।\\
धन्या माता पिता धन्यो धन्या वंश्या जना अपि ।\\
धन्या च वसुधा यत्र गुरुभक्तः प्रजायते ॥१३४॥\\
इदं रहस्यं नो वाच्यं यस्मै कस्मैचन प्रिये ।\\
अभक्ते वञ्चके धूर्ते पाषण्डे नास्तिके तथा ।\\
मनसाऽपि न वक्तव्या गुरुगीता कदाचन ॥१३५॥\\
अत्यन्तपक्वचित्तस्य श्रद्धाभक्तियुतस्य च ।\\
प्रवक्तव्या प्रयत्नेन ममात्मा प्रीयते तदा ॥१३६॥\\
गुरवो बहवो सन्ति शिष्यवित्तापहारकाः ।\\
दुर्लभः स गुरुर्लोके शिष्यसन्तापहारकः ॥१३७॥\\
ज्ञानहीनो गुरुं मन्यो मिथ्यावदी विडम्बकः ।\\
स्वविश्रान्तिं न जानाति परशान्तिं करोति किम् ॥१३८॥\\
स्वयं तरितुमक्षमः परान्निस्तारयेत्कथम् ।\\
दूरे तं वर्जयेत् प्राज्ञो धीरमेव समाश्रयेत् ॥१३९॥\\
सच्चिदानन्दरूपाय व्यापिने परमात्मने ।\\
नमः श्रीगुरुनाथाय प्रकाशानन्दमूर्तये ॥१४०॥\\
सच्चिदानन्दरूपाय कृष्णाय क्लेशहारिणे ।\\
नमो वेदान्तवेद्याय गुरवे बुद्धिसाक्षिणे ॥१४१॥\\
यस्य प्रसादादहमेव विष्णुर्मय्येव सर्वं परिकल्पितं च ।\\
इत्थं विजानामि सदात्मतत्त्वं तस्याङ्घ्रिपद्मं प्रणतोऽस्मि नित्यम् ॥१४२॥\\
यस्यान्तं नादिमध्यं नहि करचरणं नाम गोत्रं न सूत्रं\\
नो जातिर्नैव वर्णो न भवति पुरुषो नानपुंसं न च स्त्री ।\\
नाकारं नो विकारं नहि जनि मरणं नैव पुण्यं न पापं\\
नो तत्त्वं तत्त्वमेकं सहजसमरसं सद्गुरुं तं नमामि ॥१४३॥


\begin{center}॥ ओं तत्सत् श्रीदत्तात्रेयपरब्रह्मार्पणमस्तु ॥\end{center}

\end{document}
